% lista 6(calculo 1)
\documentclass[latin,20pt]{article}
%\usepackage{amssymb,latexsym,amsthm,amsmath}
\usepackage[paper=a4paper,hmargin={1cm,1cm},vmargin={1.5cm,1.5cm}]{geometry}
\usepackage{amsmath,amsfonts,amssymb}
\usepackage[utf8]{inputenc}

%\usepackage{stmaryrd} %%para graficar maximo inteiro 
\begin{document}

\title{Lista 6: Cálculo I }
 
\author{
A. Ramos \thanks{Department of Mathematics,
    Federal University of Paraná, PR, Brazil.
    Email: {\tt albertoramos@ufpr.br}.}
}

\date{\today}
 
\maketitle

\begin{abstract}
{\bf Lista em constante atualização}.
 \begin{enumerate}
 \item Aplicações da derivada;
 \item Expansão de Taylor.   
 \end{enumerate}
\end{abstract}

%%%%%%%%%%%%%%%%%%%%%%%%%%%%%%%%%%%%%%%%%%%%%%  
%\section*{Elipse} 
%Seja $\mathcal{O}$ um aberto em $\mathbb{R}^{n}$. 
%Denote por 
%$C^{1,1}_{L}(\mathcal{O})$ o conjunto das funções deriváveis 
%em $\mathcal{O}$ cuja derivada é Lipschitziana com constante de 
%     Lipschitz $L$ em $\mathcal{O}$, isto é, 
%     $\|\nabla f(x)-\nabla f(y)\|\leq L\|x-y\|$, 
%     para todo $x,y \in \mathcal{O}$.
 
  \section{Exercícios}   
 
 Faça do livro texto, os exercícios correspondentes aos temas desenvolvidos em aula. 
  
  \section{Exercícios adicionais}
    \subsection{Aplicações da derivada}
     \begin{enumerate}
     \item Determine as assíntotas e faça o gráfico de 
     $f(x):=\sqrt{9x^{2}+2x+2}$.
     \item Uma partícula P move-se sobre a a curva 
     $y^{2}=x$, $x>0$ e $y>0$. Se a abscissa $x$ 
     está variando com uma aceleração que é o dobro 
     do quadrado da velocidade da ordenada $y$. 
     Mostre que a aceleração da ordenada é nula. 
     \item Um ponto $Q=(x,y)$ está fixo a uma roda 
     de raio de 1m e de centro $O$, que gira 
     sem escorregamento sobre o eixo $x$ cujo ponto de contato é $P$. 
     Suponha que a roda gira com uma velocidade angular 
     constante $1 \ \ rad/s$. 
     Escreva as velocidades da abscisa e da 
     ordenada em função de $\theta=\angle (OP, OQ)$. 
     {\it Rpta: } $dx/dt=1-\cos(\theta)$ e 
     $dy/dt=\sin(\theta)$.
      \item Sejam $P$, $Q$ dois polinômios tal que 
      $P(a)=Q(a)=0$
      e $Q'(a)\neq 0$. Verifique que 
      $$   \lim_{x \rightarrow a}\frac{P(x)}{Q(x)}=
      \frac{P'(a)}{Q'(a)}.$$
      \item Calcule os limites
        \begin{enumerate}
        \item 
        $$ \lim_{x \rightarrow 0} 
        \frac{\ln(x+1)}{x^2+\sin(x)}=1; $$
        \item 
        $$ \lim_{x \rightarrow -1} 
        \frac{x(x+1)^{1/3}}{\sin(\pi x^2)}=-\infty; $$
        \item 
        $$ \lim_{x \rightarrow 0} 
        \frac{x-1+e^{-x^2}}{x^5-1+e^{4x}}=\frac{1}{4}. $$
        \end{enumerate} 
      \item Faço o gráfico de uma função contínua $f$ defina em 
      $\mathbb{R}\setminus\{-5,-1,4\}$ tal que satisfaz
      $(a) \ \  \lim_{x \rightarrow 4} f(x)=\infty$; \ \  
      $(b) \ \ \lim_{x \rightarrow -5^{+}} f(x)=-\infty$; \ \ 
      $(c) \ \ \lim_{x \rightarrow -5^{-}} f(x)=+\infty$; \ \ 
      $(d) \ \ \lim_{x \rightarrow -1} f(x)=0$; \ \ 
      $(e) \ \ f(-7)=-2$; \ \ 
      $(f) \ \ f(1)=5$ e 
      $(g) \ \ f(7)=-2$.
      \item Sejam $f$ e $g$ duas funções definidas em $\mathbb{R}$
      com $g$ contínua em $x=0$. Se $f(x)=xg(x)$, 
      para todo $x \in \mathbb{R}$. 
      Verifique que $f$ é derivável em $0$.
      \item Suponha que $f$ é uma função contínua em $[a,b]$ tal que $f'(x)=1$ para todo $x \in (a,b)$. Mostre que 
      $f(x)=x-a+f(a)$ para todo $x \in [a,b]$.
      \item Faça os gráficos das seguintes funções. Para isso
      determine os pontos críticos, os extremos relativos, os intervalos de crescimento/decrescimento, os pontos de 
      inflexão, a concavidade e as assíntotas.
        \begin{enumerate}
        \item  $$  f(x)=(x-1)^{2}(2x+2)^{3}. $$	
        \item  $$  f(x)=\exp(-x^2).         $$ 
        \item  $$  f(x)=\frac{x^4+1}{x^2}.  $$
        \item  $$  f(x)=\frac{3+x^2}{x-1}.  $$
        \item  $$  f(x)=\frac{x}{\sqrt{x^2+7}}.  $$
        \item  $$  f(x)=\frac{3x}{\sqrt[3]{x^2-1}}.  $$
        \item  $$  f(x)=\left(\frac{5x}{8-2x}\right)e^{-x}.  $$
        \end{enumerate} 
        \item Suponha que $f$ é derivável até a $3$ra-ordem 
        no intervalo aberto $I$ e $a \in I$. 
        Se $f^{(2)}(a)=0$, $f^{(3)}(a)\neq 0$ e $f^{(3)}$
        é contínua em $a$. Mostre que $x=a$ é 
        um ponto de inflexão. 
        \item Um campo retangular à margem dum rio deve ser cercado, com a excepção do lado ao longo do rio. O custo do material é de 12 reais por metro no lado paralelo ao rio e de 8 reais por metro nos lados transversais. Ache o campo de maior área possível que possa ser cercado com 3600 reais de material.
        {\it Rpta: }  Área máxima: 16. 875 $m^{2}$. 
        Lado paralelo$=$150 $m$. 
        \item Um cilindro circular reto deve ser inscrito numa esfera de raio $R$. Encontre a razão entre a altura e o raio da base do cilindro cuja área da superfície lateral seja o máximo possível. {\it Rpta: } Se $h=$altura e $r=$raio da base. Então 
        $h=2r$.
        \item Encontre os valores de $a$ e $b$ para que 
        $f(x)=2x^{3}+ax^{2}+b$ tenha um extremo relativo no ponto $(1,-2)$ que pertence ao gráfico de $f$. {\it Rpta: } $a=-3$, $b=-1$.        
        \item Ao construir uma sala de cinema, se estima que se houver de 40 a 80 assentos, o lucro bruto diário é de 16 reais por assento. Mas, se o número de assentos for acima de 80, 
        o lucro diário por assento decresce de 0.08 reais vezes o número de lugares acima de 80. 
        Qual deve ser o número de assentos para que o lucro bruto diário seja o máximo possível? {\it Rpta: } Número de assento deve ser 140.      
        \item Mostre que o triângulo isósceles de área máxima que pode se inscrever numa circunferência é uma triângulo equilátero.      
        \item Dois aviões A e B voam horizontalmente à mesma altitude. 
        O avião B encontra-se ao sudoeste do avião A e 20 $km$ ao oeste e 20 $km$ ao sul de A. Se o avião A está viajando para o oeste a 16 $km/min$ e o avião B está viajando pra o norte a 64/3 $km/min$, 
        (a) em quantos segundos eles estarão mais perto um do outro; 
        (b) qual será a menor distância entre eles?   
      \end{enumerate}      
\end{document}










    \subsection{Regras de cálculos para limites}   
   Calcule os seguintes limites.  
    \begin{enumerate}
    \item $\lim_{u \rightarrow 1} 
    \frac{\sqrt{3+u^2}-2}{1-u}=-\frac{1}{2}$.
    \item $\lim_{t \rightarrow 4} 
    \frac{3-\sqrt{5+t}}{1-\sqrt{5-t}}=-\frac{1}{3}$.
    \item $\lim_{x \rightarrow 1} 
    \frac{\sqrt{3x-2}+\sqrt{x}-\sqrt{5x-1}}
    {\sqrt{x}-\sqrt{2x-1}}=-\frac{3}{2}$.
    \item Se $f(x)=\sqrt{1+3x}$. Calcule 
    $\lim_{h \rightarrow 0} 
    \frac{f(x+h)-f(x)}
    {h}=\frac{3}{2\sqrt{3x+1}}$.
    \item $\lim_{x \rightarrow 1} 
    \frac{x^{100}-2x+1}
    {x^{50}-2x+1}=\frac{49}{24}$.
    \item $\lim_{x \rightarrow 2} 
    \frac{\sqrt{1+\sqrt{2+x}}-\sqrt{3}}
    {x-2}=\frac{1}{8\sqrt{3}}$.
     \item Se 
    $\lim_{x \rightarrow 0} 
    \frac{f(x)-1}
    {x}=1$. Prove $\lim_{x \rightarrow 0} 
    \frac{f(ax)-f(bx)}
    {x}=a-b$. {\it Dica:} Considere se $a$ e $b$ são iguais a zero ou não.
    \item Dado $a \in \mathbb{R}$. Mostre que $\lim_{x \rightarrow a} 
    \frac{x\sqrt{x}-a\sqrt{a}}
    {\sqrt{x}-\sqrt{a}}=3a$.
    \end{enumerate}
    \subsection{Limites laterais}   
   Calcule, se existe, os seguintes limites.  
    \begin{enumerate}
    \item $$\lim_{x \rightarrow \frac{5}{2}} 
    \sqrt{|x|+\lbrack\!\lbrack 3x \rbrack\!\rbrack}. $$ Sim, e o limite é $\sqrt{19/2}$.
     \item $$\lim_{x \rightarrow \frac{7}{3}} 
    \sqrt{|x|+\lbrack\!\lbrack 3x \rbrack\!\rbrack}. $$ Não.
    \item $$\lim_{x \rightarrow -3} 
    \frac{\lbrack\!\lbrack x-1 \rbrack\!\rbrack -x}
    {
    \sqrt{|x|^2-\lbrack\!\lbrack x \rbrack\!\rbrack}
    }.$$ Não.
     \item $$\lim_{x \rightarrow 1} 
    \frac{\lbrack\!\lbrack x\rbrack\!\rbrack^{2} -x^{2}}
    {
    \lbrack\!\lbrack x\rbrack\!\rbrack^{2} -x
    }.$$ Não.
    \item Considere a função 
     $$
    f(x)= \left\{  
            \begin{array}{lll}
    &\frac{x^3+3x^2-9x-27}{x+3} &\text{, se } x \in (-\infty, -3) \\
    &ax^2-2bx+1     &\text{, se } x \in [-3,3] \\
            & \frac{x^2-22x+57}{x-3}     &\text{, se } x \in (3,\infty) \\
            \end{array}
            \right. 
    $$
    Para quais valores de $a$ e $b$, existe os limites de $f$ em $x=-3$ e $x=3$? {\it Rpta: } $a=-1, b=4/3$.
    \end{enumerate}
    \subsection{Limites Trigonométricos}   
   Calcule os seguintes limites.  
    \begin{enumerate}
    \item $$\lim_{x \rightarrow \pi} \frac{1-\sin(\frac{x}{2})}{x-\pi}=0$$
    \item $$\lim_{x \rightarrow 0} \frac{x-\sin(x)}{x^2}=0. $$
    \item $$\lim_{x \rightarrow 0} \frac{1-\sqrt{\cos(x)}}{x^2}=\frac{1}{4}$$
    \item $$\lim_{x \rightarrow 1} 
    \frac{\cos(\frac{\pi x}{2})}{1-\sqrt{x}}=\pi$$
    \item $$\lim_{x \rightarrow 0} 
    \frac{1-\cos^7(x)}{x^2}=\frac{7}{2}$$
    \item $$\lim_{x \rightarrow 0} 
    \frac{1-\cos(1-\cos(x))}{x^4}=\frac{1}{8}$$
    \item $$\lim_{x \rightarrow \frac{\pi}{4}} 
    \frac{\sin(2x)-\cos(2x)-1}{\sin(x)-\cos(x)}=\sqrt{2}$$
    \item $$\lim_{x \rightarrow 1} 
    \frac{\sin(\pi x)+\cos(\frac{\pi x}{2})}
    {\tan(\frac{\pi x}{4})-1}=-\frac{3 \pi}{4}$$
    \end{enumerate}
    \subsection{Definição de limite}   
   Usando a definição de limite. Prove que 
    \begin{enumerate}
    \item $\lim_{x \rightarrow 2} \frac{x^2+1}{x-1}=5$.
    \item $\lim_{x \rightarrow 3} 
    \frac{1}{x^2+16}=\frac{1}{25}$. 
    \item $\lim_{x \rightarrow \frac{1}{2}} x^{2}\lbrack\!\lbrack x+2 \rbrack\!\rbrack=\frac{1}{2}$. 
    \item  $\lim_{x \rightarrow 4} 
    \frac{x^3-15x-4}{x-3}=0$.
    \item $\lim_{x \rightarrow a}\cos(x)=\cos(a)$, para qualquer $a \in \mathbb{R}$.  
    \end{enumerate}
     \subsection{Teorema de confronto e variantes}   
    \begin{enumerate}
    \item Se $f:\mathbb{R}\rightarrow \mathbb{R}$ uma função tal que $|f(x)|\leq 3|x|$, $\forall x \in \mathbb{R}$.
    Calcule $\lim_{x \rightarrow 0} \frac{f(x^3)}{x}$.
    \item Considere duas funções 
    $f,g:\mathbb{R}\rightarrow \mathbb{R}$ com a propriedade que $|\sin(x)|\leq g(x)\leq 4|x|$ e 
    $0 \leq f(x)\leq 1+|\sin(1/x)|$ para todo $x \neq 0$. 
    Calcule $\lim_{x \rightarrow 0} (f(x)g(x)+\cos(x))$.
    {\it Rpta: } 1.
    \item Se $f:\mathbb{R}\rightarrow \mathbb{R}$ uma função tal que $1+x^2+\frac{x^6}{3}\leq f(x)+ 1 \leq 
    \sec{x^2}+\frac{x^6}{3}$, para todo $x \in \mathbb{R}$. Calcule
    $$
    \lim_{x\rightarrow 0}f(x) \text{ e } 
    \lim_{x\rightarrow 0}f(x)\cos\left(\frac{1}{x^2+1}\right).
    $$ {\it Rpta: } Ambos são zeros. 
    \end{enumerate}
\end{document}

  
