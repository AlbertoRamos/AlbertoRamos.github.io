% lista 2(topology)
\documentclass[latin,20pt]{article}
%\usepackage{amssymb,latexsym,amsthm,amsmath}
\usepackage[paper=a4paper,hmargin={1cm,1cm},vmargin={1.5cm,1.5cm}]{geometry}
\usepackage{amsmath,amsfonts,amssymb}
\usepackage[utf8]{inputenc}

%\usepackage{stmaryrd} %%para graficar maximo inteiro 
\begin{document}

\title{Lista 2: Introdução à Topologia }
 
\author{
A. Ramos \thanks{Department of Mathematics,
    Federal University of Paraná, PR, Brazil.
    Email: {\tt albertoramos@ufpr.br}.}
}

\date{\today}
 
\maketitle

\begin{abstract}
{\bf Lista em constante atualização}.
 \begin{enumerate}
 \item Espaços metricos (definições básicas, completitude e compacidade)
 \end{enumerate}
\end{abstract}

%%%%%%%%%%%%%%%%%%%%%%%%%%%%%%%%%%%%%%%%%%%%%%  
%\section*{Elipse} 
%Seja $\mathcal{O}$ um aberto em $\mathbb{R}^{n}$. 
%Denote por 
%$C^{1,1}_{L}(\mathcal{O})$ o conjunto das funções deriváveis 
%em $\mathcal{O}$ cuja derivada é Lipschitziana com constante de 
%     Lipschitz $L$ em $\mathcal{O}$, isto é, 
%     $\|\nabla f(x)-\nabla f(y)\|\leq L\|x-y\|$, 
%     para todo $x,y \in \mathcal{O}$.
  
  \section*{Royden}
  \subsection*{Capítulo 9}
    \begin{enumerate}
    \item {\bf Section 9.1} 2, 6, 10;
    \item {\bf Section 9.2} 14, 15;
    \item {\bf Section 9.3} 25, 26, 29, 32, 33;
    \item {\bf Section 9.4} 37, 38, 39, 43, 44, 45, 46, 47, 49;
    \item {\bf Section 9.3} Prove Teorema 22, 51, 55, 56, 57, 58, 59, 60, 62, 64, 65, 67, 68, 69, 72, 73; 
    \item {\bf Section 9.5} 74, 76, 77, 78.
    \end{enumerate}
  \section*{Elon Lima}
  \subsection*{Topology (III)} 
   \begin{enumerate}
    \item {\bf Capítulo 2} Proposição 8, 3, 5, 11, 23;
    \item {\bf Capítulo 3} 8, 11, 53, , 55;
    \item {\bf Capítulo 5} Proposição 20, 2, 13, 25, 33;
    \item {\bf Capítulo 6} 1, 4;
    \item {\bf Capítulo 7} Mostre que os espaços $\ell^{p}$ são completos ($p\geq 1$), 1, 7, 9, 13, 17, 18, 21; 
    \item {\bf Capítulo 8} Exemplo 18, 1, 3, 8, 35.
    \end{enumerate}
\end{document}
















  












