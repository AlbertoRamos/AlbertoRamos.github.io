% lista 2 (analise II)
\documentclass{article}
\usepackage{amssymb,latexsym,amsthm,amsmath}
\usepackage{tikz}
\usepackage{verbatim}
\usepackage[brazil]{babel}
%\usepackage[latin1]{inputenc}
% parece que são conflictantes 
\usepackage[utf8]{inputenc}
\usepackage{amsfonts}
% \usepackage{showlabels}
\usepackage{latexsym}
%%%%%%%%%%%%%%%%%%%%%%%%%
\usepackage{color, colortbl}
\usepackage{tabularx,colortbl}
\usepackage{hyperref}
\usepackage{graphicx}
%%%%%%%%%%%%%%%%%%%%%%
\theoremstyle{plain}
\newtheorem{theorem}{Theorem}[section]
\newtheorem{corollary}[theorem]{Corollary}
\newtheorem*{main}{Main~Theorem}
\newtheorem{lemma}[theorem]{Lemma}
\newtheorem{proposition}[theorem]{Proposition}
\newtheorem{algorithm}{Algorithm}[section]
\theoremstyle{definition}
\newtheorem{definition}{Definition}[section]
\newtheorem{example}{Example}[section]
\newtheorem{counter}{Counter-Example}[section]

\theoremstyle{remark}
\newtheorem{remark}{Remark}

\headheight=21.06892pt
\addtolength{\textheight}{3cm}
\addtolength{\topmargin}{-2.5cm}
\setlength{\oddsidemargin}{-.4cm}
%\setlength{\evensidemargin}{-.5cm}
\setlength{\textwidth}{17cm}
%\addtolength{\textwidth}{3cm}
\newcommand{\R}{{\mathbb R}}

%%%%%%% definição de integral superior e inferior 
\def\upint{\mathchoice%
    {\mkern13mu\overline{\vphantom{\intop}\mkern7mu}\mkern-20mu}%
    {\mkern7mu\overline{\vphantom{\intop}\mkern7mu}\mkern-14mu}%
    {\mkern7mu\overline{\vphantom{\intop}\mkern7mu}\mkern-14mu}%
    {\mkern7mu\overline{\vphantom{\intop}\mkern7mu}\mkern-14mu}%
  \int}
\def\lowint{\mkern3mu\underline{\vphantom{\intop}\mkern7mu}\mkern-10mu\int}


\begin{document}

\title{Lista 2: Análise II}

\author{
A. Ramos \thanks{Department of Mathematics,
    Federal University of Paraná, PR, Brazil.
    Email: {\tt albertoramos@ufpr.br}.}
}

\date{\today}
 
\maketitle

\begin{abstract}
{\bf Lista em constante atualização}.
\begin{enumerate}
	\item Teorema Fundamental do Cálculo 
	\item Caracterização de funções integráveis
	\item Integrais improprias.
\end{enumerate}
\end{abstract}


\section{Teorema Fundamental do Cálculo e medida nula  } 

\begin{enumerate}
  \item Faça os problemas 7-13 do capítulo IX do livro texto. 
  \item Prove que toda função monótona é integrável.
  \item Calcule a derivada de $F(x)=\int_{\sin x}^{x^3} e^{-t}dt$
  \item Seja f uma função integrável em $[a,b]$
  cuja integral não é nula. Mostre que existe um 
  $c \in (a,b)$
  tal que $\int_{a}^{c} f(x)dx=\int_{c}^{b} f(x)dx$.
  \item Seja f uma função positiva, continua e estritamente crescente em $[a,b]$. Prove que 
  $$ \int_{a}^{b} f(x)dx+\int_{f(a)}^{f(b)} f^{-1}(s)ds=bf(b)-af(a).$$
  Com essa identidade, calcule $\int_{0}^{1} \sin^{-1}(x)dx$. 
  \item Seja $f$ uma função periódica com período $m$ e integrável 
  em $[0,m]$. Defina $g(x)=\int_{0}^{x} f(t)dt$. Mostre que 
  $g$ pode não ser periódica  mas existe um $c \in \mathbb{R}$ 
  tal que $g(x)-cx$ é periódica com período $m$
  \item Seja $f$ uma função contínua em $[a,b]$. Assuma que existe constantes $\alpha$ e $\beta$ tal que 
  $$ \alpha \int_{a}^{c}f(x)dx+\beta \int_{c}^{b} f(x)dx=0, \text{para todo } c \in [a,b].$$
  Mostre que $f$ dever ser a função nula.
  \item Se $f>0$ e $f$ contínua em $[0,\infty)$. Mostre que 
  se $\int_{1}^{x}f(t)dt \leq f(x)^{2}$. Então, 
  $f(x)\geq 1/2(x-1)$.  
  \item Mostre que $$ \int_{a}^{b} xf^{''}(x)dx=(bf^{'}(b)-b)-(af^{'}(a)-a), $$ 
 onde assumimos que $f^{''}$ é contínua em $[a,b]$.
  \item Se $f:[0,1]\rightarrow \mathbb{R}$ é contínua. Prove que 
  $\int_{0}^{1} f(x)dx=\lim_{n \rightarrow \infty} \frac{1}{n}\sum_{i=1}^{n}f(\frac{i}{n})$.
  Assim, a integral pode ser interpretada como uma {\it média} de $f$ em $[a,b]$.  
  \item Seja $f$ continuamente diferenciável. 
  Defina $a_{n}:=\int_{a}^{b} f(t)\sin(nt)dt$. Prove que $a_{n}$ converge a $0$.
  {\it Dica: } Integração por partes.
  \item Seja $f$ limitada em $[a,b]$. Suponha que existe uma sequencia de partições $P_{n}$ tais que  
  $S(f, P_{n})-s(f,P_{n})\rightarrow 0$. Mostre que $f$ é integrável. É necessário que $|P_n|\rightarrow 0$?
  \item Use o teorema de valor médio para provar que para 
  todo $-1<a\leq 1$, $a_{n}:=\int_{0}^{a} \frac{x^{n}}{1+x}dx$
  converge para $0$, quando $n \rightarrow \infty$. 
  \item Mostre que a função $f$ é integrável onde $f$ é a função $f:[0,1]\rightarrow \mathbb{R}$ definida como 
  $f(x)=2^{-n}$, se $x=j/2^{n}$ com 
  $j \in \mathbb{N}$, $0\leq j<2^{n}$ e $f(x)=0$ caso contrário.
  \item Demonstre a {\it fórmula de Euler}. Seja $f:[a,b] \rightarrow \mathbb{R}$ de classe $C^{1}$. Então:
  $$  \sum_{a \leq n \leq b} f(n)=\int_{a}^{b} f(x)dx+\int_{a}^{b} \{x\}f'(x)dx+\{x\}f(x)|_{a}^{b}, $$
  onde $\{x\}:=x-[x]$ a parte não inteira de $x$.
  \item Verifique que:
   \begin{itemize}
   	\item $\frac{1}{n} \sum_{i=1}^{n} \sin(\frac{i\pi}{n}) \rightarrow \frac{2}{\pi}$, 
   	quando $n \rightarrow \infty$.
   	\item $\frac{1}{n^{p+1}} \sum_{i=1}^{n} i^{p} \rightarrow \frac{1}{p+1}$, quando $n \rightarrow \infty$ para todo 
   	$p \neq -1$.
   	O que acontece se $p=-1$?
   	\item $\sum_{i=1}^{n} \frac{1}{i^{s}}=\frac{1}{n^{s-1}}+s \int_{1}^{n}\frac{[x]}{x^{s+1}}dx$, 
   	para qualquer $s\neq1$, onde $[x]$ é o maior inteiro menor ou igual a $x$.
   \end{itemize}
  \item Se $X \subset [a,b]$ tem medida nula então $f(X)$ tem medida nula, se 
  $f$ é localmente Lipschitz em $[a,b]$ (em particular, se $f$ é de classe $C^{1}$).
  Dê um exemplo que $f(X)$ não seja de medida nula mesmo que $X$ seja de medida nula.
  \item Seja $f$ contínua em $[a,b]$ e $g$ uma função integrável em $[c,d]$ com 
  $g([c,d])\subset [a,b]$.
  Mostre que $f \circ g$ é integrável.
  \item Sabemos que o conjunto de Cantor $\mathcal{C}$ tem medida nula. Podemos construir um conjunto de Cantor 
  com medida não nula.  
  \item Considere $f$ uma função contínua em $[a,b]$ e de classe $C^{1}$ em $(a,b)$. 
  Assuma que $f'(x)\neq 0$, $\forall x \in (a,b)$. Mostre que 
  $g \circ f$ é integrável, se $g:[c,d] \rightarrow \mathbb{R}$ é integrável
  e $f[a,b] \subset [c,d]$.  
\end{enumerate}
{\bf Integral impróprias em intervalos ilimitados}
 \begin{itemize}
 	\item {\it Integral impróprias sobre $[a, \infty)$}.
 	Seja $f$ contínua em $[a,\infty)$. Definimos
 	$$\int_{a}^{\infty} f(x)dx:=\lim_{b \rightarrow \infty} \int_{a}^{b} f(x)dx, $$
 	quando o limite existir. Neste caso, a integral converge, caso contrário, dizemos que a integral diverge.
 	Caso $\int_{a}^{\infty} |f(x)|dx$ convergir, dizemos que a {\it integral é absolutamente convergente}.
 	Quando $\int_{a}^{\infty} f(x)dx$ converge mas 
 	$\int_{a}^{\infty} |f(x)|dx$ diverge, dizemos que 
 	a integral imprópria é {\it condicionalmente convergente}. Similarmente para as seguinte definições.
 	\item  {\it Integral impróprias sobre $(-\infty, b]$}.
 	Seja $f$ contínua em $(-\infty,b]$. Definimos
 	$$\int_{-\infty}^{b} f(x)dx:=\lim_{a \rightarrow -\infty} \int_{a}^{b} f(x)dx, $$
 	quando o limite existir. Neste caso, a integral converge, caso contrário, dizemos que a integral diverge.
 	\item  {\it Integral impróprias sobre $(-\infty, \infty)$}.
 	Seja $f$ contínua em $\mathbb{R}$. Definimos
 	$$\int_{-\infty}^{\infty} f(x)dx:=\lim_{a, b \rightarrow \infty} \int_{-a}^{b} f(x)dx, $$
 	quando o limite existir. Neste caso, a integral converge, caso contrário, dizemos que a integral diverge.
 	Observe que $\int_{-\infty}^{\infty} f(x)dx$, não é necessariamente igual a 
 	$\lim_{c \rightarrow \infty} \int_{c}^{c} f(x)dx$, por exemplo, no caso $\int_{-\infty}^{\infty} xdx$.
 O limite $\lim_{c \rightarrow \infty} \int_{c}^{c} f(x)dx$ (caso existir) é chamado 
 {\it Valor médio de Cauchy} é denotado por 
 $PV \int_{-\infty}^{\infty} f(x)dx$.
 \end{itemize}
 {\bf Integral impróprias em intervalos finitos}
  \begin{itemize}
  	\item Seja $f$ contínua em $(a,b]$ e $|\lim_{x \rightarrow a^{+}}f(x)|=\infty$.
  	Definimos
  	$$\int_{a}^{b} f(x)dx:=\lim_{\delta \rightarrow a^{+}} \int_{\delta}^{b} f(x)dx, $$
  	quando o limite existir. Neste caso, a integral converge, caso contrário, dizemos que a integral diverge.
  	Caso $\int_{a}^{\infty} |f(x)|dx$ convergir, dizemos que a {\it integral é absolutamente convergente}.
  	\item Seja $f$ contínua em $[a,b)$ e $|\lim_{x \rightarrow b^{-}}f(x)|=\infty$.
  	Definimos
  	$$\int_{a}^{b} f(x)dx:=\lim_{\delta \rightarrow b^{-}} \int_{a}^{\delta} f(x)dx, $$
  	quando o limite existir. Neste caso, a integral converge, caso contrário, dizemos que a integral diverge.
  	Caso $\int_{a}^{\infty} |f(x)|dx$ convergir, dizemos que a {\it integral é absolutamente convergente}.
    \item  Seja $f$ contínua em $[a,b]$ exceto num ponto $c\in(a,b)$ e se um ou ambos limites 
    laterais são infinitos. Definimos
    $$\int_{a}^{b} f(x)dx:=\int_{a}^{c} f(x)dx+ \int_{a}^{c} f(x)dx, $$
    {\it desde que ambas integrais impróprias à direitas sejam convergentes}.
 \end{itemize}
Um teorema útil para decidir se certa integral imprópria converge, é o teste de comparação. \newline

  {\bf Teorema de Comparação: } Sejam dois funções contínuas $f$ e $g$ tais que 
  $0\leq f(x)\leq g(x)$, para todo $x \in [a,\infty)$, onde $a \in \mathbb{R}$.
   \begin{itemize}
   	\item Se $\int_{a}^{\infty} g(x)dx$ converge, 
   	então  $\int_{a}^{\infty} f(x)dx$ converge.
   	\item Se $\int_{a}^{\infty} f(x)dx$ diverge, 
   	então  $\int_{a}^{\infty} g(x)dx$ diverge. 
   \end{itemize}
  Verifique que as seguinte integrais impróprias são convergentes ou não.
  \begin{enumerate}
  	\item $\int_{0}^{2} x^{-3}dx$ (diverge), $\int_{0}^{\infty} (1+x^2)^{-1}dx$ (converge, $\pi/2$)
  	\item Para quais valores de $\alpha$, a integral converge $\int_{1}^{\infty} x^{-\alpha}dx$? (converge, $\alpha>1$)
  	\item $\int_{1}^{\infty} \frac{dx}{x^{2}(1+e^{x})}$ (converge),  $\int_{1}^{\infty} \frac{(x+1)}{\sqrt{x^{3}}}dx$
  	(diverge), $\int_{0}^{\infty} x\sin x dx$ (diverge)
  	\item $\int_{0}^{\infty} e^{-ax}\sin(bx)dx$ (converge, valor $b/(a^2+b^2)$)
  	\item Determine o valor de $k$, para que 
  	$\int_{-\infty}^{\infty} e^{k|x|}dx=0.5$, ($k=-4$).
  \end{enumerate}
  
 O Critério de Dirichlet é um teste para provar 
 a convergência de séries numéricas ou integrais impróprias 
 da forma $\sum_{n=1}^{\infty} a_{n}b_{n}$
 ou $\int_{a}^{\infty} f(x)g(x)dx$.  \newline
 
  {\bf Critério de Dirichlet para integrais}
    Sejam $f$ e $g$ funções tais que 
    \begin{itemize}
    \item $f$ é contínua em $[a,\infty)$ e existe um $M>0$ tal que  
    $|\int_{a}^{x}f(s)ds|\leq M$, para todo $x>a$
    \item $g \in C^{1}[a,\infty)$ com $g(x)>0$, $g'(x)\leq 0$, $\forall x>a$
    \item $\lim_{x\rightarrow \infty} g(x)=0$.
    \end{itemize}
   Então, a integral imprópria converge 
   $\int_{a}^{\infty} f(x)g(x)dx=\lim_{b \rightarrow \infty} 
   \int_{a}^{b} f(x)g(x)dx$.   
  {\it Obs} A prova é baseada em integração por partes.
  
  Responda
  \begin{enumerate}
  \item Mostre que $\int_{1}^{\infty} x^{1/5}\sin x dx$ diverge, e como consequência prove que 
  $\int_{0}^{\infty} x^{2}\sin{x^{5/2}}dx$ também diverge. Compare com o teste de Dirichlet. 
  \item Prove que $\int_{0}^{\infty} \frac{\cos x}{\sqrt{x}}dx$ converge condicionalmente (Use o critério de Dirichlet).
  \item Mostre que a integral imprópria $\int_{0}^{\infty} \frac{\sin x}{x}dx$ converge condicionalmente (Critério de Dirichlet). Ainda mais, $\int_{0}^{\infty} \frac{\sin x}{x}dx=\pi/2$.
  \item Seja $f:[a,\infty)\rightarrow \mathbb{R}$ contínua tal que exista a integral imprópria $\int_{a}^{\infty} f(x)dx$. 
  \begin{itemize}
  \item Necessariamente  $\lim_{x\rightarrow \infty}f(x)=0$? Caso seja falso, mostre um contra-exemplo.
  \item Se $f(x)>0$, para $x>a$. É verdade que 
  $\lim_{x\rightarrow \infty} f(x)=0$?
  Caso seja falso, encontre um contra-exemplo.
  \item Mostre que se $f$ é não-crescente com $f(x)>0$, para $x>a$, então 
  $\lim_{x\rightarrow}f(x)dx=0$. 
  \end{itemize}
  \item Prove que $\int_{-\infty}^{\infty} (\frac{\sin x}{x})^{2}dx$
  \end{enumerate}
\end{document}



