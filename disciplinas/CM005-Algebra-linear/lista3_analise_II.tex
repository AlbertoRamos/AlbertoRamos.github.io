% lista 3 (analise II)
\documentclass{article}
\usepackage{amssymb,latexsym,amsthm,amsmath}
\usepackage{tikz}
\usepackage{verbatim}
\usepackage[brazil]{babel}
%\usepackage[latin1]{inputenc}
% parece que são conflictantes 
\usepackage[utf8]{inputenc}
\usepackage{amsfonts}
% \usepackage{showlabels}
\usepackage{latexsym}
%%%%%%%%%%%%%%%%%%%%%%%%%
\usepackage{color, colortbl}
\usepackage{tabularx,colortbl}
\usepackage{hyperref}
\usepackage{graphicx}
%%%%%%%%%%%%%%%%%%%%%%
\theoremstyle{plain}
\newtheorem{theorem}{Theorem}[section]
\newtheorem{corollary}[theorem]{Corollary}
\newtheorem*{main}{Main~Theorem}
\newtheorem{lemma}[theorem]{Lemma}
\newtheorem{proposition}[theorem]{Proposition}
\newtheorem{algorithm}{Algorithm}[section]
\theoremstyle{definition}
\newtheorem{definition}{Definition}[section]
\newtheorem{example}{Example}[section]
\newtheorem{counter}{Counter-Example}[section]

\theoremstyle{remark}
\newtheorem{remark}{Remark}

\headheight=21.06892pt
\addtolength{\textheight}{3cm}
\addtolength{\topmargin}{-2.5cm}
\setlength{\oddsidemargin}{-.4cm}
%\setlength{\evensidemargin}{-.5cm}
\setlength{\textwidth}{17cm}
%\addtolength{\textwidth}{3cm}
\newcommand{\R}{{\mathbb R}}

%%%%%%% definição de integral superior e inferior 
\def\upint{\mathchoice%
    {\mkern13mu\overline{\vphantom{\intop}\mkern7mu}\mkern-20mu}%
    {\mkern7mu\overline{\vphantom{\intop}\mkern7mu}\mkern-14mu}%
    {\mkern7mu\overline{\vphantom{\intop}\mkern7mu}\mkern-14mu}%
    {\mkern7mu\overline{\vphantom{\intop}\mkern7mu}\mkern-14mu}%
  \int}
\def\lowint{\mkern3mu\underline{\vphantom{\intop}\mkern7mu}\mkern-10mu\int}


\begin{document}

\title{Lista 3: Análise II}

\author{
A. Ramos \thanks{Department of Mathematics,
    Federal University of Paraná, PR, Brazil.
    Email: {\tt albertoramos@ufpr.br}.}
}

\date{\today}
 
\maketitle

\begin{abstract}
{\bf Lista em constante atualização}.
\begin{enumerate}
	\item Convergência pontual e uniforme
	\item Série de potências 
	%\item Integrais improprias.
\end{enumerate}
\end{abstract}

Notação: $\mathbb{N}:=\{1,2,\dots,\}$. A convergência uniforme é denotado como $f_{n} \xrightarrow{u} f$.
\section{Convergência uniforme e séries de potências.} 
\begin{enumerate}
  \item Enuncie e demonstre em detalhe, com as hipoteses suficientes,  que:
    \begin{enumerate}
    \item Uma sequencia de funções converge uniformemente se, e somente se 
    a sequencia é de Cauchy
    \item O limite uniforme de funções contínua é contínua.
    \item O limite uniforme de funções integráveis é integrável e
    a integral do limite é o limite das integráveis
    \item O limite uniforme de funções deriváveis é derivável 
    e a derivada do limite é o limite das derivadas.
    \item Mostre o Teorema de Dini.
    \end{enumerate}
  \item Veja se as seguintes séries de funções convergem uniformemente 
  no conjunto $X$.
   \begin{enumerate}
     \item $\sum_{n \in \mathbb{N}} \frac{x}{x^2+n^2}$, $X=\mathbb{R}$;
      $\sum_{n \in \mathbb{N}} \frac{sin(nx)}{x^2+n^{3/2}}$, $X=\mathbb{R}$;
           $\ \ \sum_{n \in \mathbb{N}} \frac{cos(nx^3)}{n^3}$, $X=\mathbb{R}$
     \item $\sum_{n \in \mathbb{N}} \frac{x}{(1+nx^2)n^\varepsilon}$, $X=[-a,a], a>0,\varepsilon>0$;
      $\sum_{n \in \mathbb{N}} \frac{1}{n^{x}}$, $X=[a,\infty), a>1$
     \item $\sum_{n \in \mathbb{N}}(-1)^{n}\frac{\cos(nx)}{(2n^5+3)^{1/7}}$, $X=[\varepsilon, 2\pi-\varepsilon], \varepsilon>0$ e 
     $\sum_{n \in \mathbb{N}}\frac{(-1)^n}{x+n}$, $X=[0, \infty)$ 
     \item $\sum_{n \in \mathbb{N}}x^{ 2^{n}+\sqrt{n}}$, $X=(0,1)$ e 
     $\sum_{n \in \mathbb{N}} \sqrt{n}\sin(x/n^2)$, $X=[-a,a]$    
   \end{enumerate}
  \item (Regras de cálculo para convergência uniforme)
   Sejam $f_{n}, g_n:X \rightarrow \mathbb{R}$ sequências de funções
   com $f_{n}\xrightarrow{u} f$ e  $g_{n}\xrightarrow{u} g$. Então:
     \begin{enumerate}
     \item Mostre que $f_n+g_n \xrightarrow{u} f+g$.
     \item Para qualquer sequência convergente de numeros reais 
     $a_{n} \in \mathbb{R}$ com $a_{n} \rightarrow a$ tem-se 
     $a_{n}f_{n} \xrightarrow{u} af$.
     \item Mostre que $f_{n}g_{n}\rightarrow fg$ pontualmente. De um exemplo onde a convergência não é uniforme. 
     \item Se $f_{n}$ e $g_n$ são uniformente limitadas (isto é, 
     existe $K>0$ tal que  
     $sup\{|f_{n}(x)|: x \in X\}\leq K$, $n \in \mathbb{N}$. Similarmente para a sequência $g_{n}$). 
     Então  $f_{n}g_{n}\xrightarrow{u} fg$.
     \item Se existe $K>0$ tal que $\inf \{|f_{n}(x)|: x \in X\}\geq K$, $n \in \mathbb{N}$. Então, $1/f_{n}\xrightarrow{u} 1/f$.
     \item Prove que $\phi \circ f_{n} \xrightarrow{u} \phi \circ f$, se 
     $\phi: \mathbb{R}\rightarrow \mathbb{R}$ é uniformente contínua.
     \end{enumerate}
  \item Considere $f_{n}(x):=(1+\frac{x}{n})^{n}$, 
  $x \in \mathbb{R}$, $n \in \mathbb{N}$. 
   \begin{enumerate}
   \item Prove que $f_{n}\rightarrow \exp(x)$ pontualmente e a convergência uniforme em cada intervalo $[a,b] \in mathbb{R}$.
   \item Mostre que 
   $(\frac{n^2+(n!)^{1/n}}{n^{2}})^{n} \rightarrow \exp(1/e)$ e
   $(\frac{n^{n+1}+(n+1)^{n}}{n^{n+1}})^{n} \rightarrow \exp(e)$, 
   onde $e=\exp(1)$.
   \end{enumerate}      
  \item Sejam $f_{n}, f:X \rightarrow \mathbb{R}$ funções. 
  Prove que $f_{n}\xrightarrow{u} f$ see 
  $\text{sup}\{|f_{n}(x)-f(x)|:x \in X\} \rightarrow 0$ quando 
  $n \rightarrow \infty$.
  \item Seja $f_{n}:[a,b] \rightarrow \mathbb{R}$ uma sequencia de funções 
  contínuas tal que $f_{n}\rightarrow f$ uniformente em $D\subset [a,b]$.
  Se $D$ é denso, mostre que se $f$ é uniformente contínua então 
  $f_{n}\xrightarrow{u} f$ em $[a,b]$.
  \item Seja $f_{n}:[a,b] \rightarrow \mathbb{R}$ uma sequencia de funções
  uniformente Lipschitziana, isto é, existe $K>0$ tal que  
  $|f_{n}(x)-f_{n}(y)|\leq K|x-y|$, $\forall x,y \in [a,b]$, 
  $n \in \mathbb{N}$. Mostre que se $f_{n}\rightarrow f$ então 
  $f_{n}\xrightarrow{u} f$.
  \item Seja $f_{n}:[a,b] \rightarrow \mathbb{R}$ uma sequencia de funções, tal que para toda sequencia $x_{n}\in [a,b]$ convergente
  tem-se que $f_{n}(x_{n})\rightarrow 0$. 
  Mostre que $f_{n} \xrightarrow{u} 0$. 
  \item Verfique as seguintes igualdades: 
    \begin{enumerate}
    \item $\exp(x)=\sum_{n=0}^{\infty} \frac{x^n}{n!}$, $x \in \mathbb{R}$;
    $\sin(x)=\sum_{n=0}^{\infty} (-1)^{n}\frac{x^{2n+1}}{(2n+1)!}$, $x \in \mathbb{R}$;
           $\ \ \cos(x)=\sum_{n=0}^{\infty} (-1)^{n}\frac{x^{2n}}{(2n)!}$, $x \in \mathbb{R}$;
     \item $\text{log}(\frac{1+x}{1-x})=2\sum_{n=0}^{\infty} \frac{x^{2n+1}}{2n+1}$, $x \in (-1,1)$;
    $\frac{1}{(1-x)^{4}}=\sum_{n=0}^{\infty} \frac{(n+1)(n+2)(n+3)}{6} x^n$, $x \in (-1,1)$     
    \end{enumerate}
 \item Expanda as funções em série de potências ao redor do ponto $x^*$.
 Determine o raio de convergência da série obtida
   \begin{enumerate}
    \item $f(x)=\frac{x-1}{x^2-4}$, $x^{*}\neq \pm2$; 
          $f(x)=\frac{x}{x-3}$, $x^{*}\neq 3$;
          $f(x)=tan(x)$, $x^{*}\neq 0$.
   \end{enumerate}       
 \item Determine o intervalo de convergência de cada uma das séries de potências
   \begin{enumerate}
   \item df
   \end{enumerate}     
 \item ({\it Critério de Raabe}) Seja 
 $$L:=\lim_{n \rightarrow \infty } n(1-\frac{a_{n+1}}{a_n}).$$  
   \begin{enumerate}
   \item Se $L>1$ ou $L=\infty$, então a série 
   $\sum_{n=1}^{\infty}a_{n}$ converge.
   \item Se $L \in [0,1)$, a série $\sum_{n=1}^{\infty}a_{n}$ diverge.
   \end{enumerate}
  \item Mostre que se $\sum_{n=1}^{\infty} |a_{n}|<\infty$, então 
  as séries $\sum_{n=1}^{\infty} a_{n}\cos(nx)$ e  
   $\sum_{n=1}^{\infty} a_{n}\sin(nx)$  convergem uniformemente em 
   $\mathbb{R}$.
   \item Prove que para todo $x \in (-1,1]$, tem-se que 
   $$ \text{log}(1+x)=\sum_{n=1}^{\infty} \frac{(-1)^{n+1}}{n}x^n.$$
   Em particular, $\text{log}(2)=\sum_{n=1}^{\infty} \frac{(-1)^{n+1}}{n}$.
  \item Seja $g:\mathbb{R}\rightarrow\mathbb{R}$ uma função periódica de periodo 2, tal que    $g(x)=x$, $x \in [0,1]$ e $g(x)=2-x$, $x \in [1,2]$. 
  Defina a série formal 
  $$f(x):= \sum_{n=0}^{\infty} (\frac{3}{4})^{n}g(4^{n}x).$$
  Mostre que $f$ está bem definido que é uma função contínua em $\mathbb{R}$
  mas não é derivável em nenhum ponto. 
  \item Seja $f:\mathbb{R}\rightarrow \mathbb{R}$ uma função definida como 
  $f(x)=\exp(-1/x)$, $x>0$ e 
  $f(x)=0$, $x\geq0$. 
  Mostre que $f$ é de clase $C^{\infty}$
  em $\mathbb{R}$ mas não é analítica em $x=0$.
  \item Determine o intervalo de convergência de cada uma das séries de potências:
    \begin{enumerate}
    \item $\sum_{n=1}^{\infty} \frac{n}{4^{n}}x^n \ \ $; 
          $\sum_{n=1}^{\infty} \frac{(3n)!}{(2n)!}x^n  \ \ $; 
          $\sum_{n=1}^{\infty} \frac{n!}{n^{n}}x^n$.
    \item $\sum_{n=1}^{\infty} \frac{(x+1)^n}{a^{n}+b^n}, b>a>0 \ \ $; 
          $\sum_{n=1}^{\infty} (\frac{3n+2}{5n+7})^{n}x^n  \ \ $; 
          $\sum_{n=1}^{\infty} (-1)^{n}\sin(\frac{1}{n})x^n$.   
    \item $\sum_{n=1}^{\infty} \frac{x^n}{n^{1/n}}, \ \ $; 
          $\sum_{n=1}^{\infty}
          \frac{2^n\text{log}(n)}{3^{n}n n^{1/n}}x^n \ \ $; 
          $\sum_{n=1}^{\infty} \frac{(-2)^{n}n!}{e^{n^2}}x^n$.          
    \end{enumerate}
  \item Assuma que $\sum_{n=0}^{\infty} a_n x^n$ converge para 
  $x=-4$ e diverge para $x=6$. 
  Quais das seguintes séries divergem ou convergem?
     $$
       (a) \sum_{n=0}^{\infty} a_n;  \ \ 
       (b) \sum_{n=0}^{\infty} (-3)^n a_n \ \ \text{ e } \ \ 
       (c) \sum_{n=0}^{\infty} (-1)^{n} a_n 9^{n}.  
     $$
  \item Usando a derivação e integração termo a termo, calcule as seguintes somas de séries de potências.
     $$
       (a) \sum_{n=1}^{\infty} \frac{x^n}{n};  \ \ 
       (b) \sum_{n=0}^{\infty} (-1)^{n-1}\frac{x^{2n-1}}{2n-1}; \ \  
       (c) \sum_{n=0}^{\infty} n x^{2n-1}; \ \ 
       (d) \sum_{n=1}^{\infty} n^3 x^n; \ \  
       (e) \sum_{n=0}^{\infty} \frac{x^{4n}}{4n}. \ \  
     $$         
  \item Desenvolva as sequiente 
  funções em séries de potências ao redor do origem (série de Maclaurin). Indique os intervalos de convergência.
    $$
       (a) f(x)=x^2e^x;  \ \ 
       (b) f(x)=\sin(x^2); \ \  
       (c) f(x)=\sin^{2}(x); \ \ 
       (d) f(x)=\frac{\exp(x^2)-1}{x}; \ \  
       (e) f(x)=\int_{0}^{x} \frac{\sin t}{t}dt. \ \  
     $$
  \item Considere a função
  $$ f(x):= \sum_{n=1}^{\infty} \exp(-2^{n/2})\cos(2^{n}x).$$
  Mostre que $f$ é de clase $C^{\infty}(\mathbb{R})$ mas não é analítica em nenhum ponto.  
  \item Faça os primeiros 42 problemas do capítulo X do livro texto.
  \end{enumerate} 
\end{document}



