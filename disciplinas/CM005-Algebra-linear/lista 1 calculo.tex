% lista 1(calculo 1)
\documentclass[latin,20pt]{article}
%\usepackage{amssymb,latexsym,amsthm,amsmath}
\usepackage[paper=a4paper,hmargin={1cm,1cm},vmargin={1.5cm,1.5cm}]{geometry}
\usepackage{amsmath,amsfonts,amssymb}
\usepackage[utf8]{inputenc}

%\usepackage{stmaryrd} %%para graficar maximo inteiro 
\begin{document}

\title{Lista 1: Cálculo I }
 
\author{
A. Ramos \thanks{Department of Mathematics,
    Federal University of Paraná, PR, Brazil.
    Email: {\tt albertoramos@ufpr.br}.}
}

\date{\today}
 
\maketitle

\begin{abstract}
{\bf Lista em constante atualização}.
 \begin{enumerate}
 \item Funções 
 \end{enumerate}
\end{abstract}

%%%%%%%%%%%%%%%%%%%%%%%%%%%%%%%%%%%%%%%%%%%%%%  
%\section*{Elipse} 
%Seja $\mathcal{O}$ um aberto em $\mathbb{R}^{n}$. 
%Denote por 
%$C^{1,1}_{L}(\mathcal{O})$ o conjunto das funções deriváveis 
%em $\mathcal{O}$ cuja derivada é Lipschitziana com constante de 
%     Lipschitz $L$ em $\mathcal{O}$, isto é, 
%     $\|\nabla f(x)-\nabla f(y)\|\leq L\|x-y\|$, 
%     para todo $x,y \in \mathcal{O}$.
 
  \section{Exercícios}   
 
 Faça do livro texto, os exercícios correspondentes aos temas desenvolvidos em aula. 
  
  \section{Exercícios adicionais}  
  \subsection{Domínio, imagem e gráfico de funções}   
  
    \begin{enumerate}
    \item Seja $f:[-2,4) \rightarrow \mathbb{R}$ definida como 
    $f(x)=\frac{|x+1|-3}{1+|x-3|}$. Encontre a imagem de $f$.{\it Rpta}
    $im(f)=[-3/5,1]$.     
    \item Considere $f(x)=\frac{2\sqrt{x}}{|1-x|}$. 
    Ache o domínio, imagem e gráfico de $f$.
    \item Seja $f(x)=(x-\lbrack\!\lbrack x \rbrack\!\rbrack)^2$. Encontre o dominio e a imagem de $f$. Faça o gráfico de $f$. 
    {\it Rpta:} $dom(f)=\mathbb{R}$, $im(f)=[0,1)$.
    %\item  $\lbrack\!\lbrack  ...   \rbrack\!\rbrack$ %\llbracket f \rrbracket$
    \item Encontre o dominio, a imagem e o gráfico de 
    $$f(x)=\frac{x^3+x^2+x+1}{|x+1|}.$$
    {\it Rpta: } $dom(f)=\mathbb{R}\setminus\{-1\}$, 
    $im(f)=(-\infty, -2)\cup [1,\infty)$
    \item Seja $f(x)$ uma função lineal tal que $f(-1)=2$ e $f(2)=-3$. 
    {\it Rpta: } $f(x)=(-5x+1)/3$.
    \end{enumerate}
 \subsection{Operações básicas para funções}
     \begin{enumerate}
    \item Encontre o produto de $f$ e $g$ se 
    $$
    f(x)= \left\{  
            \begin{array}{lll}
            &2x+1  &\text{, se } x\geq 1\\
            &x^2-2 &\text{, se } x<0
            \end{array}
            \right.
     \text{ e }
    g(x)= \left\{  
            \begin{array}{lll}
            &3x+1  &\text{, se } x\leq 8\\
            &x^3   &\text{, se } x>10
            \end{array}
            \right. 
    $$
    {\it Rpta:} 
     $$
    (f.g)(x)= \left\{  
            \begin{array}{lll}
            &3x^3+x^2-6x-2  &\text{, se } x<0\\
            &6x^2+5x+1      &\text{, se } 1\leq x \leq 8 \\
            &4x^4+2x^3      &\text{, se } 10< x \\
            \end{array}
            \right. 
    $$
    \item Encontre a divisão de $f$ e $g$ se 
    $$
    f(x)= \left\{  
            \begin{array}{lll}
            &\sqrt{1-x}  &\text{, se } x\geq 1\\
            &\sqrt{x} &\text{, se } x\geq 4
            \end{array}
            \right.
     \text{ e }
    g(x)= \left\{  
            \begin{array}{lll}
            &x^2-1  &\text{, se } x<0\\
            &x   &\text{, se } 0\leq x \leq 2 \\
            &x+5 &\text{, se } x>2
            \end{array}
            \right. 
    $$
    {\it Rpta:} 
     $$
    (\frac{f}{g})(x)= \left\{  
            \begin{array}{lll}
    &\frac{\sqrt{1-x}}{x^2-1} &\text{, se } x \in (-\infty, -1)\cup (-1,0)\\
    &\frac{\sqrt{1-x}}{x}     &\text{, se } x \in (0,1] \\
            & \frac{\sqrt{x}}{x+5}     &\text{, se } x \in [4,\infty] \\
            \end{array}
            \right. 
    $$
    \item Encontre a composição de $f\circ g$
     se 
    $$
    f(x)= \left\{  
            \begin{array}{lll}
            &x+2  &\text{, se } x\geq 1\\
            &x-1 &\text{, se } x> 1
            \end{array}
            \right.
     \text{ e }
    g(x)= \left\{  
            \begin{array}{lll}
            &x^2  &\text{, se } x<0\\
            &1-x   &\text{, se } 0\leq x \geq 0 
            \end{array}
            \right. 
    $$
    {\it Rpta:} 
     $$
    (f\circ g)(x)= \left\{  
            \begin{array}{lll}
    &x^2-1 &\text{, se } x \in (-\infty, -1) \\
    &x^2+2 &\text{, se } x \in [-1,0) \\
    & 3-x  &\text{, se } x \in [0,\infty] \\
            \end{array}
            \right. 
    $$
    \item 
       \begin{enumerate}
       \item Se $f(x-1)=x-2$ e $(g\circ f)(x+2)=2x^2-x$. Encontre $g(x)$. 
    {\it Rpta: } $g(x)=2x^2-5x+3$.
       \item Se $F(x)=\cos(2x)$ e $f(x)=\sin(x)$.  Encontre $g(x)$ tal que 
       $F(x)=(g \circ f)(x)$. {\it Rpta: } $g(x)=1-2x^2$.  
      \end{enumerate}
    \item Considere as funções $f$ e $g$ definidas como  
    $$f(x)=\frac{\sqrt{2x-1}}{|1-x|} \ \ \text{ e } \ \ g(x)=2\cos(\frac{2\pi}{3}(\frac{|x|}{x^2+1})).$$ Analise a existência da composição de $f \circ g$.  
    \end{enumerate}
 \subsection{Funções injetoras, sobrejetoras e inversas} 
 
 {\it Lembre: } Uma função $f$ é injetora se para todo $a, b \in dom(f)$ tal que $f(a)=f(b)$ então temos que $a=b$. \newline 
 {\it Observação: } Sejam $f$ e $g$ duas funções injetoras tal que
 $f \circ g$ existe. Então  $f \circ g$ é injetora e a inversa satisfaz 
  $$(f \circ g)^{-1}= g^{-1} \circ f^{-1}.$$ 
  {\it Mostre essa observação}.        
    \begin{enumerate}
    \item Mostre que toda função crescente (ou decrescente) é injetora. 
    \item Considere $f: X\rightarrow (-4,1]$ com $f(x)=\frac{10+3x}{10-2x}$. 
      \begin{enumerate}
      \item Determine $X$ para que f seja sobrejetora. {\it Rpta: } 
      $X=(-\infty,0]\cup (10,\infty)$.
      \item Mostre que f é injetora. 
      \end{enumerate} 
    \item Considere a função    
    $$
    f(x)= \left\{  
            \begin{array}{lll}
    &2x-1 &\text{, se } x \in (-\infty, -1) \\
    &4x^2 &\text{, se } x \in [-1,0] \\
    &x+4  &\text{, se } x \in (0,\infty] \\
            \end{array}
            \right. 
    $$
    Calcule a inversa de $f$ e faça o gráfico de $f^{-1}$. {\it Rpta: }
     $$
    f^{-1}(x)= \left\{  
            \begin{array}{lll}
    &\frac{x+1}{2} &\text{, se } x \in (-\infty, -3) \\
    &-\frac{\sqrt{2x}}{2} &\text{, se } x \in [0,4] \\
    & x-4  &\text{, se } x \in (4,\infty] \\
            \end{array}
            \right. 
    $$
    \item Considere as funções    
    $$
    g(x)= \frac{x}{x+2}, \text{se } x<-2    
    \text{ e }
    f(x)= \left\{  
            \begin{array}{lll}
    &2x^2-12x+2 &\text{, se } x \in (-2, 3] \\
    &\frac{\sqrt{x+2}}{\sqrt{x-3}} &\text{, se } x \in (3,\infty) \\
            \end{array}
            \right.
    $$
    Calcule $g^{-1}\circ f$. {\it Rpta: }
     $$
    (g^{-1}\circ f)(x)= \left\{  
            \begin{array}{lll}
    &-\frac{4(x^2-6x+1)}{2x^2-12x+1} &\text{, se } x \in (-2, 3-\sqrt{17}) \\
    &\frac{2\sqrt{x+2}}{\sqrt{x-3-\sqrt{x+2}}} &\text{, se } x >3
            \end{array}
            \right. 
    $$
    \item Considere 
    $$f(x)=\frac{4x+|x-5|+\sqrt{x-5}+5-x\lbrack\!\lbrack x \rbrack\!\rbrack}{\sqrt{6-x}}.$$
    Ache a inversa, se ela existe. {\it Rpta:}  Se existe e 
    $f^{-1}(x)=\frac{6x^2+5}{x^2+1}$.
    \end{enumerate}         

\end{document}

  
