% lista 4(geometria analitica 2017 I)
\documentclass{article}
%\usepackage{amssymb,latexsym,amsthm,amsmath}
\usepackage{amsmath,amsfonts,amssymb}
\usepackage{tikz}
\usepackage{verbatim}
\usepackage[brazil]{babel}
%\usepackage[latin1]{inputenc}
% parece que são conflictantes 
\usepackage[utf8]{inputenc}
%\usepackage{amsfonts}
% \usepackage{showlabels}
\usepackage{latexsym}
%%%%%%%%%%%%%%%%%%%%%%%%%
\usepackage{tikz}
\usetikzlibrary{patterns,arrows}
\usetikzlibrary{arrows.meta,calc,decorations.markings,math,arrows.meta}
%%%%%%%%%%%%%%%%%%%%%%%%%

%%%%%%%%%%%%%%%%%%%%%%%%%
%\usepackage{xcolor} %, colortbl}
%\usepackage{tabularx,colortbl}
%\usepackage{hyperref}
%\usepackage{graphicx}
%%%%%%%%%%%%%%%%%%%%%%
%\usepackage{tcolorbox}
%\usepackage{verbatimbox}
%%%%%%%%%%%%%%%%%%%%%%
%\usepackage{framed,color}
%\definecolor{shadecolor}{rgb}{1,0.8,0.3}
%%%%%%%%%%%%%%%%%%%%%
%\usepackage{fancybox}

%\theoremstyle{plain}
\newtheorem{theorem}{Theorem}[section]
\newtheorem{corollary}[theorem]{Corollary}
%\newtheorem*{main}{Main~Theorem}
\newtheorem{lemma}[theorem]{Lemma}
\newtheorem{proposition}[theorem]{Proposition}
\newtheorem{algorithm}{Algorithm}[section]
%\theoremstyle{definition}
\newtheorem{definition}{Definition}[section]
\newtheorem{example}{Example}[section]
\newtheorem{counter}{Counter-Example}[section]

%\theoremstyle{remark}
\newtheorem{remark}{Remark}

\headheight=21.06892pt
\addtolength{\textheight}{3cm}
\addtolength{\topmargin}{-2.5cm}
\setlength{\oddsidemargin}{-.4cm}
%\setlength{\evensidemargin}{-.5cm}
\setlength{\textwidth}{17cm}
%\addtolength{\textwidth}{3cm}
\newcommand{\R}{{\mathbb R}}

%%%%%%% definição de integral superior e inferior 
\def\upint{\mathchoice%
    {\mkern13mu\overline{\vphantom{\intop}\mkern7mu}\mkern-20mu}%
    {\mkern7mu\overline{\vphantom{\intop}\mkern7mu}\mkern-14mu}%
    {\mkern7mu\overline{\vphantom{\intop}\mkern7mu}\mkern-14mu}%
    {\mkern7mu\overline{\vphantom{\intop}\mkern7mu}\mkern-14mu}%
  \int}
\def\lowint{\mkern3mu\underline{\vphantom{\intop}\mkern7mu}\mkern-10mu\int}

\begin{document}

\title{Lista 4: Geometria Analítica}

\author{
A. Ramos \thanks{Department of Mathematics,
    Federal University of Paraná, PR, Brazil.
    Email: {\tt albertoramos@ufpr.br}.}
}

\date{\today}
 
\maketitle

\begin{abstract}
{\bf Lista em constante atualização}.
 \begin{enumerate}
 \item Transformação de coordenadas;
 \item Equação da parabóla; %Equação da reta e do plano; %Vetores (no plano e no espaço);
 %\item Equação da elipse; 
 %\item Equação da hiperbóla.
 \end{enumerate}
\end{abstract}


\section*{Transformação de coordenadas}
 Existe dois tipos de transformações importantes: {\it traslação} e {\it rotação}, quais podem ser combinados para descrever movimentos mais complexos. É importante se sentir confortável com essas transformações.
 
% \shadowbox{gdfgd}
  
 {\it Translação de eixos.} Em $\mathbb{R}^{2}$ (em $\mathbb{R}^{3}$ é similar), considere um sistema de coordenadas cuja origem $O=(0,0)$ é trasladada a $O'=(h,k)$.
 Seja $P \in \mathbb{R}^{2}$ um ponto com coordenadas 
 $(x,y)$ no sistema de coordenadas original e 
 com coordenadas $(x',y')$ no 
 novo sistema de coordenadas (com origem $O'$). Então, temos que :
   $$  x=x'+h \ \ \text{ e } \ \ y=y'+k.$$
 {\it Rotação de eixos.} Em $\mathbb{R}^{2}$, considere dois sistemas de coordenadas, uma obtida apartir da outras atraves de uma rotação (com ângulo 
 $\theta$ e sentido antihorário). Se
 $P \in \mathbb{R}^{2}$ um ponto com coordenadas 
 $(x,y)$ no sistema de coordenadas original e 
 com coordenadas $(x',y')$ no 
 novo sistema de coordenadas. Então, 
 $$  x=x'\cos(\theta)-y'\sin(\theta) \ \ \text{ e } \ \ 
     y=x'\sin(\theta)+y'\cos(\theta).$$
 Usando matrizes podemos escrever a expressão como (talvez mais fácil para decorar):
 $$  
     \begin{pmatrix}
      x \\
      y
     \end{pmatrix}= \begin{pmatrix}
                    \cos(\theta) & -\sin(\theta) \\
                    \sin(\theta) &  \cos(\theta) \\
                    \end{pmatrix}  
                     \begin{pmatrix}
                     x' \\
                     y'
                     \end{pmatrix}.
 $$                       
 Da expressão anterior temos que
  $$  
     \begin{pmatrix}
      x'\\
      y'
     \end{pmatrix}= \begin{pmatrix}
                    \cos(\theta)  &  \sin(\theta) \\
                    -\sin(\theta) &  \cos(\theta) \\
                    \end{pmatrix}  
                     \begin{pmatrix}
                     x \\
                     y
                     \end{pmatrix}.
 $$  
 %  é similar.  
 %Em $\mathbb{R}^{3}$, a rotação de dá atraves de dois ângulos 
 %de Euler. 
 Responda:
   \begin{enumerate}
    \item Usando uma translação transforme a equação 
    $2x^2-3xy+5x+3y-8=0$ em outra equação sem termos lineares.
    {\it Rpta:} Nova origem $O'=(1,3)$ e equação $2x'^2-3x'y'-1=0$
    \item Mediante uma translação transforme a equação 
    $8x^3-x^2+24x-y+1=0$, em outra que não tem termos 
    de segunda ordem nem termo constante.
    {\it Rpta} Nova origem $O'=(-4,8)$ e equação $(x')^{2}y'-1=0$.
    \item Encontre o ângulo de rotação para que a curva
    $8x^2+3\sqrt{3}xy+11y^2=24$ não tenha o termo $xy$.
    {\it Rpta} $\theta=60^{\circ}$.
    \item Mostre que a curva $11x^2+24xy+4y^2=20$ depois de uma rotação 
    $\theta=arctan(3/4)$ se escreve como $4x'^2-y'^2=4$.
    \item Usando primeiramente uma translação com nova origem $O'=(1,1)$ e
    logo uma rotação de $45^{\circ}$, uma equação se transforma 
    em $(x'')^{2}-2(y'')^{2}=2$. Qual é a equação original?
    {\it Rpta:} $x^2-6xy+y^2+4x+4y=0$.
   \end{enumerate}

 \section*{Parábola} 
  
% \begin{tcolorbox}
   Dada uma reta $\mathcal{D}$ e um ponto $F \notin \mathcal{D}$.
   A parábola é definida como 
   $$\mathcal{P}:=\{P \in \mathbb{R}^{2}: dist(P,F)=dist(P,\mathcal{D})\}.$$
   O ponto $F$ é chamado de foco e a reta $\mathcal{D}$ é chamada de reta diretriz.
     \begin{enumerate}
      \item {\bf eixo de simetria: } reta perpendicular à diretriz que passa por F;
      \item {\bf vértice: } interseção no eixo de simetria com a parábola;
      \item {\bf corda: } qualquer segmento de une dois pontos diferentes da parábola;
      \item {\bf corda focal: } corda que passa por F;
      \item {\bf lado reto (ou corda principal): } corda focal paralela à diretriz;
      \item {\bf raio vetor: } segmento de reta que une o foco com algum ponto da parábola. 
     \end{enumerate}
  
 %\end{tcolorbox}
 
 Usando um sistema de coordenadas a parábola $\mathcal{P}$ pode ser escrita com uma das 
 seguintes formas. 
 
 {\bf Forma canônica }(também chamada de {\it forma reduzida}) $y^{2}=4px$ ou $x^2=4py$.
 onde o vertice $V=(0,0)$ e o eixo de simetria é paralelo a algum dos eixos canônicos.
 $$ \text{ Observe que } |p|=dist(V,F) \text{ e } 
                         |p|=dist(V,\mathcal{D}). $$
 
 Quando o vertice $V=(h,k)$ e o eixo de simetria é paralelo a algum dos eixos canônicos temos que $(y-k)^{2}=4p(x-h)$ ou $(x-h)^2=4p(y-k)$.
 %onde o vertice $V=(h,k)$ e o eixo de simetria é paralelo a algum dos eixos  %canônicos.
 
 {\bf Forma geral} $y^{2}+Dy+Ex+F=0$ ou $x^{2}+Dx+Ey+F=0$
 onde o eixo de simetria é paralelo a algum dos eixos canônicos.
 
 %{\bf Forma geral mesmo}. \newline
 
 {\it Retas tangentes: } 
 Em qualquer ponto sobre a parábola podemos 
 calcular retas tangentes e retas normais.
 Para as retas tangentes temos as seguintes formulas qual depende da 
 equação usada da parábola.
 
 {\bf Quando $y^{2}=4px$}. A reta tangente a $\mathcal{P}$ no ponto $P=(x_0,y_0) \in \mathcal{P}$ é dada por 
 $ r: yy_0=2p(x+x_0)$. 
 
 {\bf Quando $x^{2}=4py$}. A reta tangente a $\mathcal{P}$ no ponto $P=(x_0,y_0) \in \mathcal{P}$ é dada por 
 $ r: xx_0=2p(y+y_0)$. \newline

 
Proceda a responder as seguintes questões 
  
  \begin{enumerate}
     \item Escreva as equações das seguintes parábolas.
       \begin{enumerate}
        \item Se $F=(0,2)$ e diretriz $\mathcal{D}: y+2=0$.
        \item Se $F=(0,0)$ e diretriz $\mathcal{D}: y+x=2$.
        \item Se o vértice é $V=(-3,2)$ e foco $F=(-1,2)$.
       \end{enumerate}
     \item Ache a equação da parabóla que tem foco
     $(-5/3,0)$ e cuja reta diretriz é $3x-5=0$.
     {\it Rpta} $3y^2+20x=0$.
     \item Encontre a longitude da corda focal da parábola
     $\mathcal{P}: x^2+8y=0$ que é paralela à reta 
     $r: 3x+4y-7=0$. {\it Rpta: } $25/2$.
     \item Encontre a equação da parábola com foco $F=(2,1)$, 
     com vértice sobre a reta $r: 3x+7y+1=0$ e 
     cuja diretriz é paralela ao eixo x. 
     {\it Rpta: } $\mathcal{P}: (x-2)^2=8(y+1)$.
     \item Se uma parábola tem um vértice sobre a
      reta $r_1:3x-2y=19$, o foco sobre a 
      reta $r_2: x+4y=0$ e 
      diretriz $\mathcal{D}: x=2$.
      {\it Rpta: } $(y+2)^{2}=12(x-5)$.
     \item Encontre a equação de uma parábola cuja lado reto 
     tem como extremo os pontos 
     $A=(7,3)$ e $B=(1,3)$. 
     {\it Rpta: } $\mathcal{P}_1:(x-4)^2=6(y-3/2)$
     e  $\mathcal{P}_2:(x-4)^2=-6(y-9/2)$.
     \item Encontre o valor de $\alpha \neq 0$, para que as 
     coordenadas do foco da parábola 
     $\mathcal{P}: x^2+4x-4\alpha y=8$ somem zero.
     {\it Rpta: } $\alpha=3$ ou $\alpha=-1$.
     \item Encontre a equação da circunferência 
     que passa por o vértice 
     e os extremos do lado reto da parábola 
     $\mathcal{P}: y^2+2y-4x+9=0$. 
     {\it Rpta: } $\mathcal{C}: (x-1/2)^2+(y+1)^2=9/4$.
     \item Encontre a equação da reta tangente e normal 
     da parábola $\mathcal{P}: y^2+2y-4x-7=0$ no ponto de contato 
     $T=(7,5)$ ($T \in \mathcal{P}$). {\it Rpta: }
     reta tangente: $x-3y+8=0$ e reta normal: $3x+y-26=0$.
     \item Encontre as retas tangentes à parabóla 
     $\mathcal{P}: y^2+3x-6y+9=0$ que passa por 
     $P=(1,4)$. {\it Rpta: }
     $3x-2y+5=0$ e $x+2y-9=0$.
     \item Considere a reta $r: x-2y-8=0$. 
     Ache o ponto da parabóla $x^2=4y$ tal que a 
     distância à reta $r$ seja a mínima possível e calcule 
     tal distância.
     {\it Dica: } O ponto deve ser ponto de tangência. 
     {\it Rpta: } $T=(1,1/4)$, distancia$=3\sqrt{5}/2$.
     \item * Considere a parabóla 
     $\mathcal{P}: x^2-2x+8y-23=0$, o ponto de tangência 
     $T=(5,1)$ e um triângulo formado 
     pelo eixo y, a reta tangente e a reta normal em $T$.
     Considere um rectângulo com uns dos lados paralelos
     ao eixo y. Escreva, a àrea do triângulo em função 
     do comprimento $x$ da base e calcule a àrea do rectangulo 
     com a maior àrea possível. 
     {\it Rpta: } A àrea em função de $x$ é $Area_{\Delta}(x)=x(10-2x)$, 
     $x \in (0,5)$.
     O máximo acontece quando $x=5/2$ e $Area_{\Delta}(5/2)=12,5 u^2$.
     \item ** Se o vértice de uma parabóla $\mathcal{P}$ 
     é $V=(-3,1)$, sua reta diretriz é paralela a 
     $r: 3x+4y-6=0$ e uns dos extremos do lado reto é $(6,3)$. 
     Encontre a equação da parabóla. 
     {\it Rpta: } $p=7$, $\mathcal{P}: 16x^2-24xy+9y^2-300x-650y-475=0$.
     \item A entrada duma igreja tem a forma 
     duma parabóla de 9 m. de altura e 12 m. de base. 
     Toda a parte superior é uma janela de vidro cuja 
     base é paralela à base da entrada
     e tem um comprimento de 8 m. 
     Qual a altura máxima da janela?
     {\it Rpta: } altura=4m.
   \end{enumerate}

\end{document}
  
%%%%%%%%%%%%%%%%%%%%%%%%%%%%%%%%%%%%%%%%%%%%%%  
\section*{Elipse} 
 
 \begin{tcolorbox}
   Dado dois pontos $F_1$ e $F_2$ no plano, e dois números positivos $a$ e $c$ ($a>c$)
   com $dist(F_1,F_2)=2c$.  
   A elipse é definida como o seguinte conjunto 
   $$\mathcal{E}:=\{P \in \mathbb{R}^{2}: dist(P,F_1)+dist(P,F_2)=2a\}.$$
   Os pontos $F_1$ e $F_2$ são chamados de focos.
   
   Se $b:=\sqrt{a^{2}-c^{2}}$, o número $e:=\frac{c}{a}$ é chamado 
   de {\it excentricidade} da elipse.
    \begin{enumerate}
      \item {\bf eixo focal: } reta que contem os focos $F_1$ e $F_2$;
      \item {\bf Vértices: $V_1$ e $V_2$ } Interseção do eixo focal com a elipse;
      \item {\bf centro : } Ponto meio do segmento $F_1F_2$;
      \item {\bf eixo normal: } reta perperndicular ao eixo focal que passa pelo centro; 
      \item {\bf corda: } qualquer segmento de une dois pontos diferentes da elipse;
      \item {\bf corda focal: } corda que passa por algum foco;
      \item {\bf lado reto : } corda focal paralela à reta normal;
      \item {\bf raio vetor: } segmento de reta que une algum 
      foco com algum ponto da parábola;
      \item {\bf diámetro: } corda que passsa pelo centro.
      \item {\bf eixo maior: } segmento $V_1V_2$;
      \item {\bf eixo menor: } segmeto definido pela interseção da elipse 
      com a reta normal;
      \item {\bf retas diretrizes: }  retas paralelas à reta normal 
      cuja distância ao centro $C$ é 
     \end{enumerate}
     Observe que 
     $$ dist(V_1,V_2)=2a \ \ (\text{ eixo maior da elipse }), \ \ 
        dist(F_1,F_2)=2c \ \ (\text{ distância focal }).$$
 \end{tcolorbox}
 Usando um sistema de coordenadas a elipse $\mathcal{E}$ pode ser escrita com uma das 
 seguintes formas. 
 
 {\bf Forma canônica }(também chamada de forma reduzida) 
 $\frac{x^2}{a^2}+\frac{y^2}{b^2}=1$ ou $\frac{x^2}{b^2}+\frac{y^2}{a^2}=1$, 
 onde o centro $C=(0,0)$ e o eixo focal é paralelo a algum dos eixos canônicos.
 {\it Desenhe ambas elipse explicitando o segmento que tem comprimento $a$ e/ou $b$}.
 
 {\bf Forma comum?} 
  $\frac{(x-h)^2}{a^2}+\frac{(y-k)^2}{b^2}=1$ ou $\frac{(x-h)^2}{b^2}+\frac{(y-k)^2}{a^2}=1$, 
  %$(y-k)^{2}=4p(x-h)$ ou $(x-h)^2=4p(y-k)$.
 onde o centro $C=(h,k)$ e o eixo focal é paralelo a algum dos eixos canônicos.
 
 {\bf Forma geral ?} $x^2+y^{2}+Dy+Ex+F=0$.
 onde o eixo focal é paralelo a algum dos eixos canônicos.
 
 {\bf Forma geral mesmo} $Ax^2+Bxy+Cy^{2}+Dy+Ex+F=0$ se $B^{2}-4AC<0$. \newline
 
 Em qualquer ponto sobre a elipse podemos calcular retas tangentes e retas normais.
 
 {\bf Quando $\frac{x^2}{a^2}+\frac{y^2}{b^2}=1$}. 
 A reta tangente à $\mathcal{E}$ no ponto $P=(x_0,y_0) \in \mathcal{E}$ é dada por 
 $ r: (\frac{x_{0}}{a^2})x+(\frac{y_0}{b^{2}})y=1$. 
 
 {\bf Quando $\frac{x^2}{b^2}+\frac{y^2}{a^2}=1$}. 
 A reta tangente à $\mathcal{E}$ no ponto $P=(x_0,y_0) \in \mathcal{E}$ é dada por 
 $ r: (\frac{x_{0}}{b^2})x+(\frac{y_0}{a^{2}})y=1$. 
  
  Com essas informações responda:
    \begin{enumerate}
    \item Calcule os focos, vértices, a medida do eixo maior e a do eixo menor, esboce as elipses
       \begin{enumerate}
       \item $x^2/9+y^2/25=1  \ \  \text{ e } \ \ 4x^2+10y^2=40$
       \item $4x^2+169y^2=676  \ \ \text{ e } \ \ 16x^2-4+4y^2=0$ 
       \end{enumerate}
    \item Escreve a equação reduzida da elipse nos seguintes casos:
     \begin{enumerate}
     \item $Centro=(0,0)$, eixo focal paralelo ao eixo $x$, o eixo menor mede 6 e a distância focal é 8. 
     \item Os focos são $(0,6)$ e $(0,-6)$ e o eixo maior mede 34
     \item $Centro=(0,0)$, um foco é $(0,-\sqrt{40})$ e o ponto 
     $(\sqrt{5}, 14/3)$ pertence à elipse.
     \item Os focos são $F_1=(1,1)$ e $F_2=(-1,-1)$
     e satisfaz $dist(P,F_1)+dist(P,F_2)=4$
     \end{enumerate}        
    \item    
    \end{enumerate}
\section*{Equação do plano} 

{\bf Equação vetorial.} 
 Um plano $\pi$ em $\mathbb{R}^{n}$, pode ser escrita como 
 $\pi: P=P_{0}+tV+sW$, $t, s \in \mathbb{R}$ 
 onde $V, W$ são linearmente independentes. 
 Note que existe infinitos vetores $V$ e $W$ que geram o mesmo plano. 
 Se conhecemos três pontos sobre a reta, por exemplo $P_{0}$, $P_1$
 e $P_2$, os vetores 
 $\overrightarrow{P_0P_1}, \overrightarrow{P_0P_2} \in \mathbb{R}^{n}$ 
 servem como geradores do plano $\pi$, se
 $\overrightarrow{P_0P_1}, \overrightarrow{P_0P_2}$ são linearmente independentes ({\it por que?}). 
 Assim, qualquer vetor paralelo a $\pi$ pode ser escrito como combinação linear de $V$ e $W$.
 
 {\bf Equação normal do plano em $\mathbb{R}^{3}$}
 Quando o plano está em $\mathbb{R}^{3}$, podemos escrever o plano da 
 forma 
$$ax+by+cz+d=0, \text{ para certos } a,b,c, d \in \mathbb{R}, $$ 
com $a^{2}+b^{2}+c^2\neq0$. Dita forma se chama de 
 {\it equação geral do plano ou equação normal}. Perceba que o vetor 
 $(a,b,c) \in \mathbb{R}^{2}$ é um vetor normal ao plano. 
 
 Se $V$ e $W$ são vetores paralelos ao plano, $V \times W$ serve como 
 vetor normal ao plano.
 
 Veja que se $(x_0, y_{0}, z_0)$ está sobre o plano $\pi$ 
 $\pi: ax+by+cz+d=0$, temos que 
 $$ (a,b,c) \perp ((x,y,z)-(x_0,y_0,z_0)), \text{ para todo } (x,y,z) \in \pi. $$ 

Em $\mathbb{R}^{3}$, considere dois plano $\pi_1$ e $\pi_2$ com $\pi_1 \cap \pi_2 \neq \emptyset$.
  Certamente, $\pi_1 \cap \pi_2$ é uma reta ({\it por que?}). Podemos facilmente encontrar um {\it vetor diretor} calculando 
  $N_1 \times N_2$, onde 
  $N_1$ e $N_2$ são vetores normais ao planos $\pi_1$ e $\pi_2$ respectivamente. 
  
  {\bf Ângulo entre planos. } Em $\mathbb{R}^{3}$, podemos definir o ângulo entre dois planos $\pi_1$ e $\pi_2$ 
 como o ângulo que satisfaz a relação 
  $$\cos (\pi_1,\pi_2)= \frac{|N_1 \circ N_2|}{\|N_1\|\|N_2\|}, $$
  onde  $N_1$ e $N_2$ são vetores normais ao planos $\pi_1$ e $\pi_2$ respectivamente. Observe que na formula
   usamos o {\it valor absoluto} de  $N_1 \circ N_2$ em lugar de 
   $N_1 \circ N_2$.
   
 {\bf Distância de um ponto a um plano em $\mathbb{R}^{3}$}.
 Considere um plano $\pi: ax+by+cz+d=0$ e um ponto 
 $P=(x_{1}, y_{1}, z_1) \in \mathbb{R}^{3}$. 
 A distância de ponto $P$ ao plano $\pi$ é dado pela formula
 
 \begin{equation}
 \text{dist}(P,\pi)= \|\text{proj}_{N}\overrightarrow{P_0P}\|, 
 \label{eqn:distanciapp2}
 \end{equation}
 onde $P_0$ é um ponto em $\pi$ e $N$ é um vetor normal ao plano.
 Note que podemos usar $N=(a,b,c)$.
 
 
\begin{enumerate}
  \item Faça um esboço dos seguintes planos em $\mathbb{R}^{3}$.
     \begin{enumerate}
     \item $2x+3y+5z-1=0$
     \item $3y+2z-1=0$
     \item $2x+3z-1=0$
     \item $3x+2y-4=0$
     \end{enumerate}  
   \item Encontre a equação geral do plano paralelo 
   a $2x-y+5z-3=0$ e passa no ponto $P=(1,-2,1)$.
   {\it Rpta} $\pi: 2x-y+5z-9=0$.
   \item Ache a equação do plano que contem 
   $P=(2,1,5)$ e é perpendicular aos planos 
    $x+2y-3z+2=0$ e $2x-y+4z-1=0$.
    {\it Rpta} $x-2y-z+5=0$. {\it Dica} Use o produto vetorial.
    \item Considere as retas 
    $$ r: \frac{x-2}{2}=\frac{y}{2}=z \text{ e }
       s:  x-2=y=z.$$
     Obtenha a equação geral do plano determinado por 
     $r$ e $s$.  
     \item Sejam dois planos 
      $\pi_1: x+2y+z+2=0$ e $x-y+z-1=0$.
      Encontre o plano que contém 
      a interseção $\pi_1 \cap \pi_2$ e 
      é ortogonal ao vetor $U=(1, 1, 1)$.
      {\it Rpta: } $x-2y+z-2=0$.
      \item Encontre a equação do plano 
      que passa por $A=(1,0,-2)$ e contém 
      $\pi_1 \cap \pi_2$, onde 
      $\pi_1: x+y-z=0$ e $\pi_2: 2x-y+3z-1=0$.
      \item Qual é a equação paramétrica da reta 
      que é interseção dos planos, 
      $\pi_1: (x,y,z)=(1+\alpha, -2, -\alpha-\beta)$ e 
      $\pi_2: (x,y,z)=(1+\alpha-\beta, 2\alpha+\beta, 3-\beta)$?
      \item Sejam três vetores 
      $V=i+3j+2k$, $W=2i-j+k$ e $U=i-2j$ em $\mathbb{R}^{3}$. 
      Se $\pi$ é um plano paralelo aos vetores 
      $W$ e $U$, e $r$ é uma reta perpendicular ao plano $\pi$.
      Encontre a projeção ortogonal de $V$ 
      sobre o vetor diretor da reta $r$.  
      \item Em $\mathbb{R}^{3}$, considere os pontos 
      $A=(2,-2,4)$ e $B=(8,6,2)$.
      Encontre o lugar geométrico dos pontos equidistantes de $A$ e $B$. {\it Dica :} É um plano.
      
      \item Seja $\pi$ um plano que forma 
      um ângulo de $60^{\circ}$ com o plano 
      $\pi_1: x+z=0$ e contém a reta
      $r: x-2y+2z=0, 3x-5y+7z=0$.
      Encontre a equação do plano $\pi$.
      {\it Rpta} Dois soluções: 
      $y+z=0$ ou $4x-11y+5z=0$.      
      %$x-2y+2z=0$ ou $x+4y+8z=0$. 
      \item  O plano $\pi: x+y-z-2=0$ intercepta os 
      eixos cartesianos aos pontos $A$, $B$ e $C$. 
      Qual é a área do triângulo $ABC$? {\it Rpta} $2\sqrt{3} u^2$.
      \item Considere os planos: 
      $$ \pi_1: x-y+z+1=0, \ \ \pi_2: x+y-z-1=0, \ \ 
         \pi_3: x+y+2z-2=0. $$
      Encontre a equação geral que contém $\pi_1\cap \pi_2$ e perpendicular 
      $\pi_3$.   
      \item Ache o ângulo entre o plano $-2x+y-z=0$ e plano que passa por
      $P=(1,2,3)$ e é perpendicular $a i-2j+k$. {\it Rpta:} $arccos(5/6)$.
      \item Para quais valores de $\alpha$ e $\beta$, a reta 
      $r: (\beta, 2, 0)+t(2, \alpha, \alpha)$ está contida no plano 
      $\pi: x-3y+z=1$. {\it Rpta: } $\alpha=1$, $\beta=7$.
      \item Encontre o valor de $\alpha$ para que 
      os planos $\pi_1: (1,1,0)+t(\alpha, 1, 1)+s(1,1, \alpha)$
      e $\pi_2: 2x+3y+2z+1=0$ sejam paralelos. {\it Rpta: }
      $\alpha=1/2$.
      \item Encontre a equação geral do plano 
      $\pi$ que contém a reta $r: (1,0,1)+t(1,1,-1)$ e dista 
      $\sqrt{2}$
      do ponto $P=(1,1,-1)$. 
\end{enumerate}

\end{document}

