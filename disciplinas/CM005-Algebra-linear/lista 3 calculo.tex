% lista 3(calculo 1)
\documentclass[latin,20pt]{article}
%\usepackage{amssymb,latexsym,amsthm,amsmath}
\usepackage[paper=a4paper,hmargin={1cm,1cm},vmargin={1.5cm,1.5cm}]{geometry}
\usepackage{amsmath,amsfonts,amssymb}
\usepackage[utf8]{inputenc}

%\usepackage{stmaryrd} %%para graficar maximo inteiro 
\begin{document}

\title{Lista 3: Cálculo I }
 
\author{
A. Ramos \thanks{Department of Mathematics,
    Federal University of Paraná, PR, Brazil.
    Email: {\tt albertoramos@ufpr.br}.}
}

\date{\today}
 
\maketitle

\begin{abstract}
{\bf Lista em constante atualização}.
 \begin{enumerate}
 \item Continuação de límites e continuidade. 
 \end{enumerate}
\end{abstract}

%%%%%%%%%%%%%%%%%%%%%%%%%%%%%%%%%%%%%%%%%%%%%%  
%\section*{Elipse} 
%Seja $\mathcal{O}$ um aberto em $\mathbb{R}^{n}$. 
%Denote por 
%$C^{1,1}_{L}(\mathcal{O})$ o conjunto das funções deriváveis 
%em $\mathcal{O}$ cuja derivada é Lipschitziana com constante de 
%     Lipschitz $L$ em $\mathcal{O}$, isto é, 
%     $\|\nabla f(x)-\nabla f(y)\|\leq L\|x-y\|$, 
%     para todo $x,y \in \mathcal{O}$.
 
  \section{Exercícios}   
 
 Faça do livro texto, os exercícios correspondentes aos temas desenvolvidos em aula. 
  
  \section{Exercícios adicionais} 
    \subsection{Limites ao infinito}   
   Calcule os seguintes limites.  
    \begin{enumerate}
    \item $$\lim_{x \rightarrow \infty} 
    \frac{\sqrt{x+\sqrt{x+\sqrt{x+4}}}}{\sqrt{x+2}}. $$
    {\it Rpta: } 1.    
     \item $$\lim_{x \rightarrow \infty} 
    \left( \frac{x^3}{2x^2-1}-\frac{x^2}{2x+1}\right). $$
    {\it Rpta:  }$1/4$.
    \item Seja 
    $$
    f(x)=\frac{3x+|x|}{7x-5|x|}, \text{ para todo } x \in \mathbb{R}.
    $$ 
    Encontre 
    $\lim_{x \rightarrow \infty} f(x)$ e 
    $\lim_{x \rightarrow -\infty} f(x)$.
    {\it Rpta: } $2$ e $1/6$. 
    \item Seja $\alpha \in \mathbb{R}$ tal que 
    $\lim_{x \rightarrow \infty} 
    \left( \frac{\alpha x-1}
    {\alpha x +1}\right)^{x}=4$. Encontre $\alpha$. {\it Rpta: } 
    $\alpha=-1/ln(2)$.
     \item Encontre o maior número $\alpha$ de tal modo que 
    $$\lim_{x \rightarrow \infty} 
    \frac{\alpha x^{\alpha-1}+2x^{\alpha}}
    {\sqrt{3x^2+1}} \text{ seja finito }.$$
    Calcule dito limite. {\it Rpta: } $\alpha=1$, $L=2\sqrt{3}/3$.
     \item Calcule 
     $\lim_{x \rightarrow \infty} \left( \cos\sqrt{x+1}
     -\cos \sqrt{x} \right)$. {\it Rpta: } 0. {\it Dica:} Use a diferença de cossenos. 
    \item Ache as constantes $m$ e $b$ para que 
    $\lim_{x \rightarrow \infty} \left( mx+b- 
    \frac{x^3+1}{x^2+1}\right)=0$. {\it Rpta: } $m=1$, $b=0$. 
    \item Calcule 
    $\lim_{x \rightarrow \infty} \left(3^{x}+2^{x}\right)^{1/x}$. 
    {\it Rpta:} 3.
    \item Ache o valor de a para que 
    $$ \lim_{x \rightarrow \infty} 
    \left( 
    \frac{x^4+ax^3+1}{x^3-x+1}- 
    \sqrt{x^2+3x+10}
    \right)=\frac{3}{2}. $$
    {\it Rpta: } $a=3$. Talvez $x=1/u$ ajude.
    \end{enumerate}
    \subsection{Limites Infinitos}   
   Calcule os seguintes limites.  
    \begin{enumerate}
    \item $$\lim_{x \rightarrow 0^{+}} \frac{2-4x^{3}}{10x^{2}+6x^{3}}=\infty. $$
    \item $$\lim_{x \rightarrow 3^{-}} 
    \frac{\lbrack\!\lbrack x\rbrack\!\rbrack -x}{3-x}=-\infty. $$
    \item $$\lim_{x \rightarrow 2^{-}} 
    \frac{3x^{2}-7x+6}{x^2+x-12}=-\infty. $$
    \item $$\lim_{x \rightarrow 4} 
    \frac{1+|x^{2}-16|}{(4-x)\sqrt{5-|x+1|}}=\infty. $$
    \item $$\lim_{x \rightarrow 2} 
    \left( \frac{1}{x-2}-\frac{3}{x^2-4} \right)=\infty. $$
    \end{enumerate}
   \subsection{Limites do exponencial e do logaritmo}   
   Calcule os seguintes limites.  
    \begin{enumerate}
    \item $\lim_{u \rightarrow \infty} 
    \left(\frac{u-3}{1+u}\right)^{u-2}=e^{-4}$.
    \item $\lim_{h \rightarrow 0} 
    \frac{\ln(x+h)-\ln(x)}{h}=\frac{1}{x}$.
    \item $\lim_{h \rightarrow 0} 
    \frac{\ln(\cos h)}{h^2}=-\frac{1}{2}$.
    \item Seja $n \in \mathbb{N}$. Verifique $\lim_{x \rightarrow 0} 
    \left(x+e^{x}\right)^{\frac{n}{x}}=e^{2n}$.
    \item $\lim_{x \rightarrow \pi/2} 
    \cos(x)^{\tan(x)}=1$.
    \item $\lim_{x \rightarrow 0} 
    \left(\sqrt{2-\sqrt{\cos x}}\right)^{\frac{1}{x^2}}=e^{\frac{1}{8}}$.
     \item $\lim_{x \rightarrow 0} 
    \frac{\sin^{2}(3x)}{\ln^{2}(1+2x)}=\frac{9}{4}$
   \item $\lim_{x \rightarrow \infty} 
    \left(\cos(\frac{a}{x})+b\sin(\frac{a}{x})\right)^{x}=e^{ab}$
    \item Considere $a$, $b$ e $c$ números estritamente positivos. Calcule, $\lim_{x \rightarrow 0} 
    \left(\frac{a^x+b^x+c^x}{3}\right)^{\frac{1}{x}}$
    \end{enumerate}    
    \subsection{Continuidade, teorema do valor intermediario e teorema de Weierstrass}   
    \begin{enumerate}
    \item Determine para quais valores de $x$ a função $f$ não é contínua.
       \begin{enumerate}
       \item  $$
    f(x)= \left\{  
            \begin{array}{lll}
    &x^2-9 &\text{, se } x \in (-\infty, -3] \\
    &x     &\text{, se } x \in (3,\infty) \\
            \end{array}
            \right. 
    $$
    {\it Rpta:} Descontínua em $x=3$.
       \item 
        $$
    f(x)= \left\{  
            \begin{array}{lll}
    & sgn\left( x^{2}-\frac{1}{4}\right) &\text{, se } x \in (-\infty, -1] \\
    &\frac{x^3}{x^2-9}     &\text{, se } |x| < 1 \\
    &-\frac{1}{8}+\sqrt{x^2-2x+1}     &\text{, se } x \in [1,\infty) \\
            \end{array}
            \right. 
    $$
    {\it Rpta: } Descontínua \footnote{Lembre: A função sinal $sng(x)$
    é definida como $sgn(x)=-1$ se $x<0$, $sgn(x)=1$ se $x>0$, e
    $sgn(x)=0$ se $x=0$} em $x=-1$. 
    \item 
     $$
    f(x)= \left\{  
            \begin{array}{lll}
    &\frac{\lbrack\!\lbrack x-1\rbrack\!\rbrack+\lbrack\!\lbrack 1-x\rbrack\!\rbrack}{2\sqrt{|x|-\lbrack\!\lbrack x\rbrack\!\rbrack}} &\text{, se } x \in (0,2) \ \ \text{ e } \ \  x \neq 1 \\
    & 2x-5    &\text{, se } x \in (2,\infty) \\
            \end{array}
            \right. 
    $$
    {\it Rpta:} Contínua em $x=2$, descontinua em $x \in \{0,1\}$.
       \end{enumerate}
    \item Determine o valor de $a$ para que $f$ seja contínua.
      \begin{itemize}
      \item $$
    f(x)= \left\{  
            \begin{array}{lll}
    &\frac{x^4-x^2+x-1}{x-1} &\text{, se } x \neq 1 \\
    & a    &\text{, se } x=1\\
            \end{array}
            \right. 
    $$
    {\it Rpta: } $a=3$
      \item $$
    f(x)= \left\{  
            \begin{array}{lll}
    &\frac{\sqrt{-x}-1}{x+1} &\text{, se } x <-1 \\
    & x+a    &\text{, se } x \in [1,\infty) \\
            \end{array}
            \right. 
    $$
    {\it Rpta:} $a=1/2$.
      \end{itemize} 
    \item    Determine os valores de $a$  e $b$ para que $f$ seja contínua.            
       \begin{enumerate}
       \item 
       $$
    f(x)= \left\{  
            \begin{array}{lll}
    &\frac{3-(3+3x)^{1/3}}{a(x^{1/3}-2)} &\text{, se } x < 8 \\
    &ab    &\text{, se } x=8\\
    &\frac{2}{b|2x-7|}    &\text{, se } x>8\\
            \end{array}
            \right. 
    $$
       {\it Rpta: } $a=2$, $b=-1/3$.
       \item 
       $$
    f(x)= \left\{  
            \begin{array}{lll}
    &b \lbrack\!\lbrack 4+3x \rbrack\!\rbrack &\text{, se } x \in [1,2) \\
    &3x\sqrt{a-2x}    &\text{, se } x \in (2,3)\\
    &18    &\text{, se } x=2\\
            \end{array}
            \right. 
    $$
       {\it Rpta: } $a=13$, $b=2$.
       \end{enumerate}
    \item Mostre que 
    $4x-3+\cos(\frac{\pi x}{2})=0$ tem 
    alguma raiz real.
    \item Mostre que os gráficos de 
    $y=x^{2}\tan(x)$ e $y=1$ têm 
    interseção em pelos menos 
    um ponto do intervalo 
    $(-\frac{\pi}{2}, \frac{\pi}{2})$. 
    \item Forneça um exemplo de uma função $f$ que em dois pontos distintos $a$ e $b$, tem sinais contrários, que não seja continua em $[a,b]$ e a tese do teorema do valor intermediário é verdadeira.
    \item Considere as seguintes afirmações e decida se é verdadeira ou falsa, apresentando um contra-exemplo ou justificando através de uma demonstração.
      \begin{enumerate}
      \item Se $f:\mathbb{R}\rightarrow \mathbb{R}$
      é tal que $\lim_{x\rightarrow a} |f(x)|=0$, então $\lim_{x\rightarrow a}f(x)=0$.
      \item Se $f:\mathbb{R}\rightarrow \mathbb{R}$
      é tal que $|f|$ é continua em $x=0$, 
      então $f$ é continua em $x=0$. 
      \item Considere duas funções  $f$ e $g$ descontínuas em $x=0$, então o produto $fg$ é descontínua em $x=0$. 
      \end{enumerate}       
    \item Considere $f(x)=e^{x}-\frac{1}{x}+\frac{x}{2}$, definido para  
    $x>0$. 
      \begin{itemize}
      \item Mostre  que $f:\mathbb{R}_{+}\rightarrow \mathbb{R}$ é uma função bijetora.
    Para isso mostre que 
      $$e^{x}-\frac{1}{x}+\frac{x}{2}=y$$
      admite uma única solução para qualquer 
      $y \in \mathbb{R}$.
     % \item Denote $g$ a inversa de $f$. 
     % Mostre que $|g(x)-g(y)|\leq 2|x-y|$, para todo $x, y \in \mathbb{R}$.
      \end{itemize}  
      \item {\it Teorema do ponto fixo}. Seja $f:[0,1] \rightarrow \mathbb{R}$ uma função contínua tal que $f(x) \in [0,1]$ para todo $x$.
      Então, mostre que existe um $c \in [0,1]$ tal que $f(c)=c$. 
      Esse ponto é chamado de ponto fixo de $f$. 
    \end{enumerate}
     \subsection{Problemas diversos}   
      \begin{enumerate}
      \item Considere um círculo de raio $9$, e denote por $C_{1}(d)$
      e por $C_{2}(d)$ o comprimentos de dois cordas \footnote{ Uma corda é um segmeto que une dois pontos sobre o círculo.} cuja distância ao centro do circulo é d e $\frac{9+d}{2}$ respectivamente, com $d \in (0,9)$. Então, calcule $\lim_{d \rightarrow 9} \frac{C_{1}(d)+C_{2}(d)}{C_{1}(d)}
      (=\frac{2+\sqrt{2}}{2})$ 
      \item Considere um setor circular de radio $R=R(\theta)>0$ cujo ângulo central é $\theta$ e
      considere um triângulo equilátero $ABC$ de lado $L$ inscrito no setor circular tal que $C$ está sobre o semi-circulo e o segmeto $OC$ passa pelo ponto médio de $AB$. Encontre,   
      $\lim_{\theta \rightarrow \pi/3} \frac{R(\theta)-L\sqrt{3}}{3x-\pi}
      (=-\frac{L}{3})$. 
      \item Seja uma caixa fechada de volume 2000 $m^{3}$. O material para as partes superior e inferior é de 3 R\$ por metro quadrado, e o material para os lados é de 1.5 R\$ por metro quadrado. Se $x$ representa o comprimento (em metros) de um lado da base quadrada, e $C(x)$ o costo total do material. 
        \begin{enumerate}
        \item Escreva a expressão que define $C(x)$ e especifique o dominio. 
        \item Calcule $\lim_{x \rightarrow 0^{+}} C(x)$ e 
        $\lim_{x \rightarrow \infty} C(x)$. Explique os resultados, em termos do problema. 
        \item Faça o gráfico de $C$. 
        \end{enumerate}
      \end{enumerate}
\end{document}



\item Considere a função 
     $$
    f(x)= \left\{  
            \begin{array}{lll}
    &\frac{x^3+3x^2-9x-27}{x+3} &\text{, se } x \in (-\infty, -3) \\
    &ax^2-2bx+1     &\text{, se } x \in [-3,3] \\
            & \frac{x^2-22x+57}{x-3}     &\text{, se } x \in (3,\infty) \\
            \end{array}
            \right. 
    $$
    Para quais valores de $a$ e $b$, existe os limites de $f$ em $x=-3$ e $x=3$? {\it Rpta: } $a=-1, b=4/3$.