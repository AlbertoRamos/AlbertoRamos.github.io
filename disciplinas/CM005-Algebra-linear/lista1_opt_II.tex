% lista 5(geometria analitica 2017 I)
\documentclass[a4paper,latin]{article}
%\usepackage{amssymb,latexsym,amsthm,amsmath}
\usepackage[paper=a4paper,hmargin={1cm,1cm},vmargin={1.5cm,1.5cm}]{geometry}
\usepackage{amsmath,amsfonts,amssymb}
\usepackage[utf8]{inputenc}

\begin{document}

\title{Lista 1: Otimização II }

\author{
A. Ramos \thanks{Department of Mathematics,
    Federal University of Paraná, PR, Brazil.
    Email: {\tt albertoramos@ufpr.br}.}
}

\date{\today}
 
\maketitle

\begin{abstract}
{\bf Lista em constante atualização}.
 \begin{enumerate}
 %\item Transformação de coordenadas;
 %\item Equação da parabóla; %Equação da reta e do plano; %Vetores (no plano e no espaço);
 \item Métodos de gradiente; 
 \item Método de Newton e variantes.
 \end{enumerate}
\end{abstract}

%%%%%%%%%%%%%%%%%%%%%%%%%%%%%%%%%%%%%%%%%%%%%%  
%\section*{Elipse} 
Seja $\mathcal{O}$ um aberto em $\mathbb{R}^{n}$. 
Denote por 
$C^{1,1}_{L}(\mathcal{O})$ o conjunto das funções deriváveis em $\mathcal{O}$ 
cuja derivada é Lipschitziana com constante de 
     Lipschitz $L$ em $\mathcal{O}$, isto é, 
     $\|\nabla f(x)-\nabla f(y)\|\leq L\|x-y\|$, 
     para todo $x,y \in \mathcal{O}$.
        
 Com essas informações responda:
    \begin{enumerate}
    \item Considere a função 
    $f: \mathbb{R}^{n}\rightarrow \mathbb{R}$ definida como 
    $f(x)=x^{T}Ax+b^{T}x+c$.
    Mostre que 
      \begin{itemize}
      \item  $\nabla f(x)=Ax+A^{T}x+b$.
      E se $A$ é simetrico ($A=A^{T}$), $\nabla f(x)=2Ax+b$
      \item  $\nabla^{2} f(x)= A^{T}+A$.
      E quando $A$ é simetrico,
      $\nabla^{2} f(x)=2A$ 
      \item No caso que $A$ é simétrica. 
      Mostre que $f$ admite solução se, e somente se $A$ é definida positiva. 
      \end{itemize}
   \item Seja $f$ uma função 
   continuamente derivável em $\mathbb{R}^{n}$.
   Suponha que $d$ é uma direção de descida de $f$ em $x$.
   Mostre que existe um número $T>0$ tal que 
   $$f(x+td)<f(x) \ \ \forall t \in (0,T].$$
   % Dê um exemplo onde a conclusão é falsa, se $f$ só for derivável.
   \item Seja $d$ uma direção de descida para uma função derivável $f$
   no ponto $x \in \mathbb{R}^{n}$. 
   Mostre que se $f$ tem derivada contínua, o ponto $x^{+}:=x+td$ está bem definido, onde $t$ é escolhido 
   segundo as seguintes condições 
        \begin{enumerate}
        \item A condição de Armijo, a condição de Goldstein
        \item A condição de Wolfe e a condição de Wolfe forte.
        \end{enumerate}
    \item Considere o problema 
    $$ \text{ min } f(x):=x_1^{2}+2x_2^{2}-x_1x_2-2x_1+e^{x_1+x_2}, 
    \text{ s.a. } x=(x_1,x_2) \in \mathbb{R}^2.$$
      \begin{enumerate}
      \item Verifique que $(0,0)$ não é solução do problema
      \item Minimize a função a partir de $(0,0)$ ao 
      longo da direção de máxima descida.
      \end{enumerate}  
    \item Seja $f(x):=x_1^{4}+x_1^{2}+x_2^{2}$. Considere o 
    ponto $x^{k}:=(1,1)^{T}$ e a direção $d^{k}:=(-3,-1)^{T}$.
     \begin{enumerate}
     \item Verifique que $d^{k}$ 
     é uma direção de descida para $f$ em $x^k$.
     \item Use a condição de Wolfe para encontrar o novo ponto 
     $x^{k+1}:=x^{k}+td^{k}$, com parámetros $\rho:=0.1$ e $\sigma:=0.5$.
     
     Isto é:
     $$f(x^{k}+td^{k})\leq f(x^k)+t\rho\nabla f(x^{k})^{T}d^{k}
      \ \  (\text{ condição de descrescimo suficiente})$$ 
     e
      $$\nabla f(x^{k}+td^{k})^{T}d^{k}\geq 
      \sigma \nabla f(x^k)^{T}d^{k}, \sigma \in (\rho,1) \ \ 
      (\text{ condição sobre a curvatura}). $$
     \item Considere os valores de $t=1$, $t=0.5$ e $t=0.1$ respectivamente. Quais valores satisfazem a condição de Wolfe?
     \end{enumerate}            
    \item Seja $f:\mathbb{R}^{n}\rightarrow \mathbb{R}$ uma função derivável cuja derivada é uma função lipschitziana com constante de Lipschitz $L$ 
    (i.e. $\|\nabla f(x)-\nabla f(y)\|\leq L\|x-y\|$, para todo $x,y \in \mathbb{R}^{n}$).
    Mostre que
    $$ f(y)\leq f(x)+\nabla f(x)^{T}(y-x)+\frac{L}{2}\|x-y\|^{2}, \text{ para todo }x, y \in \mathbb{R}^{n}.$$    
    Isto é, a função quadrática 
    $Q(y):=f(x)+\nabla f(x)^{T}(y-x)+\frac{L}{2}\|x-y\|^{2}$
    sobre estima a função $f$ em todo $\mathbb{R}^{n}$     
    \item 
    Seja $f$ uma funçaõ duas vezes derivável em $\mathbb{R}^{n}$.
    As seguintes proposições são equivalentes:
      \begin{enumerate}
      \item  A derivada 
      de $f$ Lipschitziana 
      com constante de Lipschitz $L$
      \item $\|\nabla^{2} f(x)\|\leq L$,    para todo $x \in \mathbb{R}^{n}$.
      \end{enumerate} 
    \item 
    Seja $f \in C^{1,1}_{L}(\mathbb{R}^{n})$ e $\{x^{k}\}$ uma sequência generada pelo método do gradiente com passo constante $t_{k}=1/L$.
    Suponha que $x^{k}\rightarrow x^{*}$.
    Prove que se $\nabla f(x^{k})\neq 0$, para $k \in \mathbb{N}$. Então, $x^*$ não é um
    máximo local.  
    \item Seja $x^{0} \in \mathbb{R}^{n}$ um ponto inicial e seja $f \in C^{1,1}_{L}(\mathcal{O})$, onde 
    $\mathcal{O}$ é um aberto que contem o conjunto de nível 
   $ \{x \in \mathbb{R}^{n}: f(x)\leq f(x^0)\}$. Adicionalmente suponha que $f$ é limitada inferiormente.
    
    Se $\{x^{k+1}:=x^{k}+t_{k}d^{k}\}$
    é uma sequência de iterados, onde 
    $d^{k}$ é uma direçaõ de descida.
    
    Mostre que se  
    $t_{k}$ satisfaz (i) a condição de Wolfe, ou (ii) a condição de Goldstein ou (iii) a condição de Wolfe-forte. Então, a condição de Zoutendijk
     é satisfeita i.e. 
     $\sum_{k=1} \cos^{2}(\theta_{k})\|f(x^{k})\|^{2}< \infty$, 
     onde 
     $\cos(\theta_{k}):=-{d^{k}}^{T}\nabla f(x^k)/\|d^k\|\|\nabla f(x^k)\|$. 
    \item Considere o problema de minimizar 
    $\text{min} 
    \{x^{T}Ax, x \in \mathbb{R}^2\}$, 
    onde $A$ é uma matriz $2\times 2$ definida positiva. 
    Considere 
    $D=\begin{pmatrix}
      A_{11}^{-1} & 0 \\
      0   & A_{22}^{-1}.
      \end{pmatrix}$
    
    Prove que $\mathcal{K}(D^{1/2}AD^{1/2})\leq \mathcal{K}(A)$. 
    Interprete e analise o método do gradiente para este caso.   
    \item Seja $f:\mathbb{R}^{2}\rightarrow \mathbb{R}$ definida por
    $ f(x)=\frac{1}{2}x_1^{2}+\frac{1}{4}x_2^{4}-\frac{1}{2}x_2^{2}$.
    Responda:
        \begin{enumerate}
        \item Determine e classifique os pontos estacionários.
        \item Faça uma iteração do método de 
        gradiente com ponto inicial $x^{0}=(1,0)^{T}$. 
        Discuta a possível convergência do método de gradiente. 
        \end{enumerate}
     \item Seja $f:\mathbb{R}^{2}\rightarrow \mathbb{R}$ definida por
  $f(x)=\frac{1}{2}x_1^{2}+\frac{a}{2}x_2^{2}$, onde $a\geq 1$.
     Mostre que o método de gradiente, com ponto inicial $x^{0}=(a,1)^{T}$, gera 
     a seguinte sequência 
     $$ x^{k}:=(x_1^k,x_2^k)^{T}=
     \left(\frac{a-1}{a+1}\right)^{k}
     (a, (-1)^{k})^{T}, \text{ para todo }
     k \in \mathbb{N}.$$         
     \item Seja $f:\mathbb{R}^{2}\rightarrow \mathbb{R}$ definida por
    $ f(x)=5x_1^{2}+5x_2^{2}-x_1x_2+11x_2-11x_1+11$.
    Responda:
        \begin{enumerate}
        \item Calcule a taxa de convergência de $\|x^k-x^*\|$ 
        e de $f(x^k)-f(x^*)$.
        \item Considere $x^{0}=(0,0)^{T}$. 
        Quantas iterações são necessárias para 
        obter uma precisão de $10^{-8}$ no valor ótimo de $f$?. 
        \end{enumerate}    
     \item Considere    
     $ f(x)=\frac{1}{2}(x_1^{2}-x_2)^{2}+\frac{1}{2}(1-x_1)^2$.
    Responda:
        \begin{enumerate}
        \item Calcule o minimizador de $f$
        \item Calcule uma iteração do método de Newton 
        para minimizar $f$ a partir de $x^{0}=(2,2)^{T}$. 
        Esse passo é aceitável?
        {\it Dica:} Calcule $f(x^0)$ e $f(x^1)$. 
        \end{enumerate}        
    \item Em $\mathbb{R}^{n}$, considere
    $f(x):=\|x\|^{3}$. Faça o método de Newton com passo constante $t_{k}=1$.
    Mostre que o método converge linearmente para o mínimo $x^{*}=0$.
    Por quê não temos convergência quadrática?    
    \item Denote $M(n,\mathbb{R})$ o conjunto de matrices reaias. Seja $GL(n,\mathbb{R})$ o conjunto das matrices não singulares.
    A função que associa cada matriz com a sua inversa, 
    $Inv: GL(n,\mathbb{R}) \rightarrow GL(n,\mathbb{R})$, $Inv(A)=A ^{-1}$ é infinitamente derivável.
    Para isto faça o seguinte:
      \begin{enumerate}
      \item Primeiro, calcule as 
      derivadas de $Inv$ em $I$ 
      (onde $I$ é a matriz identidade) e mostre que $D^{k}Inv(I)[A,\dots,A]=(-1)^{k}k!A^{k}$.
      Para isto, use a formula de Neumann
      \footnote{Formula de Neumann: 
      Para $B \in M(n,\mathbb{R})$ 
      com $\|B\|<1$, temos que  
      $(I+B)^{-1}=
      \sum_{k=0}^{\infty} (-1)^{k}B^{k}$.}
      para escrever a expansão de $(I+tA)^{-1}$, 
      para $t$ suficientemente pequeno.
      \item No caso geral, 
      para $A \in GL(n,\mathbb{R})$ e 
      $B \in M(n,\mathbb{R})$ mostre que 
      $$
       D^{k}Inv(A)[B,\dots,B]=
       (-1)^{k}k! A^{-1} B A^{-1} B A^{-1}\dots B A^{-1}, 
      $$ 
      onde temos $k+1$ matrices $A$ e $k$ matrices $B$. 
      {\it Dica:} Escreva $A+tB=A(I+tA^{-1}B)$ e use o item anterior.
      \item Descreva o método de Newton para 
      calcular a inversa de uma matriz. 
      \end{enumerate}
   \item Considere uma matriz definida positiva $A \in M(n,\mathbb{R})$. (i) Mostre que 
   $\|x\|_{A}:=\sqrt{x^{T}Ax}$ é uma norma em $\mathbb{R}^{n}$.
   (ii) Ainda mais, prove que 
   $\sqrt{\lambda_{min}(A)}\|x\|_{2}\leq \|x\|_{A} \leq \sqrt{\lambda_{max}(A)}\|x\|_{2}$ para todo $x \in \mathbb{R}^{n}$, 
   onde $\|\cdot\|_{2}$ é a norma euclideana.
      
    (iii) Use o resultado anterior para provar que a sequência $x^{k}$ generada pelo método de máxima descida (com busca exata)  
      aplicado ao problema 
      $\min f(x):=(1/2)x^{T}Ax$ converge à solução $x^*$ de dito problema e 
     $$ \frac{\|x^{k+1}-x^*\|}{\|x^k-x^*\|}\leq \sqrt{\mathcal{K}} \left(\frac{\mathcal{K}-1}{\mathcal{K}+1}\right), \text{ onde } \mathcal{K}=\frac{\lambda_{min}(A)}{\lambda_{max}(A)}. $$ 
   \item {\bf Direções de curvaura negativa}. Considere uma função de classe $C^{2}(\mathbb{R})$. Então:
      \begin{enumerate}
      \item Se $\nabla^{2} f(x)$ tem um autovalor negativo, dizemos que $x$ é uma ponto indefinido.
      \item Se $x$ é um ponto indefinido e 
      existe uma direção $d$ 
      tal que $d^{T}\nabla^{2}f(x)d<0$.
      Dito vetor é chamado de 
      {\it direção de curvatura negativa}.
      \item Se existe um par de 
      vetores $(z,d)$ tal que
      $$ 
         \nabla f(x)^{T}z \leq 0, \ \ 
         \nabla f(x)^{T}d \leq 0, \ \
         d^{T}\nabla^{2}f(x)d<0, 
      $$ 
      dizemos que $(z,d)$ é um par de descida no ponto indefinido $x$.
      No caso que $x$ não é um ponto indefinido (i.e. 
      $\nabla^{f}(x) \succeq 0$)
      se  $(z,d)$ satisfaz
       $$ 
         \nabla f(x)^{T}z < 0, \ \ 
         \nabla f(x)^{T}d \leq 0, \ \
         d^{T}\nabla^{2}f(x)d=0, 
      $$ 
      dizemos que $(z,d)$ é um par de descida no ponto $x$.     
      \end{enumerate}
   Com um par de descida $(z^k,d^k)$ podemos fazer busca ao longo de uma curva da forma 
   $$ x(t):=x^{k}+
   \phi_1(t)z^{k}+\phi_{2}(t)d^{k}$$
   para certos $\phi_{1}$ e $\phi_2$, em lugar de fazer uma busca linear.
     \begin{enumerate}
     \item {\it Condição de Armijo de segunda-ordem}. Nesse caso, a busca é realizada ao longo de curvas da forma:
     $$ x(t):=x^{k}+
   t^2z^{k}+t d^{k}, \text{ i.e. } \phi_{1}(t)=t^{2}, \phi_{2}(t)=t. $$
   Considere $\rho, \gamma \in (0,1)$, 
   e ponha 
   $x^{k}(i):=x^{k}+
   \gamma^{2i}z^{k}+\gamma^{i}d^{k}$.
  A condição de Armijo de segunda-ordem 
  pede por encontrar $i(k)\in \mathbb{N}$
  o menor inteiro não negativo $i$ tal que 
     $$ f(x^{k}(i))\leq 
     f(x^k)+
     \rho\gamma^{2i}
     (\nabla f(x^k)^{T}z^k+\frac{1}{2}{d^{k}}^{T}\nabla^{2}f(x^k)d^k)
     $$
     Atualize $x^{k+1}:=x^{k}(i(k))$.
     Mostre o seguinte:
        \begin{enumerate}
        \item O passo de Armijo de segunda-ordem está bem definido, se temos que  
        $\nabla f(x^{k})^{T}z^{k}<0$ 
        ( quando $\nabla f(x^{k})\neq 0$)
        e
   ${d^{k}}^{T}\nabla^{2} f(x^{k})d^{k}<0$ 
        (quando $\nabla f(x^{k})=0$).
        \item Seja $f \in C^{2}(\mathbb{R}^{n})$ tal que
        $\{ x \in \mathbb{R}^{n}: f(x)\leq f(x^0)\}$ é compacto.
        Seja $\{x^{k}\}$ uma sequência que satisfaz a condição de Armijo de segunda-ordem, e suponha que as sequências $\{\|z^k\|\}$ 
        e $\{\|d^{k}\|\}$ são limitadas. 
        Prove que :
 $$
 \nabla f(x^k)^{T}z^k \rightarrow 0 \ \ \text{e} \ \  
 {d^{k}}^{T}\nabla^2 f(x^k) d^k \rightarrow 0.  
 $$       
     \item Se adicionalmente às hipoteses do item anterior, temos que existem constantes $c_{1},c_2, c_3>0$ tal que 
     \begin{enumerate}
     \item $\|z^{k}\|\geq c_3 \|\nabla f(x^k)\|$
     \item 
     ${d^{k}}^{T}\nabla^2 f(x^k) d^k
     \leq c_{2} \lambda_{min}(\nabla^2 f(x^k))$ 
     ( lembre que $\lambda_{min}(A)$ denota o mínimo autovalor de $A$).
     \item
      $-\nabla f(x^k)^{T}z^k 
      \geq c_1 \|\nabla f(x^k)\|\|z^k\|$.
     \end{enumerate}
    Então,  qualquer ponto de acumulação $x^*$ de $x^{k}$ é um ponto estacionário de segunda-ordem, isto é, 
  $\nabla f(x^*)=0$ e $\nabla^{2} f(x^*)\succeq 0$.
        \end{enumerate}
     {\bf Obs:} Além da condição de segunda-ordem de Armijo, outras condições de segunda-ordem são a condição de Goldfard, a condição de Moré-Sorensen, etc.   
     \end{enumerate}     
   \item Seja 
   $F:\mathbb{R}^{2}
   \rightarrow \mathbb{R}^{2}$ 
   uma função cuja componentes são
   $F_{1}(x)=x_1^2+x_2^2-9$ e  
   $F_{2}(x)=x_1+x_2-3$.   
     \begin{enumerate}
     \item Avalie a Jacobiana de $F$
     em $x=(1,0)^{T}$ e $x=(1,5)^{T}$.
     \item Faça duas (três) iterações do método de Newton, para resolver $F(x)=0$, partindo de $x^0=(1,5)^{T}$.
     \item Escreva o problema $F(x)=0$, como um problema de otimização com função objetivo $f(x):=\|F(x)\|^2$. 
     Encontre o gradiente e a Hessiana da 
     função objetivo.
     \end{enumerate}                 
    \item Seja $f:\mathbb{R}\rightarrow \mathbb{R}$ definida $f(x):=x^3-x$.
    Construa um modelo linear de $f$.
    Nos pontos (1) $x=0$, 
    (ii) $x=\sqrt{3}/3$ e
    (iii) $x=2$. Explique o que acontece em cada uma das situações. 
   \end{enumerate}    
\end{document}   
%%%%%%%%%%%%%%%%%%%%%%%%%%%%%%%%%%%%%%%%%%%%%%  
\section*{Hipérbole} 
 
 Dado dois pontos $F_1$ e $F_2$ no plano, e dois números positivos $a$ e $c$ ($c> a$)
   com $dist(F_1,F_2)=2c$.  
   A hipérbole é o conjunto  
   $$\mathcal{H}:=\{P \in \mathbb{R}^{2}: |dist(P,F_1)-dist(P,F_2)|=2a\}.$$
   Os pontos $F_1$ e $F_2$ são chamados de focos.
   Defina $b:=\sqrt{c^{2}-a^{2}}$. 
   Por definição de $b$, 
   temos que $c^{2}=b^2+a^2$ ({\bf perceba as diferenças com a hipérbole}). 
   O número $e:=\frac{c}{a}$ é chamado 
   de {\it excentricidade} da hipérbole. 
   Veja que para a hipérbole $e>1$.
    \begin{enumerate}
      \item {\bf eixo focal (eixo transverso): } reta que contem os focos $F_1$ e $F_2$;
      \item {\bf vértices:  } Interseção do eixo focal com a 
      hipérbole. 
      A interseção são dois pontos denotados por $V_1$ e $V_2$;
      \item {\bf centro : } Ponto meio do segmento $F_1F_2$;
      \item {\bf eixo normal (eixo conjugado): } reta perperndicular ao eixo focal que passa pelo centro; 
      \item {\bf corda: } qualquer segmento de une dois 
      pontos diferentes da hipérbole;
      \item {\bf corda focal: } corda que passa por algum foco;
      \item {\bf lado reto : } corda focal paralela ao eixo normal;
      \item {\bf raio vetor: } segmento de reta que une algum 
      foco com algum ponto da hipérbole;
      %\item {\bf diámetro: } corda que passsa pelo centro.
      \item {\bf eixo maior: } segmento $V_1V_2$. 
      Observe que o eixo maior tem comprimento $2a$;
      \item {\bf eixo menor: } segmento definido pela interseção 
      da hipérbole com o eixo normal. 
      O eixo menor tem medida $2b$ ;
      \item {\bf retas diretrizes: }  retas paralelas à reta normal 
      cuja distância ao centro $C$ é $a/e$.
      \item {\bf rectângulo fundamental: } rectângulo cujo centro é o centro da hipérbole, com lados de comprimento 2a e 2b e paralelos aos eixo transveso e 
      conjugado respectivamente.
      \item {\bf assintotas: } retas que passam por $C$, não interceptam 
      à hipérbole mas tendem à hipérbole no infinito. 
      %aproxima-sem com distância zero.  
      Ditas retas são definas pelas diagonais do rectângulo fundamental. 
      \item {\bf ramo da hipérbole: }  cada uma das curvas que definem a hipérbole.
     \end{enumerate}
     Observe que 
     $$ dist(V_1,V_2)=2a \ \ (\text{ eixo maior da elipse }), \ \ 
        dist(F_1,F_2)=2c \ \ (\text{ distância focal }).$$
 
 {\it Remark 1: }Note que a hipérbole é simetrica em relação ao eixo focal e  ao eixo normal. 
 
 {\it Remark 2: } Veja a 
 construção geometrica da hipérbole na internet, 
 por exemplo, 
% \url{https://www.youtube.com/watch?v=ETV_bWAPOqU}. 
  
 Usando um sistema de coordenadas a hipérbole $\mathcal{H}$ 
 pode ser escrita com uma das seguintes formas. 
 
 {\bf Forma canônica }(também chamada de forma reduzida).
 Nesta caso, a hipérbole é o lugar geometrico definido por 
 %As equações são  
 $\frac{x^2}{a^2}-\frac{y^2}{b^2}=1$ ( hipérbole horizontal) ou $\frac{y^2}{a^2}-\frac{x^2}{b^2}=1$ ( hipérbole vertical), 
 onde o centro $C=(0,0)$ e o eixo focal é paralelo a algum dos eixos canônicos.
 {\it Desenhe ambas hipérbole explicitando 
 o segmento que tem comprimento $a$ e/ou $b$. Lembre $dist(V_1,V_2)=2a$}.
 
 {\it Remark:} Nesse caso as assíntotas podem ser facilmente calculadas. 
 De fato:
   \begin{enumerate}
   \item Quando $\mathcal{H}$ é uma hipérbole horizontal, 
   as assíntotas são as retas $y=\pm \frac{b}{a} x$;
   \item Quando $\mathcal{H}$ é uma hipérbole vertical, 
   as assíntotas são as retas $y=\pm \frac{a}{b} x$.
   \end{enumerate}
   
  Quando o centro $C=(h,k)$ e o eixo focal é paralelo a algum dos eixos canônicos, temos que a hipérbole pode ser descrita como 
  $\frac{(x-h)^2}{a^2}-\frac{(y-k)^2}{b^2}=1$ ou 
  $\frac{(y-k)^2}{a^2}-\frac{(x-h)^2}{b^2}=1$. 
  %$(y-k)^{2}=4p(x-h)$ ou $(x-h)^2=4p(y-k)$.

 {\bf Forma geral sem rotação} $Ax^2-Cy^{2}+Dy+Ex+F=0$.
 onde o eixo focal é paralelo a algum dos eixos canônicos.
 
 {\bf Forma geral mesmo} $Ax^2+Bxy+Cy^{2}+Dy+Ex+F=0$ se $B^{2}-4AC>0$. \newline
 
 {\it Retas tangentes para a hipérbole. }
 Em qualquer ponto sobre a hipérbole 
 podemos calcular retas tangentes e retas normais.
 
 {\bf Quando $\frac{x^2}{a^2}-\frac{y^2}{b^2}=1$}. 
 A reta tangente à $\mathcal{H}$ no ponto $P=(x_0,y_0) \in \mathcal{H}$ é dada por 
 $ r: (\frac{x_{0}}{a^2})x-(\frac{y_0}{b^{2}})y=1$. 
 
 {\bf Quando $\frac{y^2}{a^2}-\frac{x^2}{b^2}=1$}. 
 A reta tangente à $\mathcal{H}$ no ponto $P=(x_0,y_0) \in \mathcal{H}$ 
 é dada por 
 $ r: (\frac{y_0}{a^{2}})y-(\frac{x_{0}}{b^2})x=1$. \newline
  
 Com essas informações responda:
 \begin{enumerate}
     \item Calcule os focos, vértices, as equações das assíntotas. Esboce as hipérboles
       \begin{enumerate}
       \item $16x^2-25y^2=400  \ \  \text{ e } \ \ 9y^2-4y^2=36$
       \item $x^2-y^2+1=0  \ \ \text{ e } \ \ x^2-4y^2=1$ 
       \end{enumerate}
    \item Escreve a equação reduzida da elipse nos seguintes casos:
     \begin{enumerate}
     \item Os focos são $F_1=(3,-1)$ e $F_2=(3,4)$
     e satisfaz $|dist(P,F_1)-dist(P,F_2)|=3$;
     \item Os focos são $F_1=(-1,1)$ e $F_2=(1,1)$
     e satisfaz $|dist(P,F_1)-dist(P,F_2)|=1$;
     \item Os vértices são 
     $(2,0)$ e $(-2,0)$ e os focos são 
     $(3,0)$ e $(-3,0)$;
     \item Os vértices são 
     $(15,0)$ e $(-15,0)$ e as assíntotas são 
     $5y-4x=0$ e $5y+4x=0$.
     \end{enumerate} 
    \item Encontre a equação da hipérbole cujos focos 
    são $(4,0)$ e $(-4,0)$, e o coeficiente ângular duma das assíntotas 
    é 3. {\it Rpta: } $\mathcal{H}: 45x^2-5y^2=72$.
    \item Seja uma hipérbole com centro na origem, focos sobre o eixo x cuja distância entre as diretrizes é 4 e passa por $P=(4,3)$.
    {\it Rpta: } $\mathcal{H}: 3x^2-2y^2=30$
    \item Considere a elipse $\mathcal{E}: 25x^2+9y^2=225$. 
    Se os focos dessa elipse coincidem 
    com os focos duma hipérbole de excentricidade 4/3.
    Escreva a equação reduzida da hipérbole.
    {\it Rpta: } $\mathcal{H}: 7y^2-9x^2=63$.
    \item Calcule a àrea do triângulo formado por as assíntotas de 
    hipérbole $\mathcal{H}: x^2-4y^2=16$ e  a reta 
    $r: 3x-2y+12=0$. {\it Rpta: } $9 u^2$
    \item Encontre a equação reduzida de uma hipérbole se os focos são 
    os pontos $(-10,0)$ e $(10,0)$, e suas assíntotas são as retas 
    $r: y=\pm 2x$. {\it Rpta: } $\mathcal{H}: 4x^2-y^2=80$.
    \item Se as assíntotas duma hipérbole, 
    que tem um foco em $(3,-2)$, são 
    $r_1: 3x-4y-5=0$ e $r_2: 3x+4y+11=0$. 
    Encontre a sua excentricidade. {\it Rpta: } $e=5/4$. 
    \item É possível construir uma hipérbole com focos em 
    $(3,4)$ e $(-1,-2)$ tal que a {\bf medida do eixo maior} é 2.
    Caso afirmativo, escreva a equação de dita hipérbole.
    {\it Rpta: } Sim, $\mathcal{H}: 3x^2+8y^2+12xy-18x-28y+11=0$.
    {\it Dica: } Use a definição da hipérbole. 
    \item Considere a hipérbole $b^2x^2-a^2y^2=a^2b^2$, com 
    foco $F_1=(c,0)$, $F_2=(-c,0)$ e um ponto 
    $P=(x_0,y_0)$ da hipérbole. Mostre que 
    o raio vetor $PF_1$ é igual $|a-ex_0|$ 
    (i.e $|\overrightarrow{PF_1}|=|a-ex_0|$ )
    e  raio vetor $PF_2$ é $|a+ex_0|$ 
    (i.e $|\overrightarrow{PF_2}|=|a+ex_0|$ )
    \item {\it Propriedade refletora da hipérbole: }  
    distancia focal prova
      \begin{enumerate}
      \item as consequência várias coisas chevere.
      \end{enumerate}
 \end{enumerate} 
 
 \section*{Estudo unificado das cônicas não degeneradas}
     
\end{document}
 
 
 
 
 
 
 
 
 
 
 
 
 
 
 
 
 
 
 
 
 
 
 
 
 
 
 
 
 
 
 
 
 
 
 
 
\section*{Transformação de coordenadas}
 Existe dois tipos de transformações importantes: {\it traslação} e {\it rotação}, quais podem ser combinados para descrever movimentos mais complexos. É importante se sentir confortável com essas transformações.
 
% \shadowbox{gdfgd}
  
 {\it Translação de eixos.} Em $\mathbb{R}^{2}$ (em $\mathbb{R}^{3}$ é similar), considere um sistema de coordenadas cuja origem $O=(0,0)$ é trasladada a $O'=(h,k)$.
 Seja $P \in \mathbb{R}^{2}$ um ponto com coordenadas 
 $(x,y)$ no sistema de coordenadas original e 
 com coordenadas $(x',y')$ no 
 novo sistema de coordenadas (com origem $O'$). Então, temos que :
   $$  x=x'+h \ \ \text{ e } \ \ y=y'+k.$$
 {\it Rotação de eixos.} Em $\mathbb{R}^{2}$, considere dois sistemas de coordenadas, uma obtida apartir da outras atraves de uma rotação (com ângulo 
 $\theta$ e sentido antihorário). Se
 $P \in \mathbb{R}^{2}$ um ponto com coordenadas 
 $(x,y)$ no sistema de coordenadas original e 
 com coordenadas $(x',y')$ no 
 novo sistema de coordenadas. Então, 
 $$  x=x'\cos(\theta)-y'\sin(\theta) \ \ \text{ e } \ \ 
     y=x'\sin(\theta)+y'\cos(\theta).$$
 Usando matrizes podemos escrever a expressão como (talvez mais fácil para decorar):
 $$  
     \begin{pmatrix}
      x \\
      y
     \end{pmatrix}= \begin{pmatrix}
                    \cos(\theta) & -\sin(\theta) \\
                    \sin(\theta) &  \cos(\theta) \\
                    \end{pmatrix}  
                     \begin{pmatrix}
                     x' \\
                     y'
                     \end{pmatrix}.
 $$                       
 Da expressão anterior temos que
  $$  
     \begin{pmatrix}
      x'\\
      y'
     \end{pmatrix}= \begin{pmatrix}
                    \cos(\theta)  &  \sin(\theta) \\
                    -\sin(\theta) &  \cos(\theta) \\
                    \end{pmatrix}  
                     \begin{pmatrix}
                     x \\
                     y
                     \end{pmatrix}.
 $$  
 %  é similar.  
 %Em $\mathbb{R}^{3}$, a rotação de dá atraves de dois ângulos 
 %de Euler. 
 Responda:
   \begin{enumerate}
    \item Usando uma translação transforme a equação 
    $2x^2-3xy+5x+3y-8=0$ em outra equação sem termos lineares.
    {\it Rpta:} Nova origem $O'=(1,3)$ e equação $2x'^2-3x'y'-1=0$
    \item Mediante uma translação transforme a equação 
    $8x^3-x^2+24x-y+1=0$, em outra que não tem termos 
    de segunda ordem nem termo constante.
    {\it Rpta} Nova origem $O'=(-4,8)$ e equação $(x')^{2}y'-1=0$.
    \item Encontre o ângulo de rotação para que a curva
    $8x^2+3\sqrt{3}xy+11y^2=24$ não tenha o termo $xy$.
    {\it Rpta} $\theta=60^{\circ}$.
    \item Mostre que a curva $11x^2+24xy+4y^2=20$ depois de uma rotação 
    $\theta=arctan(3/4)$ se escreve como $4x'^2-y'^2=4$.
    \item Usando primeiramente uma translação com nova origem $O'=(1,1)$ e
    logo uma rotação de $45^{\circ}$, uma equação se transforma 
    em $(x'')^{2}-2(y'')^{2}=2$. Qual é a equação original?
    {\it Rpta:} $x^2-6xy+y^2+4x+4y=0$.
   \end{enumerate}

 \section*{Parábola} 
  
% \begin{tcolorbox}
   Dada uma reta $\mathcal{D}$ e um ponto $F \notin \mathcal{D}$.
   A parábola é definida como 
   $$\mathcal{P}:=\{P \in \mathbb{R}^{2}: dist(P,F)=dist(P,\mathcal{D})\}.$$
   O ponto $F$ é chamado de foco e a reta $\mathcal{D}$ é chamada de reta diretriz.
     \begin{enumerate}
      \item {\bf eixo de simetria: } reta perpendicular à diretriz que passa por F;
      \item {\bf vértice: } interseção no eixo de simetria com a parábola;
      \item {\bf corda: } qualquer segmento de une dois pontos diferentes da parábola;
      \item {\bf corda focal: } corda que passa por F;
      \item {\bf lado reto (ou corda principal): } corda focal paralela à diretriz;
      \item {\bf raio vetor: } segmento de reta que une o foco com algum ponto da parábola. 
     \end{enumerate}
  
 %\end{tcolorbox}
 
 Usando um sistema de coordenadas a parábola $\mathcal{P}$ pode ser escrita com uma das 
 seguintes formas. 
 
 {\bf Forma canônica }(também chamada de {\it forma reduzida}) $y^{2}=4px$ ou $x^2=4py$.
 onde o vertice $V=(0,0)$ e o eixo de simetria é paralelo a algum dos eixos canônicos.
 $$ \text{ Observe que } |p|=dist(V,F) \text{ e } 
                         |p|=dist(V,\mathcal{D}). $$
 
 Quando o vertice $V=(h,k)$ e o eixo de simetria é paralelo a algum dos eixos canônicos temos que $(y-k)^{2}=4p(x-h)$ ou $(x-h)^2=4p(y-k)$.
 %onde o vertice $V=(h,k)$ e o eixo de simetria é paralelo a algum dos eixos  %canônicos.
 
 {\bf Forma geral} $y^{2}+Dy+Ex+F=0$ ou $x^{2}+Dx+Ey+F=0$
 onde o eixo de simetria é paralelo a algum dos eixos canônicos.
 
 %{\bf Forma geral mesmo}. \newline
 
 {\it Retas tangentes: } 
 Em qualquer ponto sobre a parábola podemos 
 calcular retas tangentes e retas normais.
 Para as retas tangentes temos as seguintes formulas qual depende da 
 equação usada da parábola.
 
 {\bf Quando $y^{2}=4px$}. A reta tangente a $\mathcal{P}$ no ponto $P=(x_0,y_0) \in \mathcal{P}$ é dada por 
 $ r: yy_0=2p(x+x_0)$. 
 
 {\bf Quando $x^{2}=4py$}. A reta tangente a $\mathcal{P}$ no ponto $P=(x_0,y_0) \in \mathcal{P}$ é dada por 
 $ r: xx_0=2p(y+y_0)$. \newline

 
Proceda a responder as seguintes questões 
  
  \begin{enumerate}
     \item Escreva as equações das seguintes parábolas.
       \begin{enumerate}
        \item Se $F=(0,2)$ e diretriz $\mathcal{D}: y+2=0$.
        \item Se $F=(0,0)$ e diretriz $\mathcal{D}: y+x=2$.
        \item Se o vértice é $V=(-3,2)$ e foco $F=(-1,2)$.
       \end{enumerate}
     \item Ache a equação da parabóla que tem foco
     $(-5/3,0)$ e cuja reta diretriz é $3x-5=0$.
     {\it Rpta} $3y^2+20x=0$.
     \item Encontre a longitude da corda focal da parábola
     $\mathcal{P}: x^2+8y=0$ que é paralela à reta 
     $r: 3x+4y-7=0$. {\it Rpta: } $25/2$.
     \item Encontre a equação da parábola com foco $F=(2,1)$, 
     com vértice sobre a reta $r: 3x+7y+1=0$ e 
     cuja diretriz é paralela ao eixo x. 
     {\it Rpta: } $\mathcal{P}: (x-2)^2=8(y+1)$.
     \item Se uma parábola tem um vértice sobre a
      reta $r_1:3x-2y=19$, o foco sobre a 
      reta $r_2: x+4y=0$ e 
      diretriz $\mathcal{D}: x=2$.
      {\it Rpta: } $(y+2)^{2}=12(x-5)$.
     \item Encontre a equação de uma parábola cuja lado reto 
     tem como extremo os pontos 
     $A=(7,3)$ e $B=(1,3)$. 
     {\it Rpta: } $\mathcal{P}_1:(x-4)^2=6(y-3/2)$
     e  $\mathcal{P}_2:(x-4)^2=-6(y-9/2)$.
     \item Encontre o valor de $\alpha \neq 0$, para que as 
     coordenadas do foco da parábola 
     $\mathcal{P}: x^2+4x-4\alpha y=8$ somem zero.
     {\it Rpta: } $\alpha=3$ ou $\alpha=-1$.
     \item Encontre a equação da circunferência 
     que passa por o vértice 
     e os extremos do lado reto da parábola 
     $\mathcal{P}: y^2+2y-4x+9=0$. 
     {\it Rpta: } $\mathcal{C}: (x-1/2)^2+(y+1)^2=9/4$.
     \item Encontre a equação da reta tangente e normal 
     da parábola $\mathcal{P}: y^2+2y-4x-7=0$ no ponto de contato 
     $T=(7,5)$ ($T \in \mathcal{P}$). {\it Rpta: }
     reta tangente: $x-3y+8=0$ e reta normal: $3x+y-26=0$.
     \item Encontre as retas tangentes à parabóla 
     $\mathcal{P}: y^2+3x-6y+9=0$ que passa por 
     $P=(1,4)$. {\it Rpta: }
     $3x-2y+5=0$ e $x+2y-9=0$.
     \item Considere a reta $r: x-2y-8=0$. 
     Ache o ponto da parabóla $x^2=4y$ tal que a 
     distância à reta $r$ seja a mínima possível e calcule 
     tal distância.
     {\it Dica: } O ponto deve ser ponto de tangência. 
     {\it Rpta: } $T=(1,1/4)$, distancia$=3\sqrt{5}/2$.
     \item * Considere a parabóla 
     $\mathcal{P}: x^2-2x+8y-23=0$, o ponto de tangência 
     $T=(5,1)$ e um triângulo formado 
     pelo eixo y, a reta tangente e a reta normal em $T$.
     Considere um rectângulo com uns dos lados paralelos
     ao eixo y. Escreva, a àrea do triângulo em função 
     do comprimento $x$ da base e calcule a àrea do rectangulo 
     com a maior àrea possível. 
     {\it Rpta: } A àrea em função de $x$ é $Area_{\Delta}(x)=x(10-2x)$, 
     $x \in (0,5)$.
     O máximo acontece quando $x=5/2$ e $Area_{\Delta}(5/2)=12,5 u^2$.
     \item ** Se o vértice de uma parabóla $\mathcal{P}$ 
     é $V=(-3,1)$, sua reta diretriz é paralela a 
     $r: 3x+4y-6=0$ e uns dos extremos do lado reto é $(6,3)$. 
     Encontre a equação da parabóla. 
     {\it Rpta: } $p=7$, $\mathcal{P}: 16x^2-24xy+9y^2-300x-650y-475=0$.
     \item A entrada duma igreja tem a forma 
     duma parabóla de 9 m. de altura e 12 m. de base. 
     Toda a parte superior é uma janela de vidro cuja 
     base é paralela à base da entrada
     e tem um comprimento de 8 m. 
     Qual a altura máxima da janela?
     {\it Rpta: } altura=4m.
   \end{enumerate}

\end{document}    
\section*{Equação do plano} 

{\bf Equação vetorial.} 
 Um plano $\pi$ em $\mathbb{R}^{n}$, pode ser escrita como 
 $\pi: P=P_{0}+tV+sW$, $t, s \in \mathbb{R}$ 
 onde $V, W$ são linearmente independentes. 
 Note que existe infinitos vetores $V$ e $W$ que geram o mesmo plano. 
 Se conhecemos três pontos sobre a reta, por exemplo $P_{0}$, $P_1$
 e $P_2$, os vetores 
 $\overrightarrow{P_0P_1}, \overrightarrow{P_0P_2} \in \mathbb{R}^{n}$ 
 servem como geradores do plano $\pi$, se
 $\overrightarrow{P_0P_1}, \overrightarrow{P_0P_2}$ são linearmente independentes ({\it por que?}). 
 Assim, qualquer vetor paralelo a $\pi$ pode ser escrito como combinação linear de $V$ e $W$.
 
 {\bf Equação normal do plano em $\mathbb{R}^{3}$}
 Quando o plano está em $\mathbb{R}^{3}$, podemos escrever o plano da 
 forma 
$$ax+by+cz+d=0, \text{ para certos } a,b,c, d \in \mathbb{R}, $$ 
com $a^{2}+b^{2}+c^2\neq0$. Dita forma se chama de 
 {\it equação geral do plano ou equação normal}. Perceba que o vetor 
 $(a,b,c) \in \mathbb{R}^{2}$ é um vetor normal ao plano. 
 
 Se $V$ e $W$ são vetores paralelos ao plano, $V \times W$ serve como 
 vetor normal ao plano.
 
 Veja que se $(x_0, y_{0}, z_0)$ está sobre o plano $\pi$ 
 $\pi: ax+by+cz+d=0$, temos que 
 $$ (a,b,c) \perp ((x,y,z)-(x_0,y_0,z_0)), \text{ para todo } (x,y,z) \in \pi. $$ 

Em $\mathbb{R}^{3}$, considere dois plano $\pi_1$ e $\pi_2$ com $\pi_1 \cap \pi_2 \neq \emptyset$.
  Certamente, $\pi_1 \cap \pi_2$ é uma reta ({\it por que?}). Podemos facilmente encontrar um {\it vetor diretor} calculando 
  $N_1 \times N_2$, onde 
  $N_1$ e $N_2$ são vetores normais ao planos $\pi_1$ e $\pi_2$ respectivamente. 
  
  {\bf Ângulo entre planos. } Em $\mathbb{R}^{3}$, podemos definir o ângulo entre dois planos $\pi_1$ e $\pi_2$ 
 como o ângulo que satisfaz a relação 
  $$\cos (\pi_1,\pi_2)= \frac{|N_1 \circ N_2|}{\|N_1\|\|N_2\|}, $$
  onde  $N_1$ e $N_2$ são vetores normais ao planos $\pi_1$ e $\pi_2$ respectivamente. Observe que na formula
   usamos o {\it valor absoluto} de  $N_1 \circ N_2$ em lugar de 
   $N_1 \circ N_2$.
   
 {\bf Distância de um ponto a um plano em $\mathbb{R}^{3}$}.
 Considere um plano $\pi: ax+by+cz+d=0$ e um ponto 
 $P=(x_{1}, y_{1}, z_1) \in \mathbb{R}^{3}$. 
 A distância de ponto $P$ ao plano $\pi$ é dado pela formula
 
 \begin{equation}
 \text{dist}(P,\pi)= \|\text{proj}_{N}\overrightarrow{P_0P}\|, 
 \label{eqn:distanciapp2}
 \end{equation}
 onde $P_0$ é um ponto em $\pi$ e $N$ é um vetor normal ao plano.
 Note que podemos usar $N=(a,b,c)$.
 
 
\begin{enumerate}
  \item Faça um esboço dos seguintes planos em $\mathbb{R}^{3}$.
     \begin{enumerate}
     \item $2x+3y+5z-1=0$
     \item $3y+2z-1=0$
     \item $2x+3z-1=0$
     \item $3x+2y-4=0$
     \end{enumerate}  
   \item Encontre a equação geral do plano paralelo 
   a $2x-y+5z-3=0$ e passa no ponto $P=(1,-2,1)$.
   {\it Rpta} $\pi: 2x-y+5z-9=0$.
   \item Ache a equação do plano que contem 
   $P=(2,1,5)$ e é perpendicular aos planos 
    $x+2y-3z+2=0$ e $2x-y+4z-1=0$.
    {\it Rpta} $x-2y-z+5=0$. {\it Dica} Use o produto vetorial.
    \item Considere as retas 
    $$ r: \frac{x-2}{2}=\frac{y}{2}=z \text{ e }
       s:  x-2=y=z.$$
     Obtenha a equação geral do plano determinado por 
     $r$ e $s$.  
     \item Sejam dois planos 
      $\pi_1: x+2y+z+2=0$ e $x-y+z-1=0$.
      Encontre o plano que contém 
      a interseção $\pi_1 \cap \pi_2$ e 
      é ortogonal ao vetor $U=(1, 1, 1)$.
      {\it Rpta: } $x-2y+z-2=0$.
      \item Encontre a equação do plano 
      que passa por $A=(1,0,-2)$ e contém 
      $\pi_1 \cap \pi_2$, onde 
      $\pi_1: x+y-z=0$ e $\pi_2: 2x-y+3z-1=0$.
      \item Qual é a equação paramétrica da reta 
      que é interseção dos planos, 
      $\pi_1: (x,y,z)=(1+\alpha, -2, -\alpha-\beta)$ e 
      $\pi_2: (x,y,z)=(1+\alpha-\beta, 2\alpha+\beta, 3-\beta)$?
      \item Sejam três vetores 
      $V=i+3j+2k$, $W=2i-j+k$ e $U=i-2j$ em $\mathbb{R}^{3}$. 
      Se $\pi$ é um plano paralelo aos vetores 
      $W$ e $U$, e $r$ é uma reta perpendicular ao plano $\pi$.
      Encontre a projeção ortogonal de $V$ 
      sobre o vetor diretor da reta $r$.  
      \item Em $\mathbb{R}^{3}$, considere os pontos 
      $A=(2,-2,4)$ e $B=(8,6,2)$.
      Encontre o lugar geométrico dos pontos equidistantes de $A$ e $B$. {\it Dica :} É um plano.
      
      \item Seja $\pi$ um plano que forma 
      um ângulo de $60^{\circ}$ com o plano 
      $\pi_1: x+z=0$ e contém a reta
      $r: x-2y+2z=0, 3x-5y+7z=0$.
      Encontre a equação do plano $\pi$.
      {\it Rpta} Dois soluções: 
      $y+z=0$ ou $4x-11y+5z=0$.      
      %$x-2y+2z=0$ ou $x+4y+8z=0$. 
      \item  O plano $\pi: x+y-z-2=0$ intercepta os 
      eixos cartesianos aos pontos $A$, $B$ e $C$. 
      Qual é a área do triângulo $ABC$? {\it Rpta} $2\sqrt{3} u^2$.
      \item Considere os planos: 
      $$ \pi_1: x-y+z+1=0, \ \ \pi_2: x+y-z-1=0, \ \ 
         \pi_3: x+y+2z-2=0. $$
      Encontre a equação geral que contém $\pi_1\cap \pi_2$ e perpendicular 
      $\pi_3$.   
      \item Ache o ângulo entre o plano $-2x+y-z=0$ e plano que passa por
      $P=(1,2,3)$ e é perpendicular $a i-2j+k$. {\it Rpta:} $arccos(5/6)$.
      \item Para quais valores de $\alpha$ e $\beta$, a reta 
      $r: (\beta, 2, 0)+t(2, \alpha, \alpha)$ está contida no plano 
      $\pi: x-3y+z=1$. {\it Rpta: } $\alpha=1$, $\beta=7$.
      \item Encontre o valor de $\alpha$ para que 
      os planos $\pi_1: (1,1,0)+t(\alpha, 1, 1)+s(1,1, \alpha)$
      e $\pi_2: 2x+3y+2z+1=0$ sejam paralelos. {\it Rpta: }
      $\alpha=1/2$.
      \item Encontre a equação geral do plano 
      $\pi$ que contém a reta $r: (1,0,1)+t(1,1,-1)$ e dista 
      $\sqrt{2}$
      do ponto $P=(1,1,-1)$. 
\end{enumerate}

\end{document}

