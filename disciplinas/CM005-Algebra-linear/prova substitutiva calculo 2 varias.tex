%% Prova substitutiva de Calculo 2
\documentclass[11pt]{exam}
\usepackage[utf8]{inputenc}
\usepackage[T1]{fontenc}
\usepackage[brazilian]{babel}
\usepackage[left=2cm,right=2cm,top=1cm,bottom=2cm]{geometry}
\usepackage{amsmath,amsfonts}
\usepackage{multicol}
%\usepackage{../../../disciplinas}
\usepackage{tikz}
\everymath{\displaystyle}
\def\answers % uncomment to show the answers

\boxedpoints
\pointname{}
%\qformat{{\bf Questão \thequestion} \dotfill \fbox{\totalpoints} }
\qformat{{\bf Questão \thequestion} \dotfill }

\begin{document}

\ifdefined\answers
\printanswers
\fi

\addpoints

\begin{center}
  {\bf \large Cálculo 2 : Prova substitutiva } \\
 Junho de 2019
\end{center}

%\ifx\undefined\answers
%\settabletotalpoints{100}
%\cellwidth{0pt}
%\hqword{Q:}
%\hpword{P:}
%\hsword{N:}

\makebox[\textwidth]{
  Nome: \enspace\hrulefill\quad}
  
%  \gradetable[h][questions]}
%\fi

\begin{center}
  \begin{tabular}{|l|}
    \hline
{\it Responda:} Qual prova vai ser substituída? 
\begin{oneparcheckboxes}
\choice P1
\choice P2
\choice P3
%\CorrectChoice Socrates
\end{oneparcheckboxes}
     \end{tabular}
  \end{center}

  \begin{center}
  \begin{tabular}{|l|}
    \hline
    {\bf Questões relativas à Prova 1} \\
    \hline
     \end{tabular}
  \end{center}
  
   \begin{questions} 
   \question (20) Responda
     \begin{parts}
     \part [5]
     Ache a equação da reta obtida pela interseção de
     $P_{1}: 2x+5z=-3$ e $P_{2}: x+z+2=3y$; 
     \part [5]
     Considere a equação  $x^{2}+y^{2}+z^{2}-4x-2y-2z+2=0$. 
     Coloque a equação na forma padrão, classifique e faça um esboço;
     \part [10] Em $\mathbb{R}^{3}$, identifique a superfície cuja equação é dada por
     $r=2\sin \theta$, $r \geq 0$.
     \end{parts}
   \question (25)
   Considere os planos 
   $\mathcal{P}_1: x+y-z=1$, $\mathcal{P}_2: x+y-2z=0$, e os pontos $P_1=(0,0,1)$
   e $P_2=(1,1,0)$. Denote por $Q_{1}$ a projeção de $P_{1}$ sobre $\mathcal{P}_{1}$ e $Q_{2}$ a projeção de $P_{2}$ sobre 
   $\mathcal{P}_{2}$. 
   Descreva analiticamente o segmento de reta que une
   $Q_{1}$ e $Q_{2}$. 
         
  \question (20) Considere $K$ o sólido limitado por 
     as superfícies $S_{1}: z=8-x^{2}-y^{2}$ 
     e 
      $S_{2}: z=x^{2}+(y-1)^{2}$. 
     Escreva o sólido em coordenadas cilíndricas. 
  \question (20) Parametrizar a interseção das superfícies 
  $S_{1}: x^{2}+y^{2}+z^{2}=10$ 
     e $S_{2}: z+y=4$. 
   \question (25)
  Seja $K$ o sólido limitado por a esfera 
  $x^{2}+y^{2}+z^{2}=z$, que está acima do cone 
  $z=4\sqrt{x^{2}+y^{2}}$ e por baixo de $z=\sqrt{x^{2}+y^{2}}$. Escreva o sólido em coordenadas esfericas.
 \end{questions}
 
  \begin{center}
  \begin{tabular}{|l|}
    \hline
    {\bf Questões relativas à Prova 2} \\
    \hline
     \end{tabular}
  \end{center}
  
  \begin{questions} 
   \question (20) Encontre o plano tangente à superfície 
  $z=2x^{2}-3xy+y^{2}$ paralelo ao plano $\mathcal{P}: 10x-7y-2z+5=0$.
  Para isso:
     \begin{parts}
     \part [10] Encontre o vetor normal ao plano, e um ponto do plano tangente requerido
     \part [10] Use a informação anterior para encontrar o plano tangente.   
     \end{parts}
   \question (30) Calcule, se existe, os seguintes limites 
       \begin{parts}
       \part [10] $\lim_{(x,y)\rightarrow (0,0)} \frac{tan(x)^{2}+tan(y)^{2}}{\cos(xy)}$. 
       \part [10] $\lim_{(x,y)\rightarrow (0,0)} \frac{x^{2}}{x^{2}+y^{2}}$.  
       \part [10] $\lim_{(x,y)\rightarrow (0,0)} \frac{xy}{\sqrt{x^{2}+y^{2}}} \cos(\frac{x^{2}y^{2}}{x^{2}+y^{2}})$.  
       \end{parts}             
  
   \question (25) Considere 
        $$
    f(x,y)= \left\{  
            \begin{array}{lll}
    &\frac{3x^{2}y}{x^{2}+y^{2}} &\text{, se } (x, y) \neq (0,0) \\
    & 0 &\text{, se } (x ,y)=(0,0)  \\
            \end{array}
            \right. 
    $$
  Mostre que $f$ é contínua em $(0,0)$. 
  Calcule $\partial f(0,0)/\partial x$. 
      
   \question (20)
  Seja $z=f(x,y)$ onde $f(x,y)=e^{x^{2}-y^{2}}(\cos 2xy+\sin 2xy)$. Mostre que 
  $$  
  \frac{\partial^{2} z }{\partial x^2}+
  \frac{\partial^{2} z }{\partial y^2}=0. 
  $$
   \question (15) Em um instante, o comprimento de um cateto é um triângulo retângulo é 10 cm e cresce à razão de 1 cm/min, e o comprimento do outro cateto é 12 cm e descresce à razão de 2 cm/min. Encontre a razão de variação da medida do ângulo agudo oposto ao cateto de 12 cm de comprimento no dado instante.  
 \end{questions}
   

   \begin{center}
  \begin{tabular}{|l|}
    \hline
    {\bf Questões relativas à Prova 3} \\
    \hline
     \end{tabular}
  \end{center}
\begin{questions} 
   \question (20) Considere a equação 
   $e^{2x+y}+\sin(y^{2}+x)=1$. Dê algum motivo teorico para afimar que próximo no ponto $(0,0)$, a variável $y$ pode ser escrito como função de $x$. Calcule $dy/dx$ no ponto $(0,0)$. 
    
   \question (20) Encontre o ângulo que formam as superfícies 
   $S_{1}: x^{2}/16+y^{2}/25+z^{2}/9=20$ 
   e $S_{2}: 2x-z+y=50$, no ponto $(8,25,-9)$.
    \question (25) Seja $V(x,y)=xy\exp(y^2x)$ e considere os pontos
     $P=(2,0)$ e $Q=(1/2, 2)$ . 
     \begin{parts}
      \part [10] Determine  a taxa de variação do $f$
       ponto $P$ na direção de $P$ a $Q$.
      \part [10] Encontre a direção em que $V$ tem a mínima taxa de variação em $P_{0}$. Qual é a mínima taxa de variação de 
      $V$ em $P_{0}$?
      \part [5] Na direção do ponto P ao ponto $(3,5)$, a função $V$ aumenta ou diminui?
     \end{parts}

    \question (20) Suponha que um disco circular metálico da forma $x^{2}+y^{2}\leq 4$ está sujeita a um potencial eletrico $V(x,y)=e^{-x^{2}-y^{2}}(2x^{2}+3y^{2})$. Encontre os valores máximos e mínimos do potencial eletrico $V$ sob o disco. 
       \begin{parts}
       \part [10] Descreva corretamente os sistemas não lineares a resolver; 
       \part [10] Resolva adequadamente os sistemas, e escreva os pontos de máximos e mínimos juntos com seus valores.  
       \end{parts}      

  \question (25) Encontre os valores máximos, mínimos locais e os
  pontos de sela de $f(x,y)=x^{3}-y^{3}+3y-3x-1$. 
   \begin{parts}
     \part [10] Ache todos os pontos críticos de $f(x,y)$; 
     \part [10] Use o teste de segunda derivada para encontrar máximos e mínimos locais
     \part [5] Quais são os valores máximos, mínimos locais e os
  pontos de sela? 
   \end{parts}
 \end{questions}
   
\end{document}     
     
 
