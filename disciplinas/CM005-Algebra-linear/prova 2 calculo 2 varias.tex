%% prova 2 de calculo II várias
\documentclass[11pt]{exam}
\usepackage[utf8]{inputenc}
\usepackage[T1]{fontenc}
\usepackage[brazilian]{babel}
\usepackage[left=2cm,right=2cm,top=1cm,bottom=2cm]{geometry}
\usepackage{amsmath,amsfonts}
\usepackage{multicol}
\usepackage{enumitem} %enumerate with letters
%\usepackage{../../../disciplinas}
\usepackage{tikz}
\everymath{\displaystyle}
%\def\answers % uncomment to show the answers

\boxedpoints
\pointname{}
\qformat{{\bf Questão \thequestion} \dotfill \fbox{\totalpoints} }

\begin{document}

\ifdefined\answers
\printanswers
\fi

\addpoints

\begin{center}
  {\bf \large CM302:  Cálculo em Várias Variáveis Reias} ( Prova 2 ) \\
  {\bf Prof.} Alberto Ramos \\
 Maio de 2019
\end{center}

\ifx\undefined\answers
%\settabletotalpoints{100} %% uncomment to set the total points in 100
\cellwidth{0pt}
\hqword{Q:}
\hpword{P:}
\hsword{N:}

\makebox[\textwidth]{
  Nome: \enspace\hrulefill\quad
  \gradetable[h][questions]}
\fi

\begin{center}
  \begin{tabular}{|l|}
    \hline
    {\bf Orientações gerais}\\
    1) As soluções devem conter o desenvolvimento e ou justificativa. 
    \hspace{2.5mm} \\
    2) A interpretação das questões é parte importante do processo de avaliação.\\
    \hspace{2.5mm} Organização e capricho também serão avaliados. \\
    3) Não é permitido a consulta nem a comunicação entre alunos.\\
   \hline 
  \end{tabular} 
\end{center}
  
 % $$\textbf{Proibido usar o L'hospital para o cálculo de limites. }$$
  
 \begin{questions} 
     \question Encontre o plano tangente à superfície 
  $z=\frac{1}{2}x^{2}-3xy+y^{2}$ paralelo ao plano $\mathcal{P}: 5=2y+2z-10x$.
  Para isso:
     \begin{parts}
     \part [20] Encontre o vetor normal ao plano, e um ponto do plano tangente requerido
     \part [10] Use a informação anterior para encontrar o plano tangente.   
     \end{parts}
    \question Calcule, se existe, os seguintes limites 
       \begin{parts}
       \part [10] $\lim_{(x,y)\rightarrow (\sqrt{3},0)} \frac{x^{2}+y^{2}}{\sqrt{x^{2}+y^{2}+1}-1}$. 
       \part [10] $\lim_{(x,y)\rightarrow (1,1)} \frac{x-1}{x^{2}+y^{2}-2}$.  
       \part [10] $\lim_{(x,y)\rightarrow (0,0)} \frac{xy}{\sqrt{x^{2}+y^{2}}} \sin(\frac{x-y}{x^{2}+y^{2}})$.  
       \end{parts}   
   \question 
  Seja $f:\mathbb{R}\rightarrow \mathbb{R}$ função duas vezes derivável
  em $\mathbb{R}$. Se $z=xf(x+y)+yg(x+y)$, mostre que 
  $$  \frac{\partial^{2} z }{\partial^{2}x}+
  \frac{\partial^{2} z }{\partial^{2}y}=
  2\frac{\partial^{2} z }{\partial x \partial y}. $$
     \begin{parts}
     \part[10] 
     Para isso calcule $\frac{\partial^{2} z }{\partial^{2}x}$ e
     $\frac{\partial^{2} z }{\partial^{2}y}$;
     \part[10] 
     Calcule $\frac{\partial^{2} z }{\partial x \partial y}$ e compare.
     \end{parts}
    \question[20] Considere a curva dada por a interseção das 
     superfícies $S_{1}: x^{2}+y^{2}+z^{2}=6$ 
     e 
      $S_{2}: z=x^{2}+y^{2}$.
      Calcule a reta tangente a essa curva no ponto $(\sqrt{2},0,2)$. 
  
   \question[20] Suponha que a água está fluindo numa piscina de plástico na forma de um cilindro circular reto, à razão de $(4/5)\pi m^{3}/min$. A piscina se enche de tal forma que, embora mantendo a forma cilíndrica, seu raio cresce à razão de $0.002 m/min$. Calcule a velocidade que estará subindo o nível da água quando o raio for $2m$ e o volume de água for $20 \pi m^{3}$.
 \end{questions}
\end{document}     
     
  