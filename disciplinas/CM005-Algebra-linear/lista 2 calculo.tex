% lista 2(calculo 1)
\documentclass[latin,20pt]{article}
%\usepackage{amssymb,latexsym,amsthm,amsmath}
\usepackage[paper=a4paper,hmargin={1cm,1cm},vmargin={1.5cm,1.5cm}]{geometry}
\usepackage{amsmath,amsfonts,amssymb}
\usepackage[utf8]{inputenc}

%\usepackage{stmaryrd} %%para graficar maximo inteiro 
\begin{document}

\title{Lista 2: Cálculo I }
 
\author{
A. Ramos \thanks{Department of Mathematics,
    Federal University of Paraná, PR, Brazil.
    Email: {\tt albertoramos@ufpr.br}.}
}

\date{\today}
 
\maketitle

\begin{abstract}
{\bf Lista em constante atualização}.
 \begin{enumerate}
 \item Límites e continuidade 
 \end{enumerate}
\end{abstract}

%%%%%%%%%%%%%%%%%%%%%%%%%%%%%%%%%%%%%%%%%%%%%%  
%\section*{Elipse} 
%Seja $\mathcal{O}$ um aberto em $\mathbb{R}^{n}$. 
%Denote por 
%$C^{1,1}_{L}(\mathcal{O})$ o conjunto das funções deriváveis 
%em $\mathcal{O}$ cuja derivada é Lipschitziana com constante de 
%     Lipschitz $L$ em $\mathcal{O}$, isto é, 
%     $\|\nabla f(x)-\nabla f(y)\|\leq L\|x-y\|$, 
%     para todo $x,y \in \mathcal{O}$.
 
  \section{Exercícios}   
 
 Faça do livro texto, os exercícios correspondentes aos temas desenvolvidos em aula. 
  
  \section{Exercícios adicionais}  
    \subsection{Regras de cálculos para limites}   
   Calcule os seguintes limites.  
    \begin{enumerate}
    \item $\lim_{u \rightarrow 1} 
    \frac{\sqrt{3+u^2}-2}{1-u}=-\frac{1}{2}$.
    \item $\lim_{t \rightarrow 4} 
    \frac{3-\sqrt{5+t}}{1-\sqrt{5-t}}=-\frac{1}{3}$.
    \item $\lim_{x \rightarrow 1} 
    \frac{\sqrt{3x-2}+\sqrt{x}-\sqrt{5x-1}}
    {\sqrt{x}-\sqrt{2x-1}}=-\frac{3}{2}$.
    \item Se $f(x)=\sqrt{1+3x}$. Calcule 
    $\lim_{h \rightarrow 0} 
    \frac{f(x+h)-f(x)}
    {h}=\frac{3}{2\sqrt{3x+1}}$.
    \item $\lim_{x \rightarrow 1} 
    \frac{x^{100}-2x+1}
    {x^{50}-2x+1}=\frac{49}{24}$.
    \item $\lim_{x \rightarrow 2} 
    \frac{\sqrt{1+\sqrt{2+x}}-\sqrt{3}}
    {x-2}=\frac{1}{8\sqrt{3}}$.
     \item Se 
    $\lim_{x \rightarrow 0} 
    \frac{f(x)-1}
    {x}=1$. Prove $\lim_{x \rightarrow 0} 
    \frac{f(ax)-f(bx)}
    {x}=a-b$. {\it Dica:} Considere se $a$ e $b$ são iguais a zero ou não.
    \item Dado $a \in \mathbb{R}$. Mostre que $\lim_{x \rightarrow a} 
    \frac{x\sqrt{x}-a\sqrt{a}}
    {\sqrt{x}-\sqrt{a}}=3a$.
    \end{enumerate}
    \subsection{Limites laterais}   
   Calcule, se existe, os seguintes limites.  
    \begin{enumerate}
    \item $$\lim_{x \rightarrow \frac{5}{2}} 
    \sqrt{|x|+\lbrack\!\lbrack 3x \rbrack\!\rbrack}. $$ Sim, e o limite é $\sqrt{19/2}$.
     \item $$\lim_{x \rightarrow \frac{7}{3}} 
    \sqrt{|x|+\lbrack\!\lbrack 3x \rbrack\!\rbrack}. $$ Não.
    \item $$\lim_{x \rightarrow -3} 
    \frac{\lbrack\!\lbrack x-1 \rbrack\!\rbrack -x}
    {
    \sqrt{|x|^2-\lbrack\!\lbrack x \rbrack\!\rbrack}
    }.$$ Não.
     \item $$\lim_{x \rightarrow 1} 
    \frac{\lbrack\!\lbrack x\rbrack\!\rbrack^{2} -x^{2}}
    {
    \lbrack\!\lbrack x\rbrack\!\rbrack^{2} -x
    }.$$ Não.
    \item Considere a função 
     $$
    f(x)= \left\{  
            \begin{array}{lll}
    &\frac{x^3+3x^2-9x-27}{x+3} &\text{, se } x \in (-\infty, -3) \\
    &ax^2-2bx+1     &\text{, se } x \in [-3,3] \\
            & \frac{x^2-22x+57}{x-3}     &\text{, se } x \in (3,\infty) \\
            \end{array}
            \right. 
    $$
    Para quais valores de $a$ e $b$, existe os limites de $f$ em $x=-3$ e $x=3$? {\it Rpta: } $a=-1, b=4/3$.
    \end{enumerate}
    \subsection{Limites Trigonométricos}   
   Calcule os seguintes limites.  
    \begin{enumerate}
    \item $$\lim_{x \rightarrow \pi} \frac{1-\sin(\frac{x}{2})}{x-\pi}=0$$
    \item $$\lim_{x \rightarrow 0} \frac{x-\sin(x)}{x^2}=0. $$
    \item $$\lim_{x \rightarrow 0} \frac{1-\sqrt{\cos(x)}}{x^2}=\frac{1}{4}$$
    \item $$\lim_{x \rightarrow 1} 
    \frac{\cos(\frac{\pi x}{2})}{1-\sqrt{x}}=\pi$$
    \item $$\lim_{x \rightarrow 0} 
    \frac{1-\cos^7(x)}{x^2}=\frac{7}{2}$$
    \item $$\lim_{x \rightarrow 0} 
    \frac{1-\cos(1-\cos(x))}{x^4}=\frac{1}{8}$$
    \item $$\lim_{x \rightarrow \frac{\pi}{4}} 
    \frac{\sin(2x)-\cos(2x)-1}{\sin(x)-\cos(x)}=\sqrt{2}$$
    \item $$\lim_{x \rightarrow 1} 
    \frac{\sin(\pi x)+\cos(\frac{\pi x}{2})}
    {\tan(\frac{\pi x}{4})-1}=-\frac{3 \pi}{4}$$
    \end{enumerate}
    \subsection{Definição de limite}   
   Usando a definição de limite. Prove que 
    \begin{enumerate}
    \item $\lim_{x \rightarrow 2} \frac{x^2+1}{x-1}=5$.
    \item $\lim_{x \rightarrow 3} 
    \frac{1}{x^2+16}=\frac{1}{25}$. 
    \item $\lim_{x \rightarrow \frac{1}{2}} x^{2}\lbrack\!\lbrack x+2 \rbrack\!\rbrack=\frac{1}{2}$. 
    \item  $\lim_{x \rightarrow 4} 
    \frac{x^3-15x-4}{x-3}=0$.
    \item $\lim_{x \rightarrow a}\cos(x)=\cos(a)$, para qualquer $a \in \mathbb{R}$.  
    \end{enumerate}
\end{document}

  
