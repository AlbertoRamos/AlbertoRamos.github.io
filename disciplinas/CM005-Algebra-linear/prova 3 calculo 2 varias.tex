%% prova 3 de calculo II várias
\documentclass[11pt]{exam}
\usepackage[utf8]{inputenc}
\usepackage[T1]{fontenc}
\usepackage[brazilian]{babel}
\usepackage[left=2cm,right=2cm,top=1cm,bottom=2cm]{geometry}
\usepackage{amsmath,amsfonts}
\usepackage{multicol}
\usepackage{enumitem} %enumerate with letters
%\usepackage{../../../disciplinas}
\usepackage{tikz}
\everymath{\displaystyle}
%\def\answers % uncomment to show the answers

\boxedpoints
\pointname{}
\qformat{{\bf Questão \thequestion} \dotfill \fbox{\totalpoints} }

\begin{document}

\ifdefined\answers
\printanswers
\fi

\addpoints

\begin{center}
  {\bf \large CM302:  Cálculo em Várias Variáveis Reias} ( Prova 3 ) \\
  {\bf Prof.} Alberto Ramos \\
 Junho de 2019
\end{center}

\ifx\undefined\answers
\settabletotalpoints{100} %% uncomment to set the total points in 100
\cellwidth{0pt}
\hqword{Q:}
\hpword{P:}
\hsword{N:}

\makebox[\textwidth]{
  Nome: \enspace\hrulefill\quad
  \gradetable[h][questions]}
\fi

\begin{center}
  \begin{tabular}{|l|}
    \hline
    {\bf Orientações gerais}\\
    1) As soluções devem conter o desenvolvimento e ou justificativa. 
    \hspace{2.5mm} \\
    2) A interpretação das questões é parte importante do processo de avaliação.\\
    \hspace{2.5mm} Organização e capricho também serão avaliados. \\
    3) Não é permitido a consulta nem a comunicação entre alunos.\\
   \hline 
  \end{tabular} 
\end{center}
  
 % $$\textbf{Proibido usar o L'hospital para o cálculo de limites. }$$
  
 \begin{questions} 
     
     \question[20] Considere a relação $z+z^{y}=x$ e o ponto $P=(4,1,2)$.
     Dê algum motivo teorico para afirmar, que próximo do $P$, a variável $z$ pode ser escrito como uma função de $x$ e $y$. Calcule $\partial z/\partial x$
     no ponto $P$.
     
     \question[20] 
       \begin{parts}
       \part Encontre a equação dos planos tangentes à superfície 
      $x^{2}+2y^{2}+3z^{2}=21$ que sejam paralelos ao plano 
      $x+4y+6z=0$. 
%       \part
%     Qual é a equação do plano tangente à superfície 
%     $(x-2)^{2}/5+(y+3)^{2}/9+z^{2}=1$, 
%     que é perperdicular a $2x+y-z=0$, e é paralelo ao plano que passa por $(2,-2,3)$ e $(3,3,-2)$
      \end{parts}
      
    \question[30] Suponha que um disco circular metálico da forma $x^{2}+y^{2}\leq 4$ está sujeita a um potencial eletrico $V(x,y)=e^{-x^{2}-y^{2}}(2x^{2}+3y^{2})$. Encontre os valores máximos e mínimos do potencial eletrico $V$ sob o disco. 
        
  \question Encontre os valores máximos, mínimos locais e os
  pontos de sela de $f(x,y)=x^{4}-4xy+y^{4}+1$. 
   \begin{parts}
     \part [8] Ache todos os pontos críticos de $f(x,y)$; 
     \part [12] Use o teste de segunda derivada para encontrar máximos e mínimos locais
     \part [5] Quais são os valores máximos, mínimos locais e os
  pontos de sela? 
   \end{parts}
   
   \question Suponha que $T(x,y)=40-x^{2}-y^{2}$ represente uma distribuição de temperatura no plano $xy$ (admita que $x$, 
   $y$ estão dados em km e a temperatura em Celsius).  
   Um indivíduo está no ponto $(2,3)$ e quer dar um passeio. 
   \begin{parts}
     \part [5] Descreva o lugar geométrico dos pontos que ele deverá percorrer se ele deseja sempre ter a mesma temperatura que no ponto $(2,3)$; 
     \part [10] Qual a direção e sentido que devera tomar se ele deseja caminhar na direção de maior crescimento da temperatura. Encontre a taxa de variação máxima. 
     \part [10] Se ele se movimentar na direção do vetor $i+j$, a temperatura estará aumentando ou diminuindo? 
   \end{parts}
 \end{questions}
\end{document}     
     
  