%% prova de calculo I
\documentclass[11pt]{exam}
\usepackage[utf8]{inputenc}
\usepackage[T1]{fontenc}
\usepackage[brazilian]{babel}
\usepackage[left=2cm,right=2cm,top=1cm,bottom=2cm]{geometry}
\usepackage{amsmath,amsfonts}
\usepackage{multicol}
\usepackage{enumitem} %enumerate with letters
%\usepackage{../../../disciplinas}
\usepackage{tikz}
\everymath{\displaystyle}
\def\answers % uncomment to show the answers

\boxedpoints
\pointname{}
\qformat{{\bf Questão \thequestion} \dotfill \fbox{\totalpoints} }

\begin{document}

\ifdefined\answers
\printanswers
\fi

\addpoints

\begin{center}
  {\bf \large CMA111:  Cálculo 1A } \\
  {\bf Prof.} Alberto Ramos \\
 Abril de 2018
\end{center}

\ifx\undefined\answers
\settabletotalpoints{100} %% uncomment to set the total points in 100
\cellwidth{0pt}
\hqword{Q:}
\hpword{P:}
\hsword{N:}

\makebox[\textwidth]{
  Nome: \enspace\hrulefill\quad
  \gradetable[h][questions]}
\fi

\begin{center}
  \begin{tabular}{|l|}
    \hline
    {\bf Orientações gerais}\\
    1) As soluções devem conter o desenvolvimento e ou justificativa. 
    \hspace{2.5mm} \\
    2) A interpretação das questões é parte importante do processo de avaliação.\\
    \hspace{2.5mm} Organização e capricho também serão avaliados. \\
    3) Não é permitido a consulta nem a comunicação entre alunos.\\
   \hline 
  \end{tabular}
\end{center}
  
 % $$\textbf{Proibido usar o L'hospital para o cálculo de limites. }$$
  
 \begin{questions}
  \question Calcule os seguintes limites (\textbf{Proibido usar o L'hospital para o cálculo de limites. })
   \begin{parts}
   \part [10] $\lim_{x \rightarrow -1} \frac{4x^{5}+x^{2}+6}{x^{4}+2}$
     \begin{solution} Como o denominador é diferente de zero quando $x=-1$ temos que 
     $$\lim_{x \rightarrow -1} \frac{4x^{5}+x^{2}+6}{x^{4}+2}=
       \frac{4(-1)^{5}+(-1)^{2}+6}{(-1)^{4}+2}=
        \frac{-4+1+6}{1+2}=\frac{3}{3}=1. $$
     \end{solution}
   \part [10] $\lim_{x \rightarrow 3} \frac{\sqrt{x+6}-3}{\sqrt{x+1}-2}$
   \begin{solution} Indeterminação $0/0$. Assim, vamos multiplicar adequadamente para eliminar dita determinação. Multiplicando adequadamente temos que  
       $$ 
       \frac{\sqrt{x+6}-3}{\sqrt{x+1}-2}=
       \left( \frac{\sqrt{x+6}-3}{\sqrt{x+1}-2} \right)
       \left( \frac{\sqrt{x+6}+3}{\sqrt{x+6}+3} \right)
       \left( \frac{\sqrt{x+1}+2}{\sqrt{x+1}+2} \right)=
       %\frac{\sqrt{x+6}^{2}-3^{2}}{\sqrt{x+1}^{2}-2^{2}}
       %\frac{\sqrt{x+1}+2}%{\sqrt{x+6}+3}=
       \frac{\sqrt{x+1}+2}{\sqrt{x+6}+3}. 
       $$
       Assim,
       $$
       \lim_{x \rightarrow 3} \frac{\sqrt{x+6}-3}{\sqrt{x+1}-2}=
       \lim_{x \rightarrow 3} \frac{\sqrt{x+1}+2}{\sqrt{x+6}+3}=
       \frac{4}{6}
       .$$ 
     \end{solution}
   \part [10]
   $\lim_{x \rightarrow 0^{+}} (e^{x}+4x)^{\frac{1}{x}}$
     \begin{solution} Temos indeterminação $1^{\frac{1}{\infty}}$. 
     A ideia é usar $\lim_{u \rightarrow 0 }\frac{\ln (1+u)}{u}=1$. Para isso vamos re-escrever $(e^{x}+4x)^{\frac{1}{x}}=e^{\frac{1}{x}\ln(e^{x}+4x)}$. 
     Assim, basta calcular $\lim_{x \rightarrow 0^{+}}\frac{1}{x}\ln(e^{x}+4x)$. Portanto, 
     $$\frac{1}{x}\ln(e^{x}+4x)=\frac{1}{x}\frac{\ln (1+e^{x}+4x-1)}{
     e^{x}+4x-1}(e^{x}+4x-1)=
     \frac{\ln (1+e^{x}+4x-1)}{
     e^{x}+4x-1}\frac{e^{x}-1+4x}{x}.$$ Agora, procederemos a calcular cada limite por separado 
       \begin{enumerate}[label=\Alph*]
       \item  $\lim_{x \rightarrow 0^{+}}\frac{\ln (1+e^{x}+4x-1)}{
     e^{x}+4x-1}=1$. Bastar notar que quando $x \rightarrow 0^{+}$, $u:=e^{x}+4x-1 \rightarrow 0$ e logo fazendo mudança de variável $\lim_{x \rightarrow 0^{+}}\frac{\ln (1+e^{x}+4x-1)}{
     e^{x}+4x-1}=\lim_{u \rightarrow 0} \frac{\ln (1+u)}{u}=1$.
      \item $\lim_{x \rightarrow 0^{+}}\frac{e^{x}-1+4x}{x}=\lim_{x \rightarrow 0^{+}}\frac{e^{x}-1}{x}+4=1+4=5$.
       \end{enumerate}
     Usando as regras de cálculo para limites 
     $$
     \lim_{x \rightarrow 0^{+}}\frac{1}{x}\ln(e^{x}+4x)=
     \lim_{x \rightarrow 0^{+}}\frac{\ln (1+e^{x}+4x-1)}{
     e^{x}+4x-1}.
     \lim_{x \rightarrow 0^{+}}\frac{e^{x}-1+4x}{x}=1.5=5
     $$
     Finalmente, $\lim_{x \rightarrow 0^{+}} (e^{x}+4x)^{\frac{1}{x}}=
     e^{\lim_{x \rightarrow 0^{+}}\frac{1}{x}\ln(e^{x}+4x)}=e^{5}$.
     \end{solution}
   \part [10] $\lim_{x \rightarrow \frac{\pi}{2}} (\frac{\pi}{2}-x)\tan(x)$.
    \footnote{ Lembre que $\sin(\frac{\pi}{2}+y)=\cos(y)$ e 
    $\cos(\frac{\pi}{2}+y)=-\sin(y)$ para todo $y \in \mathbb{R}$.}
     \begin{solution}
     Indeterminação $0 \infty$. 
     Vamos re-escrever a expressão para eliminar a indeterminação. 
     Faça a mudança de variável $-y=\frac{\pi}{2}-x$. Observe que se 
     $x \rightarrow \pi/2$, temos que $y \rightarrow 0$. Calculando,  
     $$(\frac{\pi}{2}-x)\tan(x)=-y\frac{\sin(\pi/2+y)}{\cos(\pi/2+y)}=
     -y\frac{\cos(y)}{-\sin(y)}=\frac{\cos(y)}{\frac{\sin(y)}{y}}.
     $$
     Assim, 
     $$ \lim_{x \rightarrow \frac{\pi}{2}}(\frac{\pi}{2}-x)\tan(x)=
     \lim_{y \rightarrow 0}\frac{\cos(y)}{\left(\frac{\sin(y)}{y}\right)}=
     \frac{\cos(0)}{1}=1.
     $$
     \end{solution}    
  
   \part [10] $\lim_{x \rightarrow \infty} \sqrt{x(x+4)}-x$
     \begin{solution}Indeterminação $\infty-\infty$. Assim, 
     $$
     \sqrt{x(x+4)}-x
     =\frac{\sqrt{x(x+4)}+x}{\sqrt{x(x+4)}+x}\left(\sqrt{x(x+4)}-x\right)=
     \frac{4x}{x\left(\sqrt{1+\frac{4}{x^{2}}}+1\right)}
     =
      \frac{4}{\left(\sqrt{1+\frac{4}{x^{2}}}+1\right)}
     $$
     Portanto, 
     $$\lim_{x \rightarrow \infty} \sqrt{x(x+4)}-x=\lim_{x \rightarrow \infty} 
     \frac{4}{\left(\sqrt{1+\frac{4}{x^{2}}}+1\right)}=\frac{4}{\sqrt{1}+1}=2.$$
     \end{solution}
   \end{parts}
  \question[20] Determine os valores de $a$  e $b$ para que a seguinte função seja contínua
  em $x=8$.            
    $$
    f(x)= \left\{  
            \begin{array}{lll}
    &\frac{1}{8} \lbrack\!\lbrack x-6 \rbrack\!\rbrack  
    &\text{, se } x < 8 \\
    &ab    &\text{, se } x=8\\
    &\frac{2}{b|2x-7|}    &\text{, se } x>8\\
            \end{array}
            \right. 
    $$
      \begin{solution}
      Primeiro, calculemos os limites laterais quando $x$ tende a $8$.
        \begin{enumerate}
        \item {\it Limite a direita}. Observe que  
        $$ 
        \lim_{x \rightarrow 8^{+}} f(x)=
        \lim_{x \rightarrow 8^{+}}\frac{2}{b|2x-7|} =\frac{2}{b|2.8-7|}
       =\frac{2}{9b} $$
        \item {\it Limite a esquerda}. Como 
        $\lbrack\!\lbrack x \rbrack\!\rbrack$ é uma função definida por partes, vamos analisar o que acontece com $f(x)$ quando $x \in (7, 8)$.
        Se $7<x<8$, então $1<x-6<2$ e assim $\lbrack\!\lbrack x-6 \rbrack\!\rbrack=1$. Desta observação, temos que $f(x)=\frac{1}{8}\lbrack\!\lbrack x-6 \rbrack\!\rbrack=\frac{1}{8}$ para todo $x$ tal que  $7<x<8$. 
        Logo
          $$ 
        \lim_{x \rightarrow 8^{-}} f(x)=
        \lim_{x \rightarrow 8^{-}} \frac{1}{8}\lbrack\!\lbrack x-6 \rbrack\!\rbrack=
        \lim_{x \rightarrow 8^{-}} \frac{1}{8}=\frac{1}{8} $$
        \item {\it O limite $\lim_{x \rightarrow 8} f(x)$ deve existir}.  Assim, o limite a direita e o limite a esquerda são iguais o que implica que 
        $\frac{1}{8}=\frac{2}{9b}$ e portanto 
        $b=\frac{16}{9}$ e $\lim_{x \rightarrow 8} f(x)=\frac{1}{8}$.
        \item {\it  O limite $\lim_{x \rightarrow 8} f(x)$ coincide com $f(8)$ }. Assim, $f(8)=ab=\frac{1}{8}$. Assim, $a=\frac{1}{8b}=\frac{9}{8.16}=\frac{9}{128}$. 
        \end{enumerate}
      \end{solution}
    \question[15] Sejam  
    $f,g:\mathbb{R}\rightarrow \mathbb{R}$ duas funções 
    tais que $|\sin(x)|\leq g(x)\leq 4|x|$ e 
    $0 \leq f(x)\leq 1+|\sin(1/x)|$,  $\forall x \neq 0$. 
    Calcule $\lim_{x \rightarrow 0} f(x)g(x)$.
       \begin{solution}
       Como ambas funções são não-negativas, quando multiplicamos 
       $f(x)$ com $g(x)$, a ordem das desigualdades se mantem. Portanto, 
       $$0 \leq f(x)g(x) \leq (1+|\sin(\frac{1}{x})|)(4|x|)\leq 2. 4|x|=8 |x|, $$
       onde na terceira desigualdades temos usado que $|\sin(y)|\leq 1$, 
       para todo $y \in \mathbb{R}$.
       Como $\lim_{x \rightarrow 0} 8|x|=0$, podemos usar o Teorema do Confronto, para afirmar que  $\lim_{x \rightarrow 0} f(x)g(x)=0$. Observe que não sabemos se o limite $\lim_{x \rightarrow 0} f(x)$ existe ou não.       
       \end{solution}
    \question [15] Seja 
     $$
    f(x)= \left\{  
            \begin{array}{lll}
    &x^{2}-1& \text{, se } x \leq 0 \\
    &x+2    &\text{, se } x>0\\
            \end{array}
            \right. 
    $$
    Calcule os limites laterais 
    $\lim_{x \rightarrow -1^{+}} (f \circ f)(x)$
    e $\lim_{x \rightarrow -1^{-}} (f \circ f)(x)$. Existe o limite $\lim_{x \rightarrow -1} (f \circ f)(x)$?
       \begin{solution}
       Observe que $f$ não é contínua em $x=0$. Assim, devemos analisar por partes. Primeiro, calculemos os limites laterais. 
         \begin{enumerate}
         \item {\it Limite a direita: 
         $\lim_{x \rightarrow -1^{+}} (f \circ f)(x)$}. Considere 
         $x \in (-1, 0)$, assim pela regra de correspondência 
         $f(x)=x^{2}-1$, mas como $x^{2}<1$ para todo $x \in (-1, 0)$
         temos que 
         $f(x)=x^{2}-1$ é negativo ($f(x)<0$) e como consequência 
         $(f \circ f)(x)=f(f(x))=f^{2}(x)-1=(x^{2}-1)^{2}-1$.
         Calculando limite 
         $$ \lim_{x \rightarrow -1^{+}} (f \circ f)(x)=
         \lim_{x \rightarrow -1^{+}} (x^{2}-1)^{2}-1=((-1)^{2}-1)^{2}-1=-1. 
         $$
         \item {\it Limite a esquerda: 
         $\lim_{x \rightarrow -1^{-}} (f \circ f)(x)$}. Considere 
         $x \in (-2, -1)$. Como $x<0$ temos que   
         $f(x)=x^{2}-1$. Quando $x$ pertence a $(-2,-1)$ 
         temos que $f(x)=x^{2}-1$
         é positivo $(f(x)>0)$ e como consequência da regra 
         de correspondência 
          $(f \circ f)(x)=f(f(x))=f(x)+2=(x^{2}-1)+2=x^{2}+1$.
          Calculando limite 
         $$ \lim_{x \rightarrow -1^{-}} (f \circ f)(x)=
         \lim_{x \rightarrow -1^{-}} x^{2}+1=(-1)^{2}+1=2. 
         $$
         \end{enumerate}
       O limite não existe pois ambos limites laterais são diferentes.   
       \end{solution}
    \question [10] Seja $f:[0,1] \rightarrow \mathbb{R}$ uma função contínua tal que $f(x) \in [0,1]$ para todo $x$.
      Então, mostre que existe um $c \in [0,1]$ tal que $f(c)=c$.
        \begin{solution}
        Defina $g(x):=f(x)-x$. Como $g$ é a diferença de duas funções contínuas, temos que g é contínua. Como $f(x)\in [0,1]$. Temos que 
        $g(0)=f(0)\geq 0$ e que $g(1)=f(1)-1\leq 0$. Logo, do Teorema do Valor Intermediario, temos que existe um $c \in [0,1]$ tal que $g(c)=0$. Isto é,
        $ g(c)=f(c)-c=0$ ou $f(c)=c$.
    \end{solution}
 \end{questions}
\end{document}