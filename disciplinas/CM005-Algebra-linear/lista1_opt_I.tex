% lista 1(optimization II 2017 II)
\documentclass[a4paper,latin]{article}
%\usepackage{amssymb,latexsym,amsthm,amsmath}
\usepackage[paper=a4paper,hmargin={1cm,1cm},vmargin={1.5cm,1.5cm}]{geometry}
\usepackage{amsmath,amsfonts,amssymb}
\usepackage[utf8]{inputenc}

\begin{document}

\title{Lista 1: Otimização I }

\author{
A. Ramos \thanks{Department of Mathematics,
    Federal University of Paraná, PR, Brazil.
    Email: {\tt albertoramos@ufpr.br}.}
}

\date{\today}
 
\maketitle

\begin{abstract}
{\bf Lista em constante atualização}.
 \begin{enumerate}
 \item Funções com valores extendidos, semicontinuidade inferior 
 \item Existência de soluções, coercividade
 \item Convexidade 
 \end{enumerate}
\end{abstract}

%%%%%%%%%%%%%%%%%%%%%%%%%%%%%%%%%%%%%%%%%%%%%%  
%\section*{Elipse} 
    \begin{enumerate}
    \item Considere a função 
    $f: \mathbb{R}^{n}\rightarrow \mathbb{R}$ definida como 
    $f(x)=x^{T}Ax+b^{T}x+c$, com  $A$ simetrico e $b\neq 0$.
      \begin{itemize}
      \item  
      Mostre que $f$ admite solução se, e 
      somente se $A$ é definida positiva. 
      \end{itemize}
    \item Mostre a existencia de autovalores para matrices simetricas.
    Para isso, faça o seguinte:
      \begin{enumerate}
      \item 
      Seja $f: 
      \mathbb{R}^{n}
      \setminus\{0\}   
      \rightarrow \mathbb{R}$ 
      uma função contínua tal 
      que $f(\alpha x)=f(x)$, para 
      todo $\alpha>0$ e $x \neq 0$.
      
      Mostre que $f$ admite solução.
      \item Seja $A$ uma matriz simetrica e defina $f(x):= \frac{x^{T}Ax}{\|x\|^{2}}, \text{ para } x \neq 0$.
      Prove que $f$ tem solução
      \item Calcule $\nabla f(x)$, 
      mostre que as soluções 
      são os autovetores de $A$.
      Qual é a relação dos 
      autovalores com $\inf f(x)$? 
      \end{enumerate} 
    \item Denote por 
    $\text{Sym}^{m}(\mathbb{R})$
    o conjunto das matrizes reais simetricas de ordem $m \times m$, por $\text{Sym}^{m}_{+}(\mathbb{R})$ o conjunto das reais matrizes simetricas de 
    $m\times m$ semidefinida positiva e por 
    $\text{Sym}_{++}(\mathbb{R})$ o conjunto de matrizes reais simetricas de 
    $m \times m$ definidas positivas.
    
    \begin{itemize}
    \item Prove que 
    $\text{Sym}^{m}(\mathbb{R})$
    é um espaço de Hilbert, 
    considerando o seguinte 
    produto interno 
    $\langle X, Y \rangle:=\text{tr}(XY)$, 
    onde $\text{tr}(A)$ é 
    o traço da matriz $A$. 
    \item Mostre que $\text{Sym}^{m}_{+}(\mathbb{R})$ is a cone fechado convexo, 
    cujo interior é $\text{Sym}^{m}_{++}(\mathbb{R})$.
    \end{itemize}
    Seja $C \in \text{Sym}_{++}^{m}(\mathbb{R})$, $\eta>0$, $A_{i} \in \text{Sym}^{m}_{+}(\mathbb{R})$, e
     $b_{i} \in \mathbb{R}$, para todo $i=1,\dots, n$.
    Mostre que o problema de minimizar 
    $$\text{minimizar } \langle X, C\rangle+\eta \beta(X) \text{ sujeito a } 
    X \in \text{Sym}_{++}^{m}(\mathbb{R}), \text{ e } 
    \langle X, A_{i} \rangle=b_i, \forall i $$
    admite solução, onde 
    $$ \beta(X)=-\ln \det(X) \text{ para  } X \in \text{Sym}_{++}^{m}(\mathbb{R}) \  \ (\ \ 
    \beta(X) \text{ é chamado de {\it  barreira funcional}} \ \ )$$ 
    Para isso
%     \begin{enumerate}
%     \item, 
     prove que os conjunto de niveis de 
     $\langle X, C\rangle+\eta\beta(X)$ são compactos, e concluia.
     Um roteiro é o seguinte:
       \begin{enumerate}
       \item Seja $c\in \mathbb{R}_{+}$. Mostre que $\beta(t):=ct-\text{ln}(t), t>0$ tem conjuntos de niveis compacto.
       \item Verifique que $\|X\|^{2}=\text{tr}(X^2)=\sum_{i=1}^{m} \lambda^2_{i}(X)$, 
       onde $\lambda_{1}(X)\leq \lambda_{2}(X)\leq \dots \leq \lambda_{m}(X)$ são os autovalores (com multiplicidade) da matriz $X$.
       \item Seja $X \in \text{Sym}^{m}(\mathbb{R})$ e denote por $\lambda(X)$ o vetor em $\mathbb{R}^{m}$
       cujas componentes são $\lambda_{i}(X)$, $i=1, \dots, m$.
       Use a {\it desigualdade de Fan} 
       \footnote{Sempre temos que 
       $\text{tr}(XY)
       \leq \lambda(X)^{T}\lambda(Y)$ 
       para todo 
    $X,Y \in \text{Sym}^{m}(\mathbb{R})$.
       A igualdade vale se, e somente se 
       existe uma matriz ortogonal $U$ 
       tal que
       $UXU^{T}=Diag \lambda(X)$ 
       e
       $UYU^{T}=Diag \lambda(Y)$.
       A desigualdade de Fan pode ser interpretada como um refinamento da  desigualdade de Cauchy-Schwarz.}, 
       para encontrar um limitante inferior para $\langle X, C\rangle=\text{tr}(XC)$ em função dos autovalores de $X$ e $C$.
       \item Use os itens anteriores para mostrar que 
        conjunto de niveis de 
     $\langle X, C\rangle+\eta\beta(X)$ são compactos.
       \end{enumerate}
%     \end{enumerate}              
 
    Esse tipo de problema aparece naturalmente quando estudamos 
    o método de ponto interiores para programação semidefinida positiva (SDP programming).
        
   % DERIVADAS DE LOG DET X. Borwein. 
    \item Seja $f:\mathbb{R}^{n}\rightarrow\mathbb{R}$ uma função derivável 
    tal que $$\lim_{\|x\|\rightarrow \infty} \frac{f(x)}{\|x\|}=\infty.$$
    Mostre que para todo $y \in \mathbb{R}^{n}$, existe um 
    $x \in \mathbb{R}^{n}$ tal que $\nabla f(x)=y$. Em outras palavras,
    $\nabla f: \mathbb{R}^{n}\rightarrow \mathbb{R}^{n}$ é surjetiva. 
    %\item {\it Teorema Fundamental da Álgebra}
    \item Sejam 
    $f,g:(X,\tau)\rightarrow \mathbb{R}\cap\{+\infty\}$ funções com valores extendidos e 
    $x^{*} \in X$. Mostre que
    $$ 
    \liminf_{x\rightarrow x^*}(f(x)+g(x))
    \geq 
    \liminf_{x\rightarrow x^*}f(x)
    +
    \liminf_{x\rightarrow x^*}g(x), 
    \text{ se a soma da direita não é } 
    \infty-\infty.$$     
     \item Seja $(X,\tau)$ um espaço topológico. 
      \begin{enumerate}
      \item  
      Mostre que $\delta_{C}$ é 
       $\tau$-lsc se, e somente se $C$ é $\tau$-fechado.
      \item Mostre que $\text{epi}(f)$
      é fechado na topologia produto $X \times \mathbb{R}$ se, e somente se 
      $f: X \rightarrow \mathbb{R}\cup\{+\infty\}$ é $\tau$-lsc.
      \item Mostre que 
      $\sup \{f_{i}:i \in I\}$
      é $\tau$-lsc se 
      $f_i: X \rightarrow \mathbb{R}\cup\{+\infty\}$ é $\tau$-lsc, 
      $\forall i \in I$.
      
      Em particular, mostre que o supremo de funções $\tau$-contínuas 
      é no máximo $\tau$-lsc, fornecendo um exemplo onde o 
      supremo de funções contínuas não é continua.      
      \item 
    Prove que $\sum_{i=1}^{m} \alpha_{i}f_{i}$
    é $\tau$-lsc se $f_{i}$ é $\tau$-lsc  
    e $\alpha_{i} \in \mathbb{R}_{+}$, $\forall i$ . Isto é, a combinação positiva de funções $\tau$-lsc é $\tau$-lsc.
      \end{enumerate}
    \item Seja $(X,d)$ um espaço métrico e $f:(X,d) \rightarrow 
    \mathbb{R}\cup \{+\infty\}$. Mostre que $f$ é lsc se para todo $(x,\lambda) \in X \times \mathbb{R}$
     temos que $f(x)\leq \lambda$ sempre que 
     $(x^{k},f(x^k))\rightarrow (x,\lambda)$.
    \item {\it Principio Variational de Ekeland}. 
    O seguinte teorema é uma pequena variação de 
    teorema apresentado em aula.
    
    Seja $(X,d)$ um espaço metrico completo,
    $f: X \rightarrow \mathbb{R}\cap \{+\infty\}$ uma 
    função lsc e limitada inferiormente. 
    
    Suponha que existe $\varepsilon>0$ 
    e $z \in X$ tal que 
    $$   f(z) \leq \inf f +  \varepsilon.$$
    Então, para todo $\lambda>0$, {\it existe} um elemento $z_{\lambda} \in X$
    com as seguintes propriedades:
       \begin{enumerate}
       \item 
       $$d(z_{\lambda}, z)\leq \lambda, \  \ \ \  \ \
       f(z_{\lambda})+
       \frac{\varepsilon}{\lambda}d(z_{\lambda}, z)\leq f(z)
       $$
       \item Para todo $y \neq z_{\lambda}$, 
       temos que 
        $$f(z_{\lambda})<
       f(y)+\frac{\varepsilon}{\lambda}d(z_{\lambda},y)$$
       \end{enumerate}
    Para provar o teorema use os seguinte passos.
    A ideia é construir indutivamente uma sequência $\{z^{n}\}$ 
    ( com $z^0=z$ )
    de Cauchy, cujo limite seja o ponto $z_{\lambda}$ desejado. 
    Seja $z^n$ conhecido. Então, defina 
       $$S_{n+1}:=\{x \in X: x \neq z^n \text{ e } f(x)+
       \frac{\varepsilon}{\lambda}d(x, z^n)\leq f(z^n)\}.$$
       Se $S_{n+1}=\emptyset$, faça $z^{n+1}:=z^n$. 
       Caso contrário, escolha 
       $z^{n+1} \in S_{n+1}$ tal que 
              $$
              f(z^{n+1})< \inf_{x \in S_{n+1}}f(x)+
              \frac{1}{2}(f(z^n)-\inf_{x \in S_{n+1}}f(x)).$$
       \begin{enumerate}
       \item Mostre que é possível, encontrar $z^{n+1}$ em ambos casos.        
       \item Mostre que existe 
       $\sum_{n=0}^{\infty} d(z^{n+1}, z^n)$ e como consequência que
       $\{z^n\}$ é uma sequência de Cauchy.
       \item Prove que se $S_{n}=\emptyset$ para algum 
       $n \in \mathbb{R}$, então $S_{m}=\emptyset$ para todo $m \geq n$, e assim $z_{\lambda}:=z^{n}$ satisfaz as condições requeridas.
       \end{enumerate}   
        Suponha, a partir de agora que $S_{n}\neq \emptyset$, para todo
        $n \in \mathbb{N}$ e denote por $z_{\lambda}$ o limite de 
        $\{z^{n}\}$
          \begin{enumerate}
          \item Mostre que $S_{n+1}\subset S_{n}$ para todo 
          $n \in \mathbb{N}$ e que $z_{\lambda} \in 
           \cap_{n \in \mathbb{N}} S_{n}$ 
           ( aqui usamos que $f$ é lsc ).
          \item Verifique a desigualdade 
          $f(z^{n+1})-\inf \{f(x):x \in S_{n+1}\}
          \leq f(z^{n})-f(z^{n+1})$, prove que 
          $f(z^n)-f(z^{n+1}) \rightarrow 0$, e use ditas
          propriedades para provar que $\cap_{n \in \mathbb{N}} S_{n}:=z_{\lambda}$. 
          \item Use que $\cap_{n \in \mathbb{N}} S_{n}:=z_{\lambda}$, para mostrar que se $y \neq z_{\lambda}$ (i.e.
           $y \notin \cap_{n \in \mathbb{N}} S_{n}$ ), 
           temos que 
          $f(z_{\lambda})<
       f(y)+\frac{\varepsilon}{\lambda}d(z_{\lambda},y)$.
       Complete a prova.
          \end{enumerate}              
    \item {\it Existência de pontos aproximadamente estacionários}.
    Seja $f:\mathbb{R}^m \rightarrow \mathbb{R}$ uma função derivável.
    Suponha que existe $x \in \mathbb{R}$
    tal que $f(x)<\inf f +\varepsilon$
    para certo $\varepsilon>0$.
    Mostre que para todo $\lambda>0$, existe um $x^*$ com $\|x^*-x\|\leq \lambda$
    e $\|\nabla f(x^*)\|\leq \varepsilon/\lambda$. O ponto $x^*$ 
    é chamado de ponto aproximadamente estacionário. 
    
    Ainda mais, use o resultado anterior para provar que se $\inf f>-\infty$ e $f$ derivável, existe uma sequência 
    $\{x^n\}$ tal que 
    $f(x^{n})\rightarrow \inf f$
    e 
    $\|\nabla f(x^n)\| \rightarrow 0$.      
    \item {\it Teorema de Ponto Fixo de Caristi}.
    Seja $(X,d)$ um e. m. completo e 
    $\Phi:(X,d)\rightarrow \mathbb{R}\cup \{+\infty\}$
    uma função lsc com limitada inferiormente.
    Se $T: X \rightrightarrows X$ é uma multifunção 
    (i.e. $T(x)$ é um subconjunto de $X$) tal que 
    $$  \Phi(y)\leq \Phi(x)-d(x,y)\ \ \forall x \in X, y \in T(x).$$
    Então, existe um $x^{*} \in X$ tal que $x^* \in T(x^*)$. {\it Dica: Use o principio variacional de Ekeland}.
    \item {\it Teorema do ponto fixo de Banach}.
    Seja $(X,d)$ e. m. completo e $T:X \rightarrow X$
    uma contração, isto é, existe $\eta \in [0,1)$
    tal que $d(T(x),T(y))\leq \eta d(x,y)$ para todo 
    $x, y \in X$. 
    Então, $x^{*} \in X$ tal que $x^*=T(x^*)$.
    {\it Dica:} Use o teorema de ponto fixo de Caristi com 
    $\Phi(x):=(1-\eta)^{-1}d(x,T(x))$.
    \item {\it Condição de Palais-Smale }
    Seja  $f:\mathbb{R}^{n} \rightarrow 
    \mathbb{R}$ de classe $C^{1}$.
    Dizemos que $f$ satisfaz a condição de Palais-Smale  se para toda sequência $\{x^{k}\}$ tal que 
    $$  
        f(x^k) \text{ converge para algum número } \alpha 
        \ \ \text{ e } \ \  
        \nabla f(x^k)\rightarrow 0, $$
    a sequência $\{x^{k}\}$ tem uma subsequência convergente.
    
    Prove que se $f$ é limitada inferiormente e satisfaz a condição de Palais-Smale. Então, $f$ deve ser coerciva.
    \item Seja $X$ um espaço 
    vetorial real. 
      \begin{enumerate}
      \item  
      Mostre que $\delta_{C}$ é 
      convexo se, e 
      somente se $C$ é convexo.
      \item Mostre que $\text{epi}(f)$
      é convexo se, e somente se 
      $f: X \rightarrow \mathbb{R}\cup\{+\infty\}$ é convexo.
      \item Mostre que 
      $\sup \{f_{i}:i \in I\}$
      é convexo se 
      $f_i: X \rightarrow \mathbb{R}\cup\{+\infty\}$, $\forall i \in I$.
      Dê um exemplo de que 
      $\inf \{f_{i}:i \in I\}$
      não é necessáriamente convexo, mesmo se $I$ fosse finito.      
      \item Se 
      $f: X \rightarrow \mathbb{R}\cup\{+\infty\}$ é convexo. Prove que 
      $\text{lev}_{\gamma}(f)$ é convexo 
      para todo $\gamma \in \mathbb{R}$.
      Dê um exemplo onde a implicação reversa não vale. 
      \item Mostre que $f$ é convexa se, e somente se  $f(\sum_{i=1}^{n} \alpha_{i}x_{i}) \leq \sum_{i=1}^{n}\alpha_i f(x_{i})$, para todo $x_{i} \in \text{dom}(f)$, $\alpha_{i}\geq 0$ e $\sum_{i=1}^{n}\alpha_{i}=1$. 
      \end{enumerate}
   \item (slope inequality) Seja $f$ uma função com
   $dom(f)=\mathbb{R}$.
   Então, $f$ é convexa em $[a,b]$
   se, e somente se para todos $x_{0}<x<x_{1}$ em $[a,b]$, temos que 
   $$\frac{f(x)-f(x_{0})}{x-x_{0}}\leq 
     \frac{f(x_1)-f(x_{0})}{x_1-x_{0}}\leq
     \frac{f(x_1)-f(x)}{x_1-x}.
   $$
   Faça um esboço dessas desigualdades.
    
   \item {\it (Testes de convexidade usando derivadas)}.
   Seja $f: \mathbb{R}^{n} \rightarrow \mathbb{R}$ 
   diferenciável. 
   Então, $f$ é convexa se, e somente se 
   algumas das siguentes condições valem:
     \begin{itemize}
     \item $f(y)\geq f(x)+\langle \nabla f(x), y-x \rangle$, 
     $\forall x, y$
     \item $\nabla f (x)$ é mononota. i.e. 
     $\langle \nabla f(x)-\nabla f(y), x-y \rangle \geq 0$,  
      $\forall x, y$
     \item $\nabla^{2} f(x)\geq 0$, para todo $x$.
     (aqui assumimos que $f$ é duas vezes diferenciável)
     \end{itemize}
   Com essas ferramentas, facilmente podemos provar que $-\log(x)$, $\exp(x)$, (a entropia) $S(x):=-x\log(x)$, se $1\geq x\geq0$; $S(0)=0$, 
   $f(x):=x^{-\alpha}$, $\alpha>0$, $x \in (0,\infty)$ são funções convexas.   
    \item Use a convexidade $x\rightarrow |x|^{p}$  para provar a desigualdade das médias generalizada
    (generalized mean inequality), isto é, 
    $$ (\sum_{i=1}^{n} \alpha_{i}x_{i}^{p})^{1/p}
     \leq (\sum_{i=1}^{n} \alpha_{i}x_{i}^{q})^{1/q}$$ para todo $\alpha_{i}\geq 0$, com $\sum_{i=1}^{n}\alpha_{i}=1$, $x_{i}>0$, $\forall i$ e $p\leq q$, com 
     $|p|+|q|\neq0$.
    Como caso particular, prove a desigualdade de Young: Para todo $x,y \geq 0$, $p,q\in(1,\infty)$, tal que $1/p+1/q=1$ temos que 
    %\frac{1}{p}+\frac{1}{q}=1
    $$ xy \leq \frac{1}{p} x^{p}+\frac{1}{q}y^{q}. $$
    \item
    Seja $f$ uma  função convexa com valores reais extendidos. 
        \begin{enumerate}
        \item Mostre que o conjunto das soluções é convexa. 
        \item Mostre que todo minimizador local é, de fato, 
        um minimizador global 
        \end{enumerate} 
    \item Seja $A$ um subconjunto de $(X,d)$. Mostre que 
    $d_{A}(x):=\inf \{d(x,a): a \in A\}$ é 
    uma função Lipschitziana com constante de Lipschiz igual a 1. 
   \item 
   Seja $X$ um espaço finito dimensional com produto interno $\langle \cdot, \cdot \rangle$. 
   Considere $C\neq \emptyset$ um conjunto
   fechado e convexo. 
     \begin{enumerate}
     \item Seja $x \in X$. Mostre que o problema 
     $$\text{min} \{\|x-c\|: c \in C\}$$
     admite uma {\it única} solução, qual é denotada por $\text{proj}_{C}(x)$, a projeção de $x$ sobre o conjunto $C$. 
     O valor otimo é denotado por $d_{C}(x)$.
     Se escolhemos $x=0$, 
     $\text{proj}_{C}(0)$
     representa o elemento de $C$ com norma minima. %minima norma.
     \item A projeção está caracterizada por uma desigualdade variacional. Isto é, mostre que 
     $$ 
     x^{*}:= \text{proj}_{C}(x)
     \text{ se, e somente se } 
     \langle x-x^*, c-x^*\rangle \leq 0,
     \text{ para todo } 
     c \in C
     .$$
     Se $C$ é um subespaço vetorial, a desigualdade anterior se reduz a dizer que $x-x^* \perp C$. 
     \item Em general, pode não existir elemento de norma minima. 
     Seja $X=C[0,1]$ o conjunto das função continua com a norma do supremo. 
     Seja $C$ o conjunto das funções 
     $f \in X$
     tal que 
     $$ \int_{0}^{1/2} f(t)dt-\int_{1/2}^{1} f(t)dt=1.$$
     Mostre que $C$ é um conjunto não vazio, fechado e convexo, mas que não admite elemento de norma minima.
     (o problema aqui é que $X$ não é reflexivo)
     \end{enumerate}   
   \end{enumerate}    
\end{document}   
%%%%%%%%%%%%%%%%%%%%%%%%%%%%%%%%%%%%%%%%%%%%%%  
