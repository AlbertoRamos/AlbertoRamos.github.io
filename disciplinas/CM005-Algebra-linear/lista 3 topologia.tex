% lista 3(topology)
\documentclass[latin,20pt]{article}
%\usepackage{amssymb,latexsym,amsthm,amsmath}
\usepackage[paper=a4paper,hmargin={1cm,1cm},vmargin={1.5cm,1.5cm}]{geometry}
\usepackage{amsmath,amsfonts,amssymb}
\usepackage[utf8]{inputenc}

%\usepackage{stmaryrd} %%para graficar maximo inteiro 
\begin{document}

\title{Lista 3: Introdução à Topologia }
 
\author{
A. Ramos \thanks{Department of Mathematics,
    Federal University of Paraná, PR, Brazil.
    Email: {\tt albertoramos@ufpr.br}.}
}

\date{}
 
\maketitle

\begin{abstract}
{\bf Lista em constante atualização}.
 \begin{enumerate}
 \item Espaços conexos;
 \item Lema de Urysohn; 
 \item Teorema de Tychonoff. Temas diversos. 
 \end{enumerate}
\end{abstract}

%%%%%%%%%%%%%%%%%%%%%%%%%%%%%%%%%%%%%%%%%%%%%%  
%\section*{Elipse} 
%Seja $\mathcal{O}$ um aberto em $\mathbb{R}^{n}$. 
%Denote por 
%$C^{1,1}_{L}(\mathcal{O})$ o conjunto das funções deriváveis 
%em $\mathcal{O}$ cuja derivada é Lipschitziana com constante de 
%     Lipschitz $L$ em $\mathcal{O}$, isto é, 
%     $\|\nabla f(x)-\nabla f(y)\|\leq L\|x-y\|$, 
%     para todo $x,y \in \mathcal{O}$.
  
  \section*{Royden}
  \subsection*{Real analysis: Capítulo 10}
    \begin{enumerate}
    \item {\bf Section 10.2} Teorema 6, teorema 7, 15, 21, 24, 27;
    \item {\bf Section 10.3} 29, 30, 33, 34, 36, 37.
    \end{enumerate}
  \subsection*{Real analysis: Capítulo 11}
    \begin{enumerate}
    \item {\bf Section 11.2} 12, 13, 14, 15;
    \item {\bf Section 11.3} 17, 19;
    \item {\bf Section 11.4} 34, 36;
    \item {\bf Section 11.5} Teorema 18, proposição 19, 41, 42, 45, 46;
    \item {\bf Section 11.6} 50, 51, 53, 54, 55.
    \end{enumerate}
  \subsection*{Real analysis: Capítulo 12}
    \begin{enumerate}
    \item {\bf Section 12.1} Lema de Urysohn, lema 1, lema 2, 2, 
    3, 4, 5, 7, 8;
    \item {\bf Section 12.2} 11, 12, 16, 17, 21, 22, 23.
    \end{enumerate}  
  \section*{Elon Lima}
  \subsection*{Espaços Métricos} 
   \begin{enumerate}
    \item {\bf Capítulo 4} Proposição 1, proposição 5, proposição 6, 
    proposição 11, 2, 6, 9, 11, 15, 17, 21, 22, 24, 28, 37, 48. 
    \item {\bf Capítulo 7} Prove o Teorema de Baire, exemplo 33, 
    proposição 24, proposição 25, 35, 36, 40. 
   \end{enumerate} 
\end{document}
















  












