% lista 2(calculo 2 varias)
\documentclass[11pt]{article}
%\usepackage{amssymb,latexsym,amsthm,amsmath}
\usepackage[paper=a4paper,hmargin={1cm,1cm},vmargin={1.5cm,1.5cm}]{geometry}
\usepackage{amsmath,amsfonts,amssymb}
\usepackage[utf8]{inputenc}

%\usepackage{stmaryrd} %%para graficar maximo inteiro 
\begin{document}

\title{Lista 2: Cálculo em várias variáveis reias }
 
\author{
A. Ramos \thanks{Department of Mathematics,
    Federal University of Paraná, PR, Brazil.
    Email: {\tt albertoramos@ufpr.br}.}
}

\date{\today}
 
\maketitle

  \section{Exercícios}   
 
 Faça do livro texto \footnote{Livro texto: Cálculo. Volume II. {\it J. Stewart}, 5 edição.}, os seguintes exercícios. 
  
    \subsection*{•}
     
     \begin{enumerate}
     \item Capítulo 13.1: 12, 34, 35, 40, 41(c);
     \item Capítulo 13.2: 8, 10, 33, 35, 49; 
     \item Capítulo 13.3: 1;
     \item Capítulo 13.4: Exemplo 4, 10; 
     \item Capítulo 14.1: 6, 9, 19, 24, 39; 
     \item Capítulo 14.2: Exemplo 7, 7, 11, 15, 19, 37;      
     \item Capítulo 14.3: 17, 20, 24, 27, 35, 44, 47, 70, 77, 80, 85, 86;
     \item Capítulo 14.4: Exemplo 1, 1, 3, 5, 17;      
     \item Capítulo 14.5: Exemplo 1, Exemplo 9, 5, 21, 32, 36, 39, 43, 51.           
     \end{enumerate}
 
   \subsection{Exercícios adicionais}
     \begin{enumerate}
      \item Parametrize a curva definida pela interseção $z=4-y^{2}$ e 
      $z=x^{2}+3y^{2}$.
      
     {\it Rpta} $\overrightarrow{\alpha}(t)=(2 \cos t, \sin t, 4-\sin^{2}t)$, 
     $t \in [0,2\pi]$.
       \item Encontre a reta tangente no ponto $(1,0,-2)$ da curva definida pela interseção $x^{2}+3y^{2}+z^{2}=1$ e 
      $x^{2}+3y^{2}=1$.
      
       {\it Rpta} $\overrightarrow{r}(t): (1,0,-2)+t(0,\sqrt{3},0), t \in \mathbb{R}$.
     \item Considere a função 
     $$
    f(x,y)= \left\{  
            \begin{array}{lll}
    &x^{2}+\frac{yx^{3}}{x^{4}+y^2} &\text{, se } (x,y)\neq (0,0) \\
    & 0 &\text{, se } (x,y)=(0,0)  \\
            \end{array}
            \right. 
    $$      
    A função $f$ é contínua em $(0,0)$? {\it Rpta: } Sim.
    Use $2|a||b|\leq a^{2}+b^{2}$.
      \item Encontre os valores de $a$ e $b$ para que 
     o plano $ax+by+2z+2=0$ seja tangente ao paraboloide 
     $z=y^{2}+3x^{2}+1$  no ponto $(1,1,5)$.
     {\it Rpta: } $a=-12$, $b=-4$.   
     \item Encontre a equação do plano tangete à superficie 
     $z=x^{2}+xy$ que seja perpendicular aos planos 
     $x+y-z=3$ e  $2x-y+z=4$.
     {\it Rpta: } $y+z=1$.   
    
    \item Considere $f(t)$ uma função real de classe $C^{2}$ em todo 
       $\mathbb{R}$. Se $g(x,y)=x+y+f(x^{2}+y^{2})$. 
       Verifique que  $\frac{\partial^{2} g}{\partial x^{2}}-
       \frac{\partial^{2} g}{\partial y^{2}}=4(x^{2}-y^{2})
       \frac{d^{2} f(t)}{d t^{2}}$.
       \item Se $z=xy+xe^{y/x}$. Verifique que 
       $x\frac{\partial z}{\partial x}+
       y\frac{\partial z}{\partial y}=xy+z$
     
     \end{enumerate}
\end{document}


  
  \section{Exercícios adicionais}
    \subsection{Funções vetoriais de variável real}
     \begin{enumerate}
     \item Encontre o domínio de  
     $\overrightarrow{\alpha}(t)=(\ln(t+1), \sqrt{t^{2}+2t-8})$. 
     {\it Rpta:} Domínio $[2,\infty)$.
     \item Encontre o domínio de  
     $\overrightarrow{\alpha}(t)=(\frac{1+t}{1-t}, (t+2)^{-3/2}, \frac{1}{t-4})$. 
     {\it Rpta:} Domínio $(-2,\infty)\setminus \{4\}$.
     \item Mostre que a trajétoria de 
     $\overrightarrow{\alpha}(t)=t\hat{i}+t^{2}\hat{j}$, $t \in \mathbb{R}$
     é uma parábola. 
     \item Descreva a trajétoria de 
     $\overrightarrow{\alpha}(t)=(3\cos(t),3\sin(t), t)$, 
     $t \in \mathbb{R}$. {\it Rpta:} Hélice.
     \item Descreva a trajetória de 
     $\overrightarrow{\alpha}(t)=(t\cos(t),t\sin(t), t)$, 
     $t \in \mathbb{R}$. {\it Rpta:} A trajetória está contida num cone circular.  
     \item Descreva a trajetória de 
     $\overrightarrow{\alpha}(t)=(t, t, \sin(t))$, 
     $t \in \mathbb{R}$. 
     \item Uma partícula encontra-se no primeiro quadrante de $\mathbb{R}^{2}$. A partícula move-se de forma que a distância à origem é igual à inclinação $t$ da reta que une a origem e a posição da partícula. 
     Encontre uma parametrização do movimento da partícula usando $t$
     como parâmetro.  
     {\it Rpta:}  $\overrightarrow{\alpha}(t)=
     (\frac{t}{\sqrt{1+t^{2}}},\frac{t^2}{\sqrt{1+t^{2}}} )$, $t \geq 0$.
     \item Defina uma função vetorial em $\mathbb{R}^{3}$ com 
     domínio $[-2,2]$ cuja trajetória é o triângulo 
     de vértices $A=(3,2,-1)$, $B=(2,0,1)$ e $C=(1,-2,1)$. 
     \end{enumerate}
    \subsection{Limites, continuidade, derivadas e integração}
     \begin{enumerate}
     \item Calcule, se existe, os seguintes limites 
       \begin{enumerate}
       \item $\lim_{t \rightarrow 3} \overrightarrow{\alpha}(t)$
       onde $\overrightarrow{\alpha}(t)=(\frac{t^{2}-2t-3}{t-3}, 
       \frac{t^{2}-5t+6}{t-3})$.  {\it Rpta:} (4,1).
       \item $\lim_{t \rightarrow 0} \overrightarrow{\alpha}(t)$
       onde $\overrightarrow{\alpha}(t)=(\frac{1-\sqrt{1+t}}{1-t}, 
       \frac{t}{t+1}, 1)$.  {\it Rpta:} (0,0,1).
       \item $\lim_{t \rightarrow 0} \overrightarrow{\alpha}(t)$
       onde $\overrightarrow{\alpha}(t)=(\frac{\sin 7t}{t}, 
       \frac{\sin 5t}{\sin 3t}, \frac{\tan 3t}{\sin 2t})$.  {\it Rpta:} (7,5/3,3/2).
       \item $\lim_{t \rightarrow 2} \overrightarrow{\alpha}(t)$
       onde $\overrightarrow{\alpha}(t)=(\ln t, 
       \sqrt{1+t^2}, \frac{3t}{4-t^2})$.  {\it Rpta:} Não existe.
       \end{enumerate}
    \item Seja $\mathcal{C}$ uma curva parametrizada por 
       $\overrightarrow{\alpha}(t)=
       (1-2t, t^2, 2e^{2t-2})$. Encontre a equação da reta tangente a $\mathcal{C}$ no ponto onde $\overrightarrow{\alpha}'(t)$
       é paralelo a $\overrightarrow{\alpha}(t)$. 
       {\it Rpta:} $r: (-1,1,2)+t(-1,1,2)$; $t \in \mathbb{R}$. 
    \item Sejam as curvas parametrizadas por 
       $\overrightarrow{\alpha}(t)=
       (e^t, e^{2t}, 1-e^{-t})$
       e   $\overrightarrow{\beta}(t)=
       (1-t, \cos t, \sin t)$. Encontre a interseção das trajétorias das curvas e o ângulo da interseção. 
       {\it Rpta:} Interseção $P=(1,1,0)$; Ângulo $\theta=\pi/2$.  
     \item Suponha que $\|\overrightarrow{\alpha}(t)\|$ é constante para
       todo $t \in \mathbb{R}$. Verifique que 
       $\overrightarrow{\alpha}(t)\cdot\overrightarrow{\alpha}'(t)=0$ 
     \item Encontre os pontos em que a curva     
     $\overrightarrow{\alpha}(t)=
       (t^2-1, t^2+1, 3t)$ corta o plano 
       $\mathcal{P}: 3x-2y-z+7=0$. 
       {\it Rpta: } $P=(3,5,6)$ e $Q=(0,2,3)$. 
       \item Calcule o produto interno de $\overrightarrow{a}$
       e $\overrightarrow{b}$ onde 
       $\overrightarrow{a}=(2,-4,1)$
       e $\overrightarrow{b}=\int_{0}^{1}(te^{t}, t \sinh 2t, 2te^{-2t})dt$.
       {\it Rpta: } 0. 
     \item Considere 
     $\overrightarrow{\alpha}(t)=\overrightarrow{a}\cos(\omega t)+
       \overrightarrow{b}\sin(\omega t)$, com $t \in \mathbb{R}$.
       Verifique que 
        
        $$\text{(a)} \ \ \ \ \ 
        \overrightarrow{\alpha}(t)\times\frac{d \overrightarrow{\alpha}(t)}{dt}=
       \omega \overrightarrow{a} \times \overrightarrow{b}
       \  \ \ \ \ \text{ e } \ \ \ \ \ 
       \text{(b)} \ \ \ \ \
       \frac{d^{2}\overrightarrow{\alpha}(t)}{dt^{2}}+\omega^{2}\overrightarrow{\alpha}(t)=
       \overrightarrow{0}. $$
     \item Em $\mathbb{R}^{3}$ considere $\overrightarrow{\alpha}(t)$ uma curva 
     derivável com derivada contínua, não nula. Mostre que 
        \begin{enumerate}
        \item $\overrightarrow{\alpha}(t)$ tem norma constante se, e somente se $\overrightarrow{\alpha}(t)\cdot \overrightarrow{\alpha}'(t)=0$.
        \item $\overrightarrow{\alpha}(t)$ tem direção constante se, e somente se $\overrightarrow{\alpha}(t)\times \overrightarrow{\alpha}'(t)=0$.
        \end{enumerate}
     \end{enumerate}  
    
      
      \begin{table}[h]
\centering % centering table
\begin{tabular}{c} % creating eight columns
\hline \hline%inserting double-line
Audio  \\ 
\hline % inserts single-line
Police \\ % [1ex] adds vertical space
\hline \hline % inserts single-line
\end{tabular}
     \label{tab:hresult}
     \end{table}


    \subsection{Máximos, mínimos e pontos críticos}
      \begin{enumerate}
      \item Qual é o ponto da curva $yx=2$, $x>0$, que está mais próximo ao origem. 
      \item Considere duas partículas $A$ e $B$ que se movem sobre os eixos $x$ e eixo $y$ respetivamente. Se a posição de $A$ é $(\sqrt{t}, 0)$ e 
      a posição de $B$ é $(0,t^{2}-\frac{1}{4})$, para $t \geq 0$. Encontre o instante onde a distância entre $A$ e $B$ seja o menor possível.
       \item Considere a curva $y=1-x^{2}$, $x \in [0,1]$. 
       Qual a reta tangente à curva tal que a área do triângulo que ela forma com os eixos coordenados seja mínima? 
       \item Seja $L(x)=-x^{3}+12x^{2}+60x-4$ o lucro de uma empresa ao vender certo determinado produto, onde $x$ representa a quantidade do produto produzida. Determine o lucro máximo e a produção que máximiza o lucro.
       {\it Rpta: } $x=10$, Lucro máximo $L(10)$   
      \end{enumerate}
    \subsection{Teorema de Rolle e Teorema de Valor Médio}
      \begin{enumerate}
      \item Mostre que a equação $f(x)=x^{7}+5x^{3}+x-7$ tem uma única solução.      
      \item Considere a função $f(x)=e^{x}-\frac{1}{x}-\frac{x}{2}$, com $x>0$. 
      Então: 
        \begin{enumerate}
        \item Dado $y \in \mathbb{R}$. Mostre que existe uma única solução 
        de $e^{x}-x^{-1}-x/2=y$. Conclua que $f$ tem inversa.
        \item Verifique que $|f^{-1}(x)-f^{-1}(y)| \leq 2 |x-y|$, para todo 
        $x, y \in \mathbb{R}$.
        \end{enumerate}
      \item Seja $f(x)=3x+\cos x$. Mostre que (a) 
      $f$ é bijetora e (b) calcule $f^{-1}(1)$.
      \item Use o teorema de valor médio para 
      mostrar as seguintes desigualdades:
           \begin{enumerate}
           \item $\ln(1+x)<x$, para todo $x \neq -1$.
           \item $|\ln \frac{x}{y}|\leq |x-y|$, para todo 
           $a, b \in \mathbb{R}$, 
           com a $\geq 1$, $b \geq 1$.
           \item $x-y \leq e^{x}-e^{y}$, para todo $y, x$ 
           com $x \geq y \geq 0$.
           \item $a^{a}(b-a)<b^{b}-a^{a}$, para $a, b$ com $1 \leq a<b$.
           \end{enumerate}
      \item Podemos usar o teorema do valor médio na função 
      $f(x)=\frac{2x-1}{3x-4}$ no intervalo $[1,2]$? 
      Caso afirmativo, encontre os valores que 
      verifiquem. {\it Rpta:} Não se cumple as condições 
      do teorema do valor médio.  
      \item Mostre as identidades
        \begin{enumerate}
        \item $\arcsin (1-2y^{2})=2 \arcsin (y)$, para $y \in (-1,1)$.
        \item $\arcsin \frac{x-1}{x+1}+\frac{\pi}{2}=2 \arctan \sqrt{x}$,
        para todo $x \in \mathbb{R}$.
        \end{enumerate}            
      \end{enumerate}      
      
      
      
      
      
      
      
\end{document}










    \subsection{Regras de cálculos para limites}   
   Calcule os seguintes limites.  
    \begin{enumerate}
    \item $\lim_{u \rightarrow 1} 
    \frac{\sqrt{3+u^2}-2}{1-u}=-\frac{1}{2}$.
    \item $\lim_{t \rightarrow 4} 
    \frac{3-\sqrt{5+t}}{1-\sqrt{5-t}}=-\frac{1}{3}$.
    \item $\lim_{x \rightarrow 1} 
    \frac{\sqrt{3x-2}+\sqrt{x}-\sqrt{5x-1}}
    {\sqrt{x}-\sqrt{2x-1}}=-\frac{3}{2}$.
    \item Se $f(x)=\sqrt{1+3x}$. Calcule 
    $\lim_{h \rightarrow 0} 
    \frac{f(x+h)-f(x)}
    {h}=\frac{3}{2\sqrt{3x+1}}$.
    \item $\lim_{x \rightarrow 1} 
    \frac{x^{100}-2x+1}
    {x^{50}-2x+1}=\frac{49}{24}$.
    \item $\lim_{x \rightarrow 2} 
    \frac{\sqrt{1+\sqrt{2+x}}-\sqrt{3}}
    {x-2}=\frac{1}{8\sqrt{3}}$.
     \item Se 
    $\lim_{x \rightarrow 0} 
    \frac{f(x)-1}
    {x}=1$. Prove $\lim_{x \rightarrow 0} 
    \frac{f(ax)-f(bx)}
    {x}=a-b$. {\it Dica:} Considere se $a$ e $b$ são iguais a zero ou não.
    \item Dado $a \in \mathbb{R}$. Mostre que $\lim_{x \rightarrow a} 
    \frac{x\sqrt{x}-a\sqrt{a}}
    {\sqrt{x}-\sqrt{a}}=3a$.
    \end{enumerate}
    \subsection{Limites laterais}   
   Calcule, se existe, os seguintes limites.  
    \begin{enumerate}
    \item $$\lim_{x \rightarrow \frac{5}{2}} 
    \sqrt{|x|+\lbrack\!\lbrack 3x \rbrack\!\rbrack}. $$ Sim, e o limite é $\sqrt{19/2}$.
     \item $$\lim_{x \rightarrow \frac{7}{3}} 
    \sqrt{|x|+\lbrack\!\lbrack 3x \rbrack\!\rbrack}. $$ Não.
    \item $$\lim_{x \rightarrow -3} 
    \frac{\lbrack\!\lbrack x-1 \rbrack\!\rbrack -x}
    {
    \sqrt{|x|^2-\lbrack\!\lbrack x \rbrack\!\rbrack}
    }.$$ Não.
     \item $$\lim_{x \rightarrow 1} 
    \frac{\lbrack\!\lbrack x\rbrack\!\rbrack^{2} -x^{2}}
    {
    \lbrack\!\lbrack x\rbrack\!\rbrack^{2} -x
    }.$$ Não.
    \item Considere a função 
     $$
    f(x)= \left\{  
            \begin{array}{lll}
    &\frac{x^3+3x^2-9x-27}{x+3} &\text{, se } x \in (-\infty, -3) \\
    &ax^2-2bx+1     &\text{, se } x \in [-3,3] \\
            & \frac{x^2-22x+57}{x-3}     &\text{, se } x \in (3,\infty) \\
            \end{array}
            \right. 
    $$
    Para quais valores de $a$ e $b$, existe os limites de $f$ em $x=-3$ e $x=3$? {\it Rpta: } $a=-1, b=4/3$.
    \end{enumerate}
    \subsection{Limites Trigonométricos}   
   Calcule os seguintes limites.  
    \begin{enumerate}
    \item $$\lim_{x \rightarrow \pi} \frac{1-\sin(\frac{x}{2})}{x-\pi}=0$$
    \item $$\lim_{x \rightarrow 0} \frac{x-\sin(x)}{x^2}=0. $$
    \item $$\lim_{x \rightarrow 0} \frac{1-\sqrt{\cos(x)}}{x^2}=\frac{1}{4}$$
    \item $$\lim_{x \rightarrow 1} 
    \frac{\cos(\frac{\pi x}{2})}{1-\sqrt{x}}=\pi$$
    \item $$\lim_{x \rightarrow 0} 
    \frac{1-\cos^7(x)}{x^2}=\frac{7}{2}$$
    \item $$\lim_{x \rightarrow 0} 
    \frac{1-\cos(1-\cos(x))}{x^4}=\frac{1}{8}$$
    \item $$\lim_{x \rightarrow \frac{\pi}{4}} 
    \frac{\sin(2x)-\cos(2x)-1}{\sin(x)-\cos(x)}=\sqrt{2}$$
    \item $$\lim_{x \rightarrow 1} 
    \frac{\sin(\pi x)+\cos(\frac{\pi x}{2})}
    {\tan(\frac{\pi x}{4})-1}=-\frac{3 \pi}{4}$$
    \end{enumerate}
    \subsection{Definição de limite}   
   Usando a definição de limite. Prove que 
    \begin{enumerate}
    \item $\lim_{x \rightarrow 2} \frac{x^2+1}{x-1}=5$.
    \item $\lim_{x \rightarrow 3} 
    \frac{1}{x^2+16}=\frac{1}{25}$. 
    \item $\lim_{x \rightarrow \frac{1}{2}} x^{2}\lbrack\!\lbrack x+2 \rbrack\!\rbrack=\frac{1}{2}$. 
    \item  $\lim_{x \rightarrow 4} 
    \frac{x^3-15x-4}{x-3}=0$.
    \item $\lim_{x \rightarrow a}\cos(x)=\cos(a)$, para qualquer $a \in \mathbb{R}$.  
    \end{enumerate}
     \subsection{Teorema de confronto e variantes}   
    \begin{enumerate}
    \item Se $f:\mathbb{R}\rightarrow \mathbb{R}$ uma função tal que $|f(x)|\leq 3|x|$, $\forall x \in \mathbb{R}$.
    Calcule $\lim_{x \rightarrow 0} \frac{f(x^3)}{x}$.
    \item Considere duas funções 
    $f,g:\mathbb{R}\rightarrow \mathbb{R}$ com a propriedade que $|\sin(x)|\leq g(x)\leq 4|x|$ e 
    $0 \leq f(x)\leq 1+|\sin(1/x)|$ para todo $x \neq 0$. 
    Calcule $\lim_{x \rightarrow 0} (f(x)g(x)+\cos(x))$.
    {\it Rpta: } 1.
    \item Se $f:\mathbb{R}\rightarrow \mathbb{R}$ uma função tal que $1+x^2+\frac{x^6}{3}\leq f(x)+ 1 \leq 
    \sec{x^2}+\frac{x^6}{3}$, para todo $x \in \mathbb{R}$. Calcule
    $$
    \lim_{x\rightarrow 0}f(x) \text{ e } 
    \lim_{x\rightarrow 0}f(x)\cos\left(\frac{1}{x^2+1}\right).
    $$ {\it Rpta: } Ambos são zeros. 
    \end{enumerate}
\end{document}

  
