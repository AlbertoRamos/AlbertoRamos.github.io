% lista 4(analise II 2017 I)
\documentclass{article}
%\usepackage{amssymb,latexsym,amsthm,amsmath}
\usepackage{amsmath,amsfonts,amssymb}
\usepackage{tikz}
\usepackage{verbatim}
\usepackage[brazil]{babel}
%\usepackage[latin1]{inputenc}
% parece que são conflictantes 
\usepackage[utf8]{inputenc}
%\usepackage{amsfonts}
% \usepackage{showlabels}
\usepackage{latexsym}
%%%%%%%%%%%%%%%%%%%%%%%%%
\usepackage{tikz}
\usetikzlibrary{patterns,arrows}
\usetikzlibrary{arrows.meta,calc,decorations.markings,math,arrows.meta}
%%%%%%%%%%%%%%%%%%%%%%%%%
\usepackage[pdftex]{hyperref}
%%%%%%%%%%%%%%%%%%%%%%%%%
%\usepackage{xcolor} %, colortbl}
%\usepackage{tabularx,colortbl}
%\usepackage{hyperref}
%\usepackage{graphicx}
%%%%%%%%%%%%%%%%%%%%%%
%\usepackage{tcolorbox}
%\usepackage{verbatimbox}
%%%%%%%%%%%%%%%%%%%%%%
%\usepackage{framed,color}
%\definecolor{shadecolor}{rgb}{1,0.8,0.3}
%%%%%%%%%%%%%%%%%%%%%
%\usepackage{fancybox}

%\theoremstyle{plain}
\newtheorem{theorem}{Theorem}[section]
\newtheorem{corollary}[theorem]{Corollary}
%\newtheorem*{main}{Main~Theorem}
\newtheorem{lemma}[theorem]{Lemma}
\newtheorem{proposition}[theorem]{Proposition}
\newtheorem{algorithm}{Algorithm}[section]
%\theoremstyle{definition}
\newtheorem{definition}{Definition}[section]
\newtheorem{example}{Example}[section]
\newtheorem{counter}{Counter-Example}[section]

%\theoremstyle{remark}
\newtheorem{remark}{Remark}

\headheight=21.06892pt
\addtolength{\textheight}{3cm}
\addtolength{\topmargin}{-2.5cm}
\setlength{\oddsidemargin}{-.4cm}
%\setlength{\evensidemargin}{-.5cm}
\setlength{\textwidth}{17cm}
%\addtolength{\textwidth}{3cm}
\newcommand{\R}{{\mathbb R}}

%%%%%%% definição de integral superior e inferior 
\def\upint{\mathchoice%
    {\mkern13mu\overline{\vphantom{\intop}\mkern7mu}\mkern-20mu}%
    {\mkern7mu\overline{\vphantom{\intop}\mkern7mu}\mkern-14mu}%
    {\mkern7mu\overline{\vphantom{\intop}\mkern7mu}\mkern-14mu}%
    {\mkern7mu\overline{\vphantom{\intop}\mkern7mu}\mkern-14mu}%
  \int}
\def\lowint{\mkern3mu\underline{\vphantom{\intop}\mkern7mu}\mkern-10mu\int}

\begin{document}

\title{Lista 4: Analise II}

\author{
A. Ramos \thanks{Department of Mathematics,
    Federal University of Paraná, PR, Brazil.
    Email: {\tt albertoramos@ufpr.br}.}
}

\date{\today}
 
\maketitle

\begin{abstract}
{\bf Lista em constante atualização}.
 \begin{enumerate}
 %\item Transformação de coordenadas;
 %\item Equação da parabóla; %Equação da reta e do plano; %Vetores (no plano e no espaço);
 \item Teorema de Weiestrass e teorema de Arzela-Áscoli; 
 \item Série de Fourier.
 \end{enumerate}
\end{abstract}

%%%%%%%%%%%%%%%%%%%%%%%%%%%%%%%%%%%%%%%%%%%%%%  
    \begin{enumerate}
      \item Mostre a identidade 
      $$ \frac{x(1-x)}{n}= \sum_{k=0}^{n}
         x^k(1-x)^{n-k}\left(x-\frac{k}{n}\right)^2,  $$
       para $x \in [0,1]$.  
       \item Seja $f:[0,1]\rightarrow \mathbb{R}$ uma função continua. Se $\int_{0}^{1} x^{n} f(x)dx=0$ para todo $n \in \mathbb{N}$, então $f(x)=0$ para todo $x \in [0,1]$.  
       Dê um contra-exemplo, quando $f$ não for contínuo.  
       \item Seja $f:[0,1]\rightarrow \mathbb{R}$ uma função continua. Usando o teorema de aproximação de Weiestrass, 
       mostre que existem uma sequência de polinômios $p_{n}$
       tal que $\sum_{n} p_{n}=f$ em $[0,1]$.
      \item Seja $\mathcal{P}_{k}$, o conjunto de polinômios de grau menor ou igual a $k$ e $I$ um intervalo compacto. 
      Dado $M>0$, defina   
      $\mathcal{P}_{k}(I;M):=\{p \in \mathcal{P}_{k}:
      |p(x)|\leq M, \forall x \in I\}$.
      Prove que $\mathcal{P}_{k}(I;M)$ é equicontinuo. 
      \item Mostre que não existe polinômios $p_{n}$ tal que 
      $p_{n} \xrightarrow{u} f$ em $\mathbb{R}$, onde 
      $f(x)=\sin(x)$ ou $f(x)=exp(x)$. Por que não existe contradição com o teorema de aproximação de Weirestrass ?
      \item Defina $f:(0,1)\rightarrow \mathbb{R}$ com 
      $f(x)=1/x$. Mostre que não existe sequência de polinômios
       $p_{n} \xrightarrow{u} f$ em $(0,1)$.
      \item Considere a sequência $f_{n}(x):=nx^3$. 
      Mostre que $f_{n}$ possui derivadas limitadas 
      no ponto $x=0$, mas $f_{n}$ não é equicontinua 
      nesse mesmo ponto.    
      \item Seja $f: I \times [a,\infty) \rightarrow \mathbb{R}$
      contínua e suponha também que 
      $F(t):=\int_{a}^{\infty} f(t,x)dx$, para todo $t \in I$.
      Defina $F_n(t):=\int_{a}^{n} f(t,x)dx$, $t \in I$, 
      $n \in \mathbb{N}$. 
      Mostre que a integral 
      $\int_{a}^{\infty} f(t,x)dx$
      converge uniformemente em $I$ se e somente se, 
      se $\{F_{n}\}$ converge uniformemente para $F$ em $I$.
      \item Seja $f_{n}$ uma sequência equicontínua e simplesmente limitada num compacto $K \subset \mathbb{R}$. 
      Suponha que toda subsequência uniforme convergente em $K$  tem o mesmo limite $f:K \rightarrow \mathbb{R}$. 
      Então, $f_{n}$ converge uniformemente a $f$ em $K$.
      \item Dê um exemplo de uma sequência 
      equicontínua de funções 
      $f_{n}:(0,1)\rightarrow (0,1)$ que não possua subsequência 
      uniformemente convergente em $(0,1)$.       
      \item Considere uma sequência de funções 
      $f_{n}:I \rightarrow \mathbb{R}$ de 
      classe $\mathcal{C}^2$.
      Suponha que (i) $f_{n}$ converge a $f$ em $I$; 
      (ii) existe algum $a \in I$ tal que $\{f'_{n}(a)\}$
      é limitada e (iii) $\{f''_{n}\}$ é uniformemente limitada em $I$. Mostre que $f$ é de classe $C^1$. 
      {\it Dica: } Use a equicontinuidade. 
      \item Calcule a série de Fourier das seguintes funções definidas em $[-\pi, \pi)$ e calcule dita soma      
         \begin{enumerate}
         \item $f(x)=a, 
         \text{ se }(-\pi\leq x<0) \text{ e } f(x)=b,
         \text{ se } (0\leq x<\pi)$.
         \item $f(x)=ax, 
         \text{ se }(-\pi\leq x<0) \text{ e } 
         f(x)=bx, \text{ se } (0\leq x<\pi)$.
          \item $f(x)=|x|$ e 
          $f(x)=\exp(\alpha x)$, $\alpha \neq 0$.
         \item $f(x)=\sin^2(x)$ e $f(x)=ax+b$ 
         \end{enumerate}            
      \item Defina $f(x)=x$, se $0\leq x <2\pi$. Use o teorema de Parseval para concluir que 
      $$  \sum_{n=1}^{\infty} \frac{1}{n^2}= \frac{\pi^2}{6}.$$      
      \item Considere $\alpha \in (0, \pi)$. 
      Defina 
      $$f(x)=1, \text{ se } |x|\leq \alpha \text{ e } 
        f(x)=0, \text{ se } \alpha< |x|\leq \pi.$$
      Além disso, defina $f(x+2\pi)=f(x)$ para todo 
      $x \in \mathbb{R}$. 
        \begin{enumerate}
        \item Calcule os coeficiente de Fourier de $f$
        \item Mostre que 
        $$  \sum_{n=1}^{\infty} \frac{\sin(n\alpha)}{n}=\frac{\pi-\alpha}{2}, (0 < \alpha < \pi). $$
        \item Usando o teorema de Parseval, prove que 
        $$ \sum_{n=1}^{\infty} \frac{\sin^{2}(n\alpha)}{n^2 \alpha}=\frac{\pi-\alpha}{2}. $$
        \item Faça $\alpha$ ir para zero e prove que 
        $$ \int_{0}^{\infty} \left(\frac{\sin x}{x}\right)^2=\frac{\pi}{2}.   $$
        \end{enumerate}  
      \item Considere $f : [-\pi,\pi)\rightarrow \mathbb{R}$
      de classe $C^1$. Suponha que existe $k \in \mathbb{N}$
      tal que 
      $$
        \lim_{n \rightarrow \infty} n^{k}|a_n|=0 
        \text{ e } 
        \lim_{n \rightarrow \infty} n^{k}|b_n|=0, 
      $$  
      onde $a_n$ e $b_n$ são os coeficientes de 
      Fourier da função $f$. 
        \begin{enumerate}
        \item Mostre que $f$ é de classe $C^k$
        \item Prove que as derivadas $f^{(m)}$ (com $m \leq k$)
        é a derivada $m$-ésima da série de Fourier,  
        com convergência uniforme da série resultante.
        \end{enumerate}         
      \item Sejam $f,g : [-\pi,\pi)\rightarrow \mathbb{R}$
      funções de classe $C^1$ por partes. 
      Se ambas funções possuem a mesma
      série de Fourier, então $f$ e $g$ são funções identicas.                
      \item Seja $a \in \mathbb{R}$ com $|a|<1$. 
      Encontre as funções cujas séries de Fourier 
      são dadas por
     %     \begin{enumerate}
     %     \item $\sum_{n=1}^{\infty} \frac{a^n \cos(nx)}{n}$, 
     %     se $|a|<1$
     %     \item $\sum_{n=1}^{\infty} \frac{a^n \sin(nx)}{n}$, 
     %     se $|a|<1$
     %     \item $\sum_{n=1}^{\infty} \frac{\cos(nx)}{n!}$
     %     \end{enumerate}
     $$
       (a) \sum_{n=1}^{\infty} \frac{a^n \cos(nx)}{n}, \ \ \ \ \ \ \ \ 
       (b) \sum_{n=1}^{\infty} \frac{a^n \sin(nx)}{n}, \ \ \ \ \ \ \ \
       (c) \sum_{n=1}^{\infty} \frac{\cos(nx)}{n!}
     $$                    
      \end{enumerate}
\end{document}




















        
      \item {\bf vértices: } Interseção do eixo focal com a elipse.   A interseção é dada por dois pontos, denotado $V_1$ e $V_2$;;
      \item {\bf centro : } Ponto meio do segmento $F_1F_2$;
      \item {\bf eixo normal (eixo conjugado): } reta perperndicular ao eixo focal que passa pelo centro; 
      \item {\bf corda: } qualquer segmento de une dois pontos diferentes da elipse;
      \item {\bf corda focal: } corda que passa por algum foco;
      \item {\bf lado reto : } corda focal paralela à reta normal;
      \item {\bf raio vetor: } segmento de reta que une algum 
      foco com algum ponto da parábola;
      \item {\bf diámetro: } corda que passsa pelo centro.
      \item {\bf eixo maior: } segmento $V_1V_2$. 
      Observe que o eixo maior tem comprimento $2a$;
      \item {\bf eixo menor: } segmeto definido pela interseção da elipse 
      com a reta normal. Note que o eixo menor tem medida $2b$ ;
      \item {\bf retas diretrizes: }  retas paralelas à reta normal 
      cuja distância ao centro $C$ é $a/e$.
     \end{enumerate}
     Observe que 
     $$ dist(V_1,V_2)=2a \ \ (\text{ eixo maior da elipse }), \ \ 
        dist(F_1,F_2)=2c \ \ (\text{ distância focal }).$$
 
 {\it Remark 1: }Note que a elipse é simetrica em relação ao eixo focal e também ao eixo normal. 
 
 {\it Remark 2: } Veja a 
 construção geometrica da elipse na internet, 
 por exemplo, 
 \url{https://www.youtube.com/watch?v=RYV-uBWdb8Y}.
  
 Usando um sistema de coordenadas a elipse $\mathcal{E}$ pode ser escrita com uma das seguintes formas. 
 
 {\bf Forma canônica }(também chamada de forma reduzida) 
 $\frac{x^2}{a^2}+\frac{y^2}{b^2}=1$ ou $\frac{x^2}{b^2}+\frac{y^2}{a^2}=1$, 
 onde o centro $C=(0,0)$ e o eixo focal é paralelo a algum dos eixos canônicos.
 {\it Desenhe ambas elipse explicitando o segmento que tem comprimento $a$ e/ou $b$}.
 
  Quando o centro $C=(h,k)$ e o eixo focal é paralelo a algum dos eixos canônicos, 
  $\frac{(x-h)^2}{a^2}+\frac{(y-k)^2}{b^2}=1$ ou $\frac{(x-h)^2}{b^2}+\frac{(y-k)^2}{a^2}=1$, 
  %$(y-k)^{2}=4p(x-h)$ ou $(x-h)^2=4p(y-k)$.

 {\bf Forma geral sem rotação} $x^2+y^{2}+Dy+Ex+F=0$.
 onde o eixo focal é paralelo a algum dos eixos canônicos.
 
 {\bf Forma geral mesmo} $Ax^2+Bxy+Cy^{2}+Dy+Ex+F=0$ se $B^{2}-4AC<0$. \newline
 
 {\it Retas tangentes para a Elipse. }
 Em qualquer ponto sobre a elipse podemos calcular retas tangentes e retas normais.
 
 {\bf Quando $\frac{x^2}{a^2}+\frac{y^2}{b^2}=1$}. 
 A reta tangente à $\mathcal{E}$ no ponto $P=(x_0,y_0) \in \mathcal{E}$ é dada por 
 $ r: (\frac{x_{0}}{a^2})x+(\frac{y_0}{b^{2}})y=1$. 
 
 {\bf Quando $\frac{x^2}{b^2}+\frac{y^2}{a^2}=1$}. 
 A reta tangente à $\mathcal{E}$ no ponto $P=(x_0,y_0) \in \mathcal{E}$ é dada por 
 $ r: (\frac{x_{0}}{b^2})x+(\frac{y_0}{a^{2}})y=1$. \newline
  
  Com essas informações responda:
    \begin{enumerate}
    \item Calcule os focos, vértices, a medida do eixo maior e a do eixo menor, esboce as elipses
       \begin{enumerate}
       \item $x^2/9+y^2/25=1  \ \  \text{ e } \ \ 4x^2+10y^2=40$
       \item $4x^2+169y^2=676  \ \ \text{ e } \ \ 16x^2-4+4y^2=0$ 
       \end{enumerate}
    \item Escreve a equação reduzida da elipse nos seguintes casos:
     \begin{enumerate}
     \item $Centro=(0,0)$, eixo focal paralelo ao eixo $x$, o eixo menor mede 6 e a distância focal é 8. 
     \item Os focos são $(0,6)$ e $(0,-6)$ e o eixo maior mede 34
     \item $Centro=(0,0)$, um foco é $(0,-\sqrt{40})$ e o ponto 
     $(\sqrt{5}, 14/3)$ pertence à elipse.
     \item Os focos são $F_1=(1,1)$ e $F_2=(-1,-1)$
     e satisfaz $dist(P,F_1)+dist(P,F_2)=4$
     \end{enumerate} 
    \item Considere uma elipse com foco $F=(-2,0)$ que passa 
    por $P=(2,-3)$ e tem como reta diretriz é $r:x+8=0$. Encontre a 
    excentricidade da elipse. {\it Rpta: } $e=1/2$.     
    \item  Encontre a equeção da elipse cujo focos e vértices
    coincidem com os focos e vértices das parábolas 
    $\mathcal{P}_1:y^2+4x=12$ e  
    $\mathcal{P}_2:y^2-4x=12$. 
    {\it Rpta: } $5x^2+9y^2=45$. 
    \item Considere a equação da elipse em forma reduzida. Mostre que se 
    $(x_0,y_0)$ está na elipse, os pontos 
    $(x_0,-y_0)$, $(-x_0,y_0)$ e $(-x_0,-y_0)$ também pertencem à elipse. 
    \item Se a distância entre as diretrizes de uma elipse é 18, e os focos 
    são os pontos (1,5) e (1,3). Encontre a equação da elipse. 
    {\it Rpta: } $9(x-1)^2+8(y-4)^2=72$.
    \item Considere a elipse $b^2x^2+a^2y^2=a^2b^2$, com 
    foco $F_1=(c,0)$, $F_2=(-c,0)$ e um ponto 
    $P=(x_0,y_0)$ da elipse. Mostre que 
    o raio vetor $PF_1$ é igual $a-ex_0$ 
    (i.e $|\overrightarrow{PF_1}|=a-ex_0$ )
    e  raio vetor $PF_2$ é $a a+ex_0$ 
    (i.e $|\overrightarrow{PF_2}|=a+ex_0$ )
    \item Encontre a equação da corda focal da 
    elipse $16x^2+25y^2=400$, cujo comprimento é 8 unidades e 
    passa por o foco com coordenadas positivas. {\it Rpta: }
    $\sqrt{2}y\pm 2(x-3)=0$.
    {\it Dica: } Considere que a corda é PQ, onde P e Q estão na elipse. 
    (1) Encontre primeiro o foco $F=(f_1,f_2)$,  
    (2) Escreva a equação da reta que define a corda PQ,
    tipo $y-f_1=m(x-f_2)$, onde $m$ é a incognita,  
    (3) Note que o comprimento do segmento PQ é igual à soma dos segmentos
    PF e FQ, (4) Use o problema anterior para calcular PF e FQ. 
    \item Encontre a equação da elipse com centro $(1,-3)$, com um foco 
    em $(0,-6)$ e a interseção do eixo focal com uma diretriz da elipse 
    é $(3,3)$. {\it Rpta: } $e=1/\sqrt{2}$, $\mathcal{E}: 19x^2-6xy+11y^2-56x+72y-64=0$.
    \item Seja $\mathcal{E}$ uma elipse e $P$ um ponto exterior à elipse 
    ($P \notin \mathcal{E}$). Encontre as retas tangentes da elipse que passam por P, nos seguintes casos: 
       \begin{enumerate}
       \item $\mathcal{E}: 9y^2+4x^2=72$, $P=(0,4)$ 
       {\it Rpta: } $2x+3y-12=0$, $2x-3y+12=0$;
       \item $\mathcal{E}: 2x^2+3y^2+x-y=5$, $P=(3,-1)$
       {\it Rpta: } $x+y=2$, $9x-191y=218$.
       \item A reta $r: 2x-y-3=0$ é tangente à elipse 
       $\mathcal{E}:9x^2+16y^2=144$? 
       {\it Rpta: } não, $r$ é uma reta secante 
       (i.e. corta a elipse em dois pontos)
       \item A reta $r: 2x+y=10$ é tangente à elipse 
       $\mathcal{E}:4x^2+9y^2=36$? 
       {\it Rpta: } não, 
       $r$ não intercepta a elipse.
       \end{enumerate}
    \item Seja $\mathcal{E}: x^2+3y^2 +3x-4y=3$. 
    Encontre os valores de $\alpha \in \mathbb{R}$ 
    para que as retas $5x+2y+\alpha$ sejam tangentes à elipse. 
    {\it Rpta: } $\alpha=-7$ e $\alpha=58/3$.
    \item {\it Propriedade refletora da Elipse: }  
    distancia focal prova
      \begin{enumerate}
      \item as consequência várias coisas chevere.
      \end{enumerate}
    \item {\bf problema mais faceis}
    \end{enumerate}

%%%%%%%%%%%%%%%%%%%%%%%%%%%%%%%%%%%%%%%%%%%%%%  
\section*{Hipérbole} 
 
 Dado dois pontos $F_1$ e $F_2$ no plano, e dois números positivos $a$ e $c$ ($c> a$)
   com $dist(F_1,F_2)=2c$.  
   A hipérbole é o conjunto  
   $$\mathcal{H}:=\{P \in \mathbb{R}^{2}: |dist(P,F_1)-dist(P,F_2)|=2a\}.$$
   Os pontos $F_1$ e $F_2$ são chamados de focos.
   Defina $b:=\sqrt{c^{2}-a^{2}}$. 
   Por definição de $b$, 
   temos que $c^{2}=b^2+a^2$ ({\bf perceba as diferenças com a hipérbole}). 
   O número $e:=\frac{c}{a}$ é chamado 
   de {\it excentricidade} da hipérbole. 
   Veja que para a hipérbole $e>1$.
    \begin{enumerate}
      \item {\bf eixo focal (eixo transverso): } reta que contem os focos $F_1$ e $F_2$;
      \item {\bf vértices:  } Interseção do eixo focal com a 
      hipérbole. 
      A interseção são dois pontos denotados por $V_1$ e $V_2$;
      \item {\bf centro : } Ponto meio do segmento $F_1F_2$;
      \item {\bf eixo normal (eixo conjugado): } reta perperndicular ao eixo focal que passa pelo centro; 
      \item {\bf corda: } qualquer segmento de une dois 
      pontos diferentes da hipérbole;
      \item {\bf corda focal: } corda que passa por algum foco;
      \item {\bf lado reto : } corda focal paralela ao eixo normal;
      \item {\bf raio vetor: } segmento de reta que une algum 
      foco com algum ponto da hipérbole;
      %\item {\bf diámetro: } corda que passsa pelo centro.
      \item {\bf eixo maior: } segmento $V_1V_2$. 
      Observe que o eixo maior tem comprimento $2a$;
      \item {\bf eixo menor: } segmento definido pela interseção 
      da hipérbole com o eixo normal. 
      O eixo menor tem medida $2b$ ;
      \item {\bf retas diretrizes: }  retas paralelas à reta normal 
      cuja distância ao centro $C$ é $a/e$.
      \item {\bf rectângulo fundamental: } rectângulo cujo centro é o centro da hipérbole, com lados de comprimento 2a e 2b e paralelos aos eixo transveso e 
      conjugado respectivamente.
      \item {\bf assintotas: } retas que passam por $C$, não interceptam 
      à hipérbole mas tendem à hipérbole no infinito. 
      %aproxima-sem com distância zero.  
      Ditas retas são definas pelas diagonais do rectângulo fundamental. 
      \item {\bf ramo da hipérbole: }  cada uma das curvas que definem a hipérbole.
     \end{enumerate}
     Observe que 
     $$ dist(V_1,V_2)=2a \ \ (\text{ eixo maior da elipse }), \ \ 
        dist(F_1,F_2)=2c \ \ (\text{ distância focal }).$$
 
 {\it Remark 1: }Note que a hipérbole é simetrica em relação ao eixo focal e  ao eixo normal. 
 
 {\it Remark 2: } Veja a 
 construção geometrica da hipérbole na internet, 
 por exemplo, 
 \url{https://www.youtube.com/watch?v=ETV_bWAPOqU}. 
  
 Usando um sistema de coordenadas a hipérbole $\mathcal{H}$ 
 pode ser escrita com uma das seguintes formas. 
 
 {\bf Forma canônica }(também chamada de forma reduzida).
 Nesta caso, a hipérbole é o lugar geometrico definido por 
 %As equações são  
 $\frac{x^2}{a^2}-\frac{y^2}{b^2}=1$ ( hipérbole horizontal) ou $\frac{y^2}{a^2}-\frac{x^2}{b^2}=1$ ( hipérbole vertical), 
 onde o centro $C=(0,0)$ e o eixo focal é paralelo a algum dos eixos canônicos.
 {\it Desenhe ambas hipérbole explicitando 
 o segmento que tem comprimento $a$ e/ou $b$. Lembre $dist(V_1,V_2)=2a$}.
 
 {\it Remark:} Nesse caso as assíntotas podem ser facilmente calculadas. 
 De fato:
   \begin{enumerate}
   \item Quando $\mathcal{H}$ é uma hipérbole horizontal, 
   as assíntotas são as retas $y=\pm \frac{b}{a} x$;
   \item Quando $\mathcal{H}$ é uma hipérbole vertical, 
   as assíntotas são as retas $y=\pm \frac{a}{b} x$.
   \end{enumerate}
   
  Quando o centro $C=(h,k)$ e o eixo focal é paralelo a algum dos eixos canônicos, temos que a hipérbole pode ser descrita como 
  $\frac{(x-h)^2}{a^2}-\frac{(y-k)^2}{b^2}=1$ ou 
  $\frac{(y-k)^2}{a^2}-\frac{(x-h)^2}{b^2}=1$. 
  %$(y-k)^{2}=4p(x-h)$ ou $(x-h)^2=4p(y-k)$.

 {\bf Forma geral sem rotação} $Ax^2-Cy^{2}+Dy+Ex+F=0$.
 onde o eixo focal é paralelo a algum dos eixos canônicos.
 
 {\bf Forma geral mesmo} $Ax^2+Bxy+Cy^{2}+Dy+Ex+F=0$ se $B^{2}-4AC>0$. \newline
 
 {\it Retas tangentes para a hipérbole. }
 Em qualquer ponto sobre a hipérbole 
 podemos calcular retas tangentes e retas normais.
 
 {\bf Quando $\frac{x^2}{a^2}-\frac{y^2}{b^2}=1$}. 
 A reta tangente à $\mathcal{H}$ no ponto $P=(x_0,y_0) \in \mathcal{H}$ é dada por 
 $ r: (\frac{x_{0}}{a^2})x-(\frac{y_0}{b^{2}})y=1$. 
 
 {\bf Quando $\frac{y^2}{a^2}-\frac{x^2}{b^2}=1$}. 
 A reta tangente à $\mathcal{H}$ no ponto $P=(x_0,y_0) \in \mathcal{H}$ 
 é dada por 
 $ r: (\frac{y_0}{a^{2}})y-(\frac{x_{0}}{b^2})x=1$. \newline
  
 Com essas informações responda:
 \begin{enumerate}
     \item Calcule os focos, vértices, as equações das assíntotas. Esboce as hipérboles
       \begin{enumerate}
       \item $16x^2-25y^2=400  \ \  \text{ e } \ \ 9y^2-4y^2=36$
       \item $x^2-y^2+1=0  \ \ \text{ e } \ \ x^2-4y^2=1$ 
       \end{enumerate}
    \item Escreve a equação reduzida da elipse nos seguintes casos:
     \begin{enumerate}
     \item Os focos são $F_1=(3,-1)$ e $F_2=(3,4)$
     e satisfaz $|dist(P,F_1)-dist(P,F_2)|=3$;
     \item Os focos são $F_1=(-1,1)$ e $F_2=(1,1)$
     e satisfaz $|dist(P,F_1)-dist(P,F_2)|=1$;
     \item Os vértices são 
     $(2,0)$ e $(-2,0)$ e os focos são 
     $(3,0)$ e $(-3,0)$;
     \item Os vértices são 
     $(15,0)$ e $(-15,0)$ e as assíntotas são 
     $5y-4x=0$ e $5y+4x=0$.
     \end{enumerate} 
    \item Encontre a equação da hipérbole cujos focos 
    são $(4,0)$ e $(-4,0)$, e o coeficiente ângular duma das assíntotas 
    é 3. {\it Rpta: } $\mathcal{H}: 45x^2-5y^2=72$.
    \item Seja uma hipérbole com centro na origem, focos sobre o eixo x cuja distância entre as diretrizes é 4 e passa por $P=(4,3)$.
    {\it Rpta: } $\mathcal{H}: 3x^2-2y^2=30$
    \item Considere a elipse $\mathcal{E}: 25x^2+9y^2=225$. 
    Se os focos dessa elipse coincidem 
    com os focos duma hipérbole de excentricidade 4/3.
    Escreva a equação reduzida da hipérbole.
    {\it Rpta: } $\mathcal{H}: 7y^2-9x^2=63$.
    \item Calcule a àrea do triângulo formado por as assíntotas de 
    hipérbole $\mathcal{H}: x^2-4y^2=16$ e  a reta 
    $r: 3x-2y+12=0$. {\it Rpta: } $9 u^2$
    \item Encontre a equação reduzida de uma hipérbole se os focos são 
    os pontos $(-10,0)$ e $(10,0)$, e suas assíntotas são as retas 
    $r: y=\pm 2x$. {\it Rpta: } $\mathcal{H}: 4x^2-y^2=80$.
    \item Se as assíntotas duma hipérbole, 
    que tem um foco em $(3,-2)$, são 
    $r_1: 3x-4y-5=0$ e $r_2: 3x+4y+11=0$. 
    Encontre a sua excentricidade. {\it Rpta: } $e=5/4$. 
    \item É possível construir uma hipérbole com focos em 
    $(3,4)$ e $(-1,-2)$ tal que a {\bf medida do eixo maior} é 2.
    Caso afirmativo, escreva a equação de dita hipérbole.
    {\it Rpta: } Sim, $\mathcal{H}: 3x^2+8y^2+12xy-18x-28y+11=0$.
    {\it Dica: } Use a definição da hipérbole. 
    \item Considere a hipérbole $b^2x^2-a^2y^2=a^2b^2$, com 
    foco $F_1=(c,0)$, $F_2=(-c,0)$ e um ponto 
    $P=(x_0,y_0)$ da hipérbole. Mostre que 
    o raio vetor $PF_1$ é igual $|a-ex_0|$ 
    (i.e $|\overrightarrow{PF_1}|=|a-ex_0|$ )
    e  raio vetor $PF_2$ é $|a+ex_0|$ 
    (i.e $|\overrightarrow{PF_2}|=|a+ex_0|$ )
    \item {\it Propriedade refletora da hipérbole: }  
    distancia focal prova
      \begin{enumerate}
      \item as consequência várias coisas chevere.
      \end{enumerate}
 \end{enumerate} 
 
 \section*{Estudo unificado das cônicas não degeneradas}
     
\end{document}
 
 
 
 
 
 
 
 
 
 
 
 
 
 
 
 
 
 
 
 
 
 
 
 
 
 
 
 
 
 
 
 
 
 
 
 
\section*{Transformação de coordenadas}
 Existe dois tipos de transformações importantes: {\it traslação} e {\it rotação}, quais podem ser combinados para descrever movimentos mais complexos. É importante se sentir confortável com essas transformações.
 
% \shadowbox{gdfgd}
  
 {\it Translação de eixos.} Em $\mathbb{R}^{2}$ (em $\mathbb{R}^{3}$ é similar), considere um sistema de coordenadas cuja origem $O=(0,0)$ é trasladada a $O'=(h,k)$.
 Seja $P \in \mathbb{R}^{2}$ um ponto com coordenadas 
 $(x,y)$ no sistema de coordenadas original e 
 com coordenadas $(x',y')$ no 
 novo sistema de coordenadas (com origem $O'$). Então, temos que :
   $$  x=x'+h \ \ \text{ e } \ \ y=y'+k.$$
 {\it Rotação de eixos.} Em $\mathbb{R}^{2}$, considere dois sistemas de coordenadas, uma obtida apartir da outras atraves de uma rotação (com ângulo 
 $\theta$ e sentido antihorário). Se
 $P \in \mathbb{R}^{2}$ um ponto com coordenadas 
 $(x,y)$ no sistema de coordenadas original e 
 com coordenadas $(x',y')$ no 
 novo sistema de coordenadas. Então, 
 $$  x=x'\cos(\theta)-y'\sin(\theta) \ \ \text{ e } \ \ 
     y=x'\sin(\theta)+y'\cos(\theta).$$
 Usando matrizes podemos escrever a expressão como (talvez mais fácil para decorar):
 $$  
     \begin{pmatrix}
      x \\
      y
     \end{pmatrix}= \begin{pmatrix}
                    \cos(\theta) & -\sin(\theta) \\
                    \sin(\theta) &  \cos(\theta) \\
                    \end{pmatrix}  
                     \begin{pmatrix}
                     x' \\
                     y'
                     \end{pmatrix}.
 $$                       
 Da expressão anterior temos que
  $$  
     \begin{pmatrix}
      x'\\
      y'
     \end{pmatrix}= \begin{pmatrix}
                    \cos(\theta)  &  \sin(\theta) \\
                    -\sin(\theta) &  \cos(\theta) \\
                    \end{pmatrix}  
                     \begin{pmatrix}
                     x \\
                     y
                     \end{pmatrix}.
 $$  
 %  é similar.  
 %Em $\mathbb{R}^{3}$, a rotação de dá atraves de dois ângulos 
 %de Euler. 
 Responda:
   \begin{enumerate}
    \item Usando uma translação transforme a equação 
    $2x^2-3xy+5x+3y-8=0$ em outra equação sem termos lineares.
    {\it Rpta:} Nova origem $O'=(1,3)$ e equação $2x'^2-3x'y'-1=0$
    \item Mediante uma translação transforme a equação 
    $8x^3-x^2+24x-y+1=0$, em outra que não tem termos 
    de segunda ordem nem termo constante.
    {\it Rpta} Nova origem $O'=(-4,8)$ e equação $(x')^{2}y'-1=0$.
    \item Encontre o ângulo de rotação para que a curva
    $8x^2+3\sqrt{3}xy+11y^2=24$ não tenha o termo $xy$.
    {\it Rpta} $\theta=60^{\circ}$.
    \item Mostre que a curva $11x^2+24xy+4y^2=20$ depois de uma rotação 
    $\theta=arctan(3/4)$ se escreve como $4x'^2-y'^2=4$.
    \item Usando primeiramente uma translação com nova origem $O'=(1,1)$ e
    logo uma rotação de $45^{\circ}$, uma equação se transforma 
    em $(x'')^{2}-2(y'')^{2}=2$. Qual é a equação original?
    {\it Rpta:} $x^2-6xy+y^2+4x+4y=0$.
   \end{enumerate}

 \section*{Parábola} 
  
% \begin{tcolorbox}
   Dada uma reta $\mathcal{D}$ e um ponto $F \notin \mathcal{D}$.
   A parábola é definida como 
   $$\mathcal{P}:=\{P \in \mathbb{R}^{2}: dist(P,F)=dist(P,\mathcal{D})\}.$$
   O ponto $F$ é chamado de foco e a reta $\mathcal{D}$ é chamada de reta diretriz.
     \begin{enumerate}
      \item {\bf eixo de simetria: } reta perpendicular à diretriz que passa por F;
      \item {\bf vértice: } interseção no eixo de simetria com a parábola;
      \item {\bf corda: } qualquer segmento de une dois pontos diferentes da parábola;
      \item {\bf corda focal: } corda que passa por F;
      \item {\bf lado reto (ou corda principal): } corda focal paralela à diretriz;
      \item {\bf raio vetor: } segmento de reta que une o foco com algum ponto da parábola. 
     \end{enumerate}
  
 %\end{tcolorbox}
 
 Usando um sistema de coordenadas a parábola $\mathcal{P}$ pode ser escrita com uma das 
 seguintes formas. 
 
 {\bf Forma canônica }(também chamada de {\it forma reduzida}) $y^{2}=4px$ ou $x^2=4py$.
 onde o vertice $V=(0,0)$ e o eixo de simetria é paralelo a algum dos eixos canônicos.
 $$ \text{ Observe que } |p|=dist(V,F) \text{ e } 
                         |p|=dist(V,\mathcal{D}). $$
 
 Quando o vertice $V=(h,k)$ e o eixo de simetria é paralelo a algum dos eixos canônicos temos que $(y-k)^{2}=4p(x-h)$ ou $(x-h)^2=4p(y-k)$.
 %onde o vertice $V=(h,k)$ e o eixo de simetria é paralelo a algum dos eixos  %canônicos.
 
 {\bf Forma geral} $y^{2}+Dy+Ex+F=0$ ou $x^{2}+Dx+Ey+F=0$
 onde o eixo de simetria é paralelo a algum dos eixos canônicos.
 
 %{\bf Forma geral mesmo}. \newline
 
 {\it Retas tangentes: } 
 Em qualquer ponto sobre a parábola podemos 
 calcular retas tangentes e retas normais.
 Para as retas tangentes temos as seguintes formulas qual depende da 
 equação usada da parábola.
 
 {\bf Quando $y^{2}=4px$}. A reta tangente a $\mathcal{P}$ no ponto $P=(x_0,y_0) \in \mathcal{P}$ é dada por 
 $ r: yy_0=2p(x+x_0)$. 
 
 {\bf Quando $x^{2}=4py$}. A reta tangente a $\mathcal{P}$ no ponto $P=(x_0,y_0) \in \mathcal{P}$ é dada por 
 $ r: xx_0=2p(y+y_0)$. \newline

 
Proceda a responder as seguintes questões 
  
  \begin{enumerate}
     \item Escreva as equações das seguintes parábolas.
       \begin{enumerate}
        \item Se $F=(0,2)$ e diretriz $\mathcal{D}: y+2=0$.
        \item Se $F=(0,0)$ e diretriz $\mathcal{D}: y+x=2$.
        \item Se o vértice é $V=(-3,2)$ e foco $F=(-1,2)$.
       \end{enumerate}
     \item Ache a equação da parabóla que tem foco
     $(-5/3,0)$ e cuja reta diretriz é $3x-5=0$.
     {\it Rpta} $3y^2+20x=0$.
     \item Encontre a longitude da corda focal da parábola
     $\mathcal{P}: x^2+8y=0$ que é paralela à reta 
     $r: 3x+4y-7=0$. {\it Rpta: } $25/2$.
     \item Encontre a equação da parábola com foco $F=(2,1)$, 
     com vértice sobre a reta $r: 3x+7y+1=0$ e 
     cuja diretriz é paralela ao eixo x. 
     {\it Rpta: } $\mathcal{P}: (x-2)^2=8(y+1)$.
     \item Se uma parábola tem um vértice sobre a
      reta $r_1:3x-2y=19$, o foco sobre a 
      reta $r_2: x+4y=0$ e 
      diretriz $\mathcal{D}: x=2$.
      {\it Rpta: } $(y+2)^{2}=12(x-5)$.
     \item Encontre a equação de uma parábola cuja lado reto 
     tem como extremo os pontos 
     $A=(7,3)$ e $B=(1,3)$. 
     {\it Rpta: } $\mathcal{P}_1:(x-4)^2=6(y-3/2)$
     e  $\mathcal{P}_2:(x-4)^2=-6(y-9/2)$.
     \item Encontre o valor de $\alpha \neq 0$, para que as 
     coordenadas do foco da parábola 
     $\mathcal{P}: x^2+4x-4\alpha y=8$ somem zero.
     {\it Rpta: } $\alpha=3$ ou $\alpha=-1$.
     \item Encontre a equação da circunferência 
     que passa por o vértice 
     e os extremos do lado reto da parábola 
     $\mathcal{P}: y^2+2y-4x+9=0$. 
     {\it Rpta: } $\mathcal{C}: (x-1/2)^2+(y+1)^2=9/4$.
     \item Encontre a equação da reta tangente e normal 
     da parábola $\mathcal{P}: y^2+2y-4x-7=0$ no ponto de contato 
     $T=(7,5)$ ($T \in \mathcal{P}$). {\it Rpta: }
     reta tangente: $x-3y+8=0$ e reta normal: $3x+y-26=0$.
     \item Encontre as retas tangentes à parabóla 
     $\mathcal{P}: y^2+3x-6y+9=0$ que passa por 
     $P=(1,4)$. {\it Rpta: }
     $3x-2y+5=0$ e $x+2y-9=0$.
     \item Considere a reta $r: x-2y-8=0$. 
     Ache o ponto da parabóla $x^2=4y$ tal que a 
     distância à reta $r$ seja a mínima possível e calcule 
     tal distância.
     {\it Dica: } O ponto deve ser ponto de tangência. 
     {\it Rpta: } $T=(1,1/4)$, distancia$=3\sqrt{5}/2$.
     \item * Considere a parabóla 
     $\mathcal{P}: x^2-2x+8y-23=0$, o ponto de tangência 
     $T=(5,1)$ e um triângulo formado 
     pelo eixo y, a reta tangente e a reta normal em $T$.
     Considere um rectângulo com uns dos lados paralelos
     ao eixo y. Escreva, a àrea do triângulo em função 
     do comprimento $x$ da base e calcule a àrea do rectangulo 
     com a maior àrea possível. 
     {\it Rpta: } A àrea em função de $x$ é $Area_{\Delta}(x)=x(10-2x)$, 
     $x \in (0,5)$.
     O máximo acontece quando $x=5/2$ e $Area_{\Delta}(5/2)=12,5 u^2$.
     \item ** Se o vértice de uma parabóla $\mathcal{P}$ 
     é $V=(-3,1)$, sua reta diretriz é paralela a 
     $r: 3x+4y-6=0$ e uns dos extremos do lado reto é $(6,3)$. 
     Encontre a equação da parabóla. 
     {\it Rpta: } $p=7$, $\mathcal{P}: 16x^2-24xy+9y^2-300x-650y-475=0$.
     \item A entrada duma igreja tem a forma 
     duma parabóla de 9 m. de altura e 12 m. de base. 
     Toda a parte superior é uma janela de vidro cuja 
     base é paralela à base da entrada
     e tem um comprimento de 8 m. 
     Qual a altura máxima da janela?
     {\it Rpta: } altura=4m.
   \end{enumerate}

\end{document}    
\section*{Equação do plano} 

{\bf Equação vetorial.} 
 Um plano $\pi$ em $\mathbb{R}^{n}$, pode ser escrita como 
 $\pi: P=P_{0}+tV+sW$, $t, s \in \mathbb{R}$ 
 onde $V, W$ são linearmente independentes. 
 Note que existe infinitos vetores $V$ e $W$ que geram o mesmo plano. 
 Se conhecemos três pontos sobre a reta, por exemplo $P_{0}$, $P_1$
 e $P_2$, os vetores 
 $\overrightarrow{P_0P_1}, \overrightarrow{P_0P_2} \in \mathbb{R}^{n}$ 
 servem como geradores do plano $\pi$, se
 $\overrightarrow{P_0P_1}, \overrightarrow{P_0P_2}$ são linearmente independentes ({\it por que?}). 
 Assim, qualquer vetor paralelo a $\pi$ pode ser escrito como combinação linear de $V$ e $W$.
 
 {\bf Equação normal do plano em $\mathbb{R}^{3}$}
 Quando o plano está em $\mathbb{R}^{3}$, podemos escrever o plano da 
 forma 
$$ax+by+cz+d=0, \text{ para certos } a,b,c, d \in \mathbb{R}, $$ 
com $a^{2}+b^{2}+c^2\neq0$. Dita forma se chama de 
 {\it equação geral do plano ou equação normal}. Perceba que o vetor 
 $(a,b,c) \in \mathbb{R}^{2}$ é um vetor normal ao plano. 
 
 Se $V$ e $W$ são vetores paralelos ao plano, $V \times W$ serve como 
 vetor normal ao plano.
 
 Veja que se $(x_0, y_{0}, z_0)$ está sobre o plano $\pi$ 
 $\pi: ax+by+cz+d=0$, temos que 
 $$ (a,b,c) \perp ((x,y,z)-(x_0,y_0,z_0)), \text{ para todo } (x,y,z) \in \pi. $$ 

Em $\mathbb{R}^{3}$, considere dois plano $\pi_1$ e $\pi_2$ com $\pi_1 \cap \pi_2 \neq \emptyset$.
  Certamente, $\pi_1 \cap \pi_2$ é uma reta ({\it por que?}). Podemos facilmente encontrar um {\it vetor diretor} calculando 
  $N_1 \times N_2$, onde 
  $N_1$ e $N_2$ são vetores normais ao planos $\pi_1$ e $\pi_2$ respectivamente. 
  
  {\bf Ângulo entre planos. } Em $\mathbb{R}^{3}$, podemos definir o ângulo entre dois planos $\pi_1$ e $\pi_2$ 
 como o ângulo que satisfaz a relação 
  $$\cos (\pi_1,\pi_2)= \frac{|N_1 \circ N_2|}{\|N_1\|\|N_2\|}, $$
  onde  $N_1$ e $N_2$ são vetores normais ao planos $\pi_1$ e $\pi_2$ respectivamente. Observe que na formula
   usamos o {\it valor absoluto} de  $N_1 \circ N_2$ em lugar de 
   $N_1 \circ N_2$.
   
 {\bf Distância de um ponto a um plano em $\mathbb{R}^{3}$}.
 Considere um plano $\pi: ax+by+cz+d=0$ e um ponto 
 $P=(x_{1}, y_{1}, z_1) \in \mathbb{R}^{3}$. 
 A distância de ponto $P$ ao plano $\pi$ é dado pela formula
 
 \begin{equation}
 \text{dist}(P,\pi)= \|\text{proj}_{N}\overrightarrow{P_0P}\|, 
 \label{eqn:distanciapp2}
 \end{equation}
 onde $P_0$ é um ponto em $\pi$ e $N$ é um vetor normal ao plano.
 Note que podemos usar $N=(a,b,c)$.
 
 
\begin{enumerate}
  \item Faça um esboço dos seguintes planos em $\mathbb{R}^{3}$.
     \begin{enumerate}
     \item $2x+3y+5z-1=0$
     \item $3y+2z-1=0$
     \item $2x+3z-1=0$
     \item $3x+2y-4=0$
     \end{enumerate}  
   \item Encontre a equação geral do plano paralelo 
   a $2x-y+5z-3=0$ e passa no ponto $P=(1,-2,1)$.
   {\it Rpta} $\pi: 2x-y+5z-9=0$.
   \item Ache a equação do plano que contem 
   $P=(2,1,5)$ e é perpendicular aos planos 
    $x+2y-3z+2=0$ e $2x-y+4z-1=0$.
    {\it Rpta} $x-2y-z+5=0$. {\it Dica} Use o produto vetorial.
    \item Considere as retas 
    $$ r: \frac{x-2}{2}=\frac{y}{2}=z \text{ e }
       s:  x-2=y=z.$$
     Obtenha a equação geral do plano determinado por 
     $r$ e $s$.  
     \item Sejam dois planos 
      $\pi_1: x+2y+z+2=0$ e $x-y+z-1=0$.
      Encontre o plano que contém 
      a interseção $\pi_1 \cap \pi_2$ e 
      é ortogonal ao vetor $U=(1, 1, 1)$.
      {\it Rpta: } $x-2y+z-2=0$.
      \item Encontre a equação do plano 
      que passa por $A=(1,0,-2)$ e contém 
      $\pi_1 \cap \pi_2$, onde 
      $\pi_1: x+y-z=0$ e $\pi_2: 2x-y+3z-1=0$.
      \item Qual é a equação paramétrica da reta 
      que é interseção dos planos, 
      $\pi_1: (x,y,z)=(1+\alpha, -2, -\alpha-\beta)$ e 
      $\pi_2: (x,y,z)=(1+\alpha-\beta, 2\alpha+\beta, 3-\beta)$?
      \item Sejam três vetores 
      $V=i+3j+2k$, $W=2i-j+k$ e $U=i-2j$ em $\mathbb{R}^{3}$. 
      Se $\pi$ é um plano paralelo aos vetores 
      $W$ e $U$, e $r$ é uma reta perpendicular ao plano $\pi$.
      Encontre a projeção ortogonal de $V$ 
      sobre o vetor diretor da reta $r$.  
      \item Em $\mathbb{R}^{3}$, considere os pontos 
      $A=(2,-2,4)$ e $B=(8,6,2)$.
      Encontre o lugar geométrico dos pontos equidistantes de $A$ e $B$. {\it Dica :} É um plano.
      
      \item Seja $\pi$ um plano que forma 
      um ângulo de $60^{\circ}$ com o plano 
      $\pi_1: x+z=0$ e contém a reta
      $r: x-2y+2z=0, 3x-5y+7z=0$.
      Encontre a equação do plano $\pi$.
      {\it Rpta} Dois soluções: 
      $y+z=0$ ou $4x-11y+5z=0$.      
      %$x-2y+2z=0$ ou $x+4y+8z=0$. 
      \item  O plano $\pi: x+y-z-2=0$ intercepta os 
      eixos cartesianos aos pontos $A$, $B$ e $C$. 
      Qual é a área do triângulo $ABC$? {\it Rpta} $2\sqrt{3} u^2$.
      \item Considere os planos: 
      $$ \pi_1: x-y+z+1=0, \ \ \pi_2: x+y-z-1=0, \ \ 
         \pi_3: x+y+2z-2=0. $$
      Encontre a equação geral que contém $\pi_1\cap \pi_2$ e perpendicular 
      $\pi_3$.   
      \item Ache o ângulo entre o plano $-2x+y-z=0$ e plano que passa por
      $P=(1,2,3)$ e é perpendicular $a i-2j+k$. {\it Rpta:} $arccos(5/6)$.
      \item Para quais valores de $\alpha$ e $\beta$, a reta 
      $r: (\beta, 2, 0)+t(2, \alpha, \alpha)$ está contida no plano 
      $\pi: x-3y+z=1$. {\it Rpta: } $\alpha=1$, $\beta=7$.
      \item Encontre o valor de $\alpha$ para que 
      os planos $\pi_1: (1,1,0)+t(\alpha, 1, 1)+s(1,1, \alpha)$
      e $\pi_2: 2x+3y+2z+1=0$ sejam paralelos. {\it Rpta: }
      $\alpha=1/2$.
      \item Encontre a equação geral do plano 
      $\pi$ que contém a reta $r: (1,0,1)+t(1,1,-1)$ e dista 
      $\sqrt{2}$
      do ponto $P=(1,1,-1)$. 
\end{enumerate}

\end{document}

