%% Prova 3 de Geometria Analítica
\documentclass[11pt]{exam}
\usepackage[utf8]{inputenc}
\usepackage[T1]{fontenc}
\usepackage[brazilian]{babel}
\usepackage[left=2cm,right=2cm,top=1cm,bottom=2cm]{geometry}
\usepackage{amsmath,amsfonts}
\usepackage{multicol}
%\usepackage{../../../disciplinas}
\usepackage{tikz}
\everymath{\displaystyle}
\def\answers % uncomment to show the answers

\boxedpoints
\pointname{}
\qformat{{\bf Questão \thequestion} \dotfill \fbox{\totalpoints} }

\begin{document}

\ifdefined\answers
\printanswers
\fi

\addpoints

\begin{center}
  {\bf \large Geometria Analítica : Prova 3 } \\
 22 de junho de 2017
\end{center}

\ifx\undefined\answers
\settabletotalpoints{100}
\cellwidth{0pt}
\hqword{Q:}
\hpword{P:}
\hsword{N:}

\makebox[\textwidth]{
  Nome: \enspace\hrulefill\quad
  \gradetable[h][questions]}
\fi

\begin{center}
  \begin{tabular}{|l|}
    \hline
    {\bf Orientações gerais}\\
    1) As soluções devem conter o desenvolvimento e ou justificativa. \\
    \hspace{2.5mm} Questões sem justificativa ou sem raciocínio lógico coerente 
    não pontuam. \\
    2) A interpretação das questões é parte importante do processo de avaliação.\\
    \hspace{2.5mm} Organização e capricho também serão avaliados. \\
    3) Não é permitido a consulta nem a comunicação entre alunos.\\
   \hline 
  \end{tabular}
\end{center}

 \begin{questions}
 \question Encontre as coordenadas 
 dos vértices e dos focos das seguintes equações:
   \begin{parts}
   \part[10] $4x^2+169y^2=676$. Esboce
     \begin{solution}
     Multiplicando adequadamente, temos a seguinte equação 
     $x^{2}/13^{2}+y^{2}/2^{2}=1$. Isto é uma elipse com centro na origem cujo eixo focal é o eixo x. Segue da equação que $a=13$, $b=2$ e $c=\sqrt{165}$. 
     Temos que os vértices são $V=(0,0)\pm a(1,0)=(\pm 13, 0)$ e 
     os focos são $F=(0,0)\pm c(1,0)=(\pm \sqrt{165}, 0)$.
     \end{solution}
   \part[10] $x^2+2x-4y+9=0$. Esboce 
   \begin{solution}
     Dita equação foi analisada em aula. 
     Completando quadrados temos a seguinte equação 
     $(x+1)^{2}=4(y-2)$. Isto é uma parábola com vértice em 
     $(-1,2)$ e cujo eixo focal é o eixo $y$. 
     Segue da equação que $p=1$. Assim, o foco é  
     os focos são $F=(-1,2)+p(0,1)=(-1, 3)$.
     \end{solution}
   \part[15] $x^2-2y^2-4x-4y-1=0$. Esboce.
     \begin{solution}
     Completando quadrados temos a seguinte equação 
     $(x-2)^{2}/3-2(y+1)^{2}/3=1$. Isto é uma hipérbole com centro 
     $(2,-1)$ e cujo eixo focal é o eixo x. Segue da equação que $a=\sqrt{3}$, $b=\sqrt{3/2}$ e $c=3\sqrt{2}/2$. 
     Temos que os vértices são $V=(2,-1)\pm a(1,0)=(2\pm \sqrt{3}, -1)$ e 
     os focos são $F=(2,-1)\pm c(1,0)=(2\pm 3\sqrt{2}/2,-1)$.
     \end{solution}
   \end{parts}    
   \question[15] Seja $\mathcal{P}$ uma parábola com reta diretriz $\mathcal{D}: x-2=0$. Se o vértice está sobre a
 reta $r_1: 3x-2y=19$ e o foco está sobre a reta
 $r_2: x+4y=0$, encontre a equação da parábola. 
  \begin{solution} Feito em aula. Resposta: $(y+2)^{2}=12(x-5)$.
  \end{solution}
 \question[15] Ache a equação reduzida da hipérbole 
 cuja distância focal é $2\sqrt{5}$, 
 os focos pertencem ao eixo $x$ 
 e uma das assíntotas é a reta $r: x+3y=0$. 
   \begin{solution}
   Dos dados do problema, $c=\sqrt{5}$. Como $r: x+3y=0$ é uma assíntota
   que intercepta o eixo $x$ (eixo focal) na origem, temos que o centro da hipérbole é a origem , assim a equação da hiperbole é da forma 
    $x^{2}/a^{2}-y^{2}/b^{2}=1$. 
    
    Procedemos a encontrar $a$ e $b$. 
    Como r é uma assintota (fazendo um desenho por exemplo)
    temos que $b/a=1/3$. {\it Cuidado: $b/a=-1/3$ está errado. a, b são SEMPRE positivos, são distâncias}.
    
    De $a=3b$, $c=\sqrt{5}$ e de 
    $c^{2}=a^{2}+b^{2}$. Temos que $a^{2}=9/2$ e
     $b^{2}=1/2$. Assim, a equação da hipérbole é $2x^{2}/9-2y^{2}=1$.
     {\it Exercício similar foi resolvido na aula}.  
   \end{solution}
 % {\it Rpta: } $-2x^2+(2/9)y^2=1$ 
    \question Considere uma hipérbole $\mathcal{H}$ com
     excentricidade é $e=2$ e cujos vértices  
     são $(2,5)$ e $(0,-1)$. 
      \begin{parts}
      \part[10] Encontre os focos da hipérbole;
       \begin{solution}
       Claramente, $e=2$, $a=\sqrt{10}$ e $c=2\sqrt{10}$.
       Além disso, o centro é $(1,2)$ (ponto meio do segmento que ume os vértices). Observe que como os focos ESTÂO sobre a mesma reta que contem os vértices podemos usar um vetor diretor dessa reta para calular os focos. 
       {\it Use vetores tal como vc aprendeu na primeira parte da disciplina, né?}. 
       
       Um vetor diretor da reta é $(1,3)/\sqrt{10}$. Portanto, os focos são  
       $$F=(1,2)\pm c(1,3)/\sqrt{10}=(1,2)\pm 2\sqrt{10}(1,3)/\sqrt{10}$$
       Calculando, os focos são $(3,8)$ e $(-1,-4)$.
       \end{solution}
      \part[10] Ache as equações das retas diretrizes associada à hipérbole
         \begin{solution}
         A reta diretriz é uma reta perpendicular ao eixo focal cuja
         distância ao centro é $a/e$. Assim, como conhecemos o eixo focal, conhecemos também um vetor diretor de dita reta. Para calcular o ponto onde a reta deve passar, de novo, vamos 
       {\it usar vetores tal como vc aprendeu na primeira parte da disciplina}. 
       
       Um vetor diretor da reta diretriz é $(-3,1)$. Portanto, os pontos
       $P$ por onde passam as diretrizes são   
       $$P=(1,2)\pm (a/e)(1,3)/\sqrt{10}=(1,2)\pm (\sqrt{10}/2)(1,3)/\sqrt{10}=(1,2)\pm(1/2,3/2). $$
       Calculando, os pontos são $(3/2,7/2)$ e $(1/2,1/2)$.
       Logo, as retas diretrizes são 
       $r_{1}; (3/2,7/2)+t(-3,1), t \in \mathbb{R}$ 
       (cuja equação em forma geral é 
       $r_1: x+3y-12=0$) e 
       $r_{2}; (1/2,1/2)+t(-3,1), t \in \mathbb{R}$ 
       (cuja equação em forma geral é 
       $r_2: x+3y-2=0$). 
       Você tambem poderia achar direitamente 
       as retas na forma geral, ambas respostas são certas.  
       \end{solution}      
      \end{parts}
 \question[20] Seja $\mathcal{E}$ uma elipse com focos em 
 $F_1=(2,2)$ e $F_2=(8,2)$. A equação de uma reta tangente à elipse 
 é $r: x+2y-21=0$. Encontre o perímetro do triângulo cujos vértices são os focos da elipse e o correspondente ponto de tangência. 
   \begin{solution}
   Usando a definição de elipse é fácil ver que 
   o perimetro do triângulo é  $2a+2c$. Assim, nosso objetivo é calcular 
   $a$ e $c$. Já que conhecemos os focos temos que $c=3$. Além disso, dos focos temos que o centro é $C=(5,2)$ e o eixo focal é paralelo ao eixo $x$.
    
   Com essas informações temos que a equação da elipse deve ser 
   da forma $(x-5)^{2}/a^2+(y-2)^{2}/b^2=1$. Fazendo $x'=x-5$, $y'=y-2$. 
   A equação se reduz a  $(x')^{2}/a^2+(y')^{2}/b^2=1$.
   {\it Já que a equação está na forma reduzida podemos usar a formula da reta tangente}. Seja $(x'_0,y'_0)$ ponto de tangencia, assim da formula temos que 
   \begin{equation}\label{eqn:1}
     r: (b^2x'_{0})x'+(a^2 y'_0)y'-a^2b^2=0 
   \end{equation}      
   Por outro lado sabemos que a reta tangente é $x+2y-21=0$, mas {\it NÃO podemos comparar ambas tangentes porque estão em coordenadas diferentes}.
   Re-escreva  $x+2y-21=0$ usando coordendas x' e y'. 
   Portanto  $x+2y-21=(5+x')+2(2+y')-21=x'+2y'-12$. Assim, obtemos que 
   \begin{equation}\label{eqn:2}
     r: x'+2y'-12=0
   \end{equation}
     Como as retas \eqref{eqn:1} e \eqref{eqn:2} representam a mesma reta, elas devem ser multiplos. Assim, existe um número $k \in \mathbb{R}$
     tal que 
     $$  b^{2}x'_{0}=k , \ \  a^{2}y'_{0}=2k  \  \ \text{ e }  \  \ a^2b^{2}=12 k $$
     Assim, $x'_{0}=a^2/12$ e $y'_{0}=b^2/6.$ Como $(x'_0, y'_0)$ está sobre a elipse, ele satisfaz $(x')^{2}/a^2+(y')^{2}/b^2=1$. Substituindo, obtemos que $a^2+4b^2=4.36$. Agora, usando o fato que $c=3$ e 
     $a^{2}=c^{2}+b^{2}=9+b^2$ temos que 
     $$ a^2+4b^2=4.36 \Rightarrow a^{2}+4(a^{2}-9)=4.36 \Rightarrow 5a^{2}=5.36 \Rightarrow a^{2}=36
     \Rightarrow a=6.$$
     
     Para terminar observe que o perimetro do triângulo é 
    $2a+2c=2(6)+2(3)=12+6=18$.
    \end{solution}
 \end{questions}
 
 {\bf Formulas: }
 {\it Retas tangentes }
 
 {\it Quando $y^2=4px$}. 
 A reta tangente à $\mathcal{P}$ no ponto $P=(x_0,y_0) \in \mathcal{P}$ é dada por $ r: y_{0}y=2p(x_0+x)$.
  
 {\it Quando $b^2 x^2+a^2y^2=a^2b^2$}.
  
 A reta tangente da $\mathcal{E}$ no ponto $P=(x_0,y_0) \in \mathcal{E}$ é dada por 
 $ r: (b^2x_{0})x+(a^2 y_0)y=a^2b^2$. 
 
 {\it Quando $b^2 x^2-a^2y^2=a^2b^2$}. 
 
 A reta tangente da $\mathcal{H}$ no ponto $P=(x_0,y_0) \in \mathcal{H}$ é dada por $ r: (b^2 x_{0})x-(a^2 y_0)y=a^2b^2$.
 
 
 
\end{document}