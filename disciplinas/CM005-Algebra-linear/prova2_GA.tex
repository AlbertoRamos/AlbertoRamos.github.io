%% Prova 2 de Geometria Analítica
\documentclass[11pt]{exam}
\usepackage[utf8]{inputenc}
\usepackage[T1]{fontenc}
\usepackage[brazilian]{babel}
\usepackage[left=2cm,right=2cm,top=1cm,bottom=2cm]{geometry}
\usepackage{amsmath,amsfonts}
\usepackage{multicol}
%\usepackage{../../../disciplinas}
\usepackage{tikz}
\everymath{\displaystyle}
\def\answers % uncomment to show the answers

\boxedpoints
\pointname{}
\qformat{{\bf Questão \thequestion} \dotfill \fbox{\totalpoints} }

\begin{document}
	
	\ifdefined\answers
	\printanswers
	\fi
	
	\addpoints
	
	\begin{center}
		{\bf \large Geometria Analítica : Prova 2 } \\
		18 de maio de 2017
	\end{center}
	
	\ifx\undefined\answers
	%\settabletotalpoints{100}
	\cellwidth{0pt}
	\hqword{Q:}
	\hpword{P:}
	\hsword{N:}
	
	\makebox[\textwidth]{
		Nome: \enspace\hrulefill\quad
		\gradetable[h][questions]}
	\fi
	
	\begin{center}
		\begin{tabular}{|l|}
			\hline
			{\bf Orientações gerais}\\
			1) As soluções devem conter o desenvolvimento e ou justificativa. \\
			\hspace{2.5mm} Questões sem justificativa ou sem raciocínio lógico coerente 
			não pontuam. \\
			2) A interpretação das questões é parte importante do processo de avaliação.\\
			\hspace{2.5mm} Organização e capricho também serão avaliados. \\
			3) Não é permitido a consulta nem a comunicação entre alunos.\\
			\hline 
		\end{tabular}
	\end{center}
	
	\begin{questions}
		\question[20] Ache o ângulo entre o plano 
		$\pi_1: 2x-y+z=0$ e o plano $\pi_2$ que contem
		uma reta com vetor diretor $U=(1,1,1)$
		e é perpendicular ao vetor  $\hat{i}-2\hat{j}+\hat{k}$.
		\begin{solution}
			Como  $N_2:=\hat{i}-2\hat{j}+\hat{k}$ é normal ao plano $\pi_2$
			temos 
			$$
			\cos(\pi_1,\pi_2)=
			\frac{|N_{2}.(2,-1,1)|}{\|N_2\|\|(2,-1,1)\|}=
			\frac{|(1,-2,1).(2,-1,1)|}{\|(1,-2,1)\|\|(2,-1,1)\|}=
			\frac{5}{6}.
			$$ 
		\end{solution}
		\question[20] Encontre a equação geral do plano
		$\pi$ que contém a
		reta $r: (0,0,1)+(t,-t,-t), t \in \mathbb{R}$, 
		equidista de $P=(1,0,0)$ e $Q=(0,1,0)$, 
		e separa $P$ e $Q$.
		\begin{solution}
			Queremos achar um plano $\pi: ax+by+cz+d=0$ que satisfaz as condições
			requeridas. Claramente, $N=(a,b,c)$ é um vetor normal a $\pi$.
			\begin{enumerate}
				\item De $r \in \pi$, temos que :
				\begin{enumerate}
					\item $(0,0,1) \in r \subset \pi$ e assim $c+d=0$
					\item Como $(1,-1,1) // \pi$ temos que 
					$(1,-1,1)\perp N$ e assim $a-b+c=0$
				\end{enumerate}
				\item Sabe-se que 
				$$
				dist(P,\pi)=\frac{|a+d|}{\sqrt{a^{2}+b^2+c^2}} \text{ e }
				dist(Q,\pi)=\frac{|b+d|}{\sqrt{a^{2}+b^2+c^2}}.
				$$ 
				Já que 
				$dist(Q,\pi)=dist(P,\pi)$. Temos que $|a+d|=|b+d|$ qual implica que 
				$(a+d)^{2}-(b+d)^2=0$. Logo, $(a-b)(a+b+2d)=0$.
			\end{enumerate}
			Dos item anteriores tem-se que $a-b=0$ ou $a+b+2d=0$.
			\begin{enumerate}
				\item Caso $a-b=0$. Neste, caso usando (b) 
				temos que $c=0$ e de (a) que $d=0$. 
				Assim $\pi: ax+ay+0z+0=0$, com $a\neq0$, ou equivalentemente 
				$\pi: x+y=0$. Mas esse plano não separa $P$ e $Q$ porque $x+y$ tem o mesmo sinal quando é avaliado em $P$ e em $Q$.  
				\item Caso $a+b+2d=0$. De (a) $d=-c$, substuindo e usando (b) tem-se 
				$c=\frac{2}{3}a$, $b=a-c=\frac{1}{3}a$ e $d=-c=-\frac{2}{3}a$.
				Logo, $\pi: ax+\frac{1}{3}ay+\frac{2}{3}az-\frac{2}{3}a=0$, com $a\neq0$, ou equivalentemente 
				$\pi: 3x+y+2z-2=0$. Note que 
				 $3(1)+(0)+2(0)-2=1>0$ e 
				 $3(0)+(1)+2(0)-2=-1<0$. Assim, $P$ e $Q$ são separado por 
				 $\pi: 3x+y+2z-2=0$. 
			\end{enumerate}       
		\end{solution}
		\begin{solution}
			{\it Outra forma}. 
			%Temos dois casos. Se P e Q estão e.
			Como $P$ e $Q$ são equidistantes temos que $R:=(P+Q)/2$
			pertence a $\pi$. Calculando temos que  $R=(1/2,1/2,0)$. Já que 
			$A=(0,0,1) \in \pi$, o vetor $\overrightarrow{AR}:=(1/2,1/2,-1)$
			é paralelo a $\pi$. 
			Assim $\overrightarrow{AR}\times (1,-1,-1)=(3/2,1/2,1)$ $//$ $(3,1,2)$
			é um normal ao plano $\pi$. 
			Então, $\pi:=3x+y+2z+d=0$. Para calcular $d$, usamos 
			que $(0,0,1) \in \pi$,  o que implica que $d=-2$. Portanto, o plano 
			é $\pi:=3x+y+2z-2=0$
		\end{solution}
		\question[20] Seja $\mathcal{C}$ uma circunferência de centro 
		$C=(-2,3)$ e raio $\sqrt{5}$. Encontre a equação da reta tangente à circunferência que passa por $P=(3,3)$. 
		\begin{solution}
			Seja $T=(x,y)$ ponto de tangência. 
			\begin{enumerate}
				\item Como $T$ é ponto de tangência. 
				$\overrightarrow{TC} \perp \overrightarrow{TP}$ e
				assim $(-2-x, 3-y).(3-x,3-y)=0$. Portanto, $(x+2)(x-3)+(3-y)^{2}=0$. 
				\item Já que $T$ está na circunferência, tem-se $\|(-2-x, 3-y)\|=\sqrt{5}$. Assim, $(2+x)^{2}+(3-y)^{2}=5$.
			\end{enumerate}
			Restando (2)-(1), $(x+2)^{2}-(x+2)(x-3)=0$. Assim, $(x+2)(x+2-(x-3))=5$
			e daí, $x=-1$. De (2) temos que $1+(3-y)^{2}=5$, então $y=1$ ou $y=5$.
			
			Assim, os pontos de tangencia são $T_1:=(-1,1)$ e $T_{2}:=(-1,5)$.
			As retas desejadas (em forma vetorial) são 
			$r_1:=(3,3)+t\overrightarrow{T_{1}P}=(3,3)+t(4,2)$
			e     $r_2:=(3,3)+t\overrightarrow{T_2P}=(3,3)+t(4,-2)$. A equação geral seria 
			$r_{1}:=-2x+4y-6=0$ e  
			$r_{2}:=2x+4y-18=0$. 
		\end{solution}
		\question Considere o ponto $P=(1,0,1)$ e o plano $\pi: x-2y+4z=1$.
		\begin{parts}
			\part[10] Determine as coordenadas da projeção ortogonal 
			do $P$ e plano $\pi$
			\begin{solution}
				Calculemos a 
				$dist(P,\pi)=\frac{|1+4-1|}{\sqrt{1^{2}+(-2)^2+4^2}}=\frac{4}{\sqrt{21}}$.
				Como $P$ está acima do plano $\pi$ (considerando a direção da normal $N=(1,-2,4)$), temos que se $Q$ é a projeção ortogonal de P sobre $\pi$, então   $$ \overrightarrow{QP}= dist(P,\pi) \frac{N}{\|N\|}, $$
				calculando temos que     	
				$$ Q=P- \frac{4}{\sqrt{21}}\frac{N}{\|N\|}=(1,0,1)-\frac{4}{\sqrt{21}}\frac{(1,-2,4)}{\sqrt{21}}=(1,0,1)-\frac{4}{21}(1,-2,4)=
				(\frac{17}{21}, \frac{8}{21}, \frac{5}{21}).$$
			\end{solution}
			\part[5] Encontre as coordenadas do {\it ponto simétrico} a $P$ em relação ao plano $\pi$
			\begin{solution}
				O ponto simétrico $\widehat{P}$ satisfaz  a relação %temos que 
				$$ \overrightarrow{\widehat{P}P}= 2dist(P,\pi) \frac{N}{\|N\|}. $$     
				Assim, podemos calcular $\widehat{P}$ como 
				$$ \widehat{P}=P- 2\frac{4}{\sqrt{21}}\frac{N}{\|N\|}=
				(\frac{13}{21}, \frac{16}{21}, -\frac{11}{21}). $$
			\end{solution}
		\end{parts}	
		\question Considere duas retas 
		$r: (1,0,2)+(2t,t,3t), t \in \mathbb{R}$ e
		$s: (0,1,-1)+(t,\alpha t, 2\alpha t), t \in \mathbb{R}$.
		\begin{parts}
			\part[10] Determine o valor de $\alpha$ para que as retas sejam coplanares.
			\begin{solution}
				Como elas são retas reversas (não são paralelas), para que elas sejam coplanares a distância entre elas deve ser zero.  Assim, 
				$$dist(r,s)=\frac{|\overrightarrow{P_sP_r}.V_{s}\times V_r|}{\|V_{s}\times V_{r}\|}=0, $$
				onde $P_s=(0,1,-1)$, $P_r=(1,0,2)$, 
				$V_{s}=(1,\alpha,2\alpha)$ e 
				$V_{r}=(2,1,3)$.
				
				Calculando o produto misto (usando determinantes) temos que 
				$\overrightarrow{P_sP_r}.V_{s}\times V_r=0$ implica que $\alpha=2/3$.
			\end{solution}
			\part[3] Para o valor de $\alpha$ encontrado, determine a posição relativa entre $s$ e $r$ 
			\begin{solution}
				Como $V_s=(2,1,3)$ e $V_{r}=(1,2/3,4/3)$ não são paralelas elas são concorrentes 
			\end{solution}
			\part[2] Determine a equação do plano determinado por 
			$r$ e $s$
			\begin{solution}
				O plano $\pi$ dever ser $P_{0}+tV_s+sV_r$, com $t,s \in \mathbb{R}$. 
				Devemos achar o ponto $P_0$. Como as retas já são coplanares (para esse valor de $\alpha$) podemos pegar $P_0=(0,1,-1)$ (ou $P_0=(1,0,2)$).
			\end{solution}
		\end{parts}
		\question[10] Considere um tetraedro com vértices 
		$O=(0,0,0)$, $A=(1,0,0)$, $B=(0,2,0)$ e $C=(0,0,3)$.
		Encontre a equação geral do plano que dista $2/7$ 
		da face ABC e intercepta o tetraedro.
		\begin{solution}
			Primeiro calculemos o plano $\pi_{ABC}$ que contem a face ABC.
			Um vetor normal a $\pi_{ABC}$ é $\overrightarrow{AB} \times \overrightarrow{AC}=(6,3,2)$.
			Assim, como plano  $\pi_{ABC}$ é $6x+3y+2z+d=0$ para algum $d$. 
			Já que $A=(1,0,0) \in \pi_{ABC}$, temos que $d=-6$.
			
			Se temos um plano $\pi$ que dista 2/7 de $\pi_{ABC}$, 
			o plano $\pi$ deve ser $\pi:=6x+3y+2z+\widehat{d}=0$, 
			para algum $\widehat{d}$. Calculemos $\widehat{d}$. 
			Já que 
			$$dist(\pi,\pi_{ABC})=
			\frac{|\widehat{d}-(-6)|}{\sqrt{6^2+3^{2}+2^2}}=
			\frac{|\widehat{d}+6|}{\sqrt{49}}=
			\frac{|\widehat{d}+6|}{7}=\frac{2}{7}.$$
			Da expressão anterior temos que $\widehat{d}=-4$ ou 
			$\widehat{d}=-8$. Assim, existe dois planos 
			$6x+3y+2z=4$
			e 
			$6x+3y+2z=8$. Já que 
			o plano que contem a face ABC é 
			$6x+3y+2z=6$, só o plano $6x+3y+2z=4$ intercepta o tetraedro.
			Assim, a resposta é  $6x+3y+2z=4$.
		\end{solution}
	\end{questions}
\end{document}